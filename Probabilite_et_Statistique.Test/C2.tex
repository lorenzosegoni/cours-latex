\chapter{Tests de comparaison moyennes/mediannes pour 2 échantillons}

\begin{obs}
\begin{itemize}
\item $(X_1,...,X_{n_1})$ $n_1$-echantillion d'une $1^{ere}$ population de loi $P$
\item $(Y_1,...,Y_{n_2})$ $n_2$-echantillion d'une $2^{eme}$ popuation de loi $Q$
\end{itemize}
\end{obs}

Soit $\mu_x$ l'esperance de $P$ et $\mu_y$ l'esperance de $Q$

On veut tester: 

\[ \mathcal{H}_0 : \mu_x = \mu_y \quad \text{contre} \quad \mathcal{H}_1 : \mu_x \neq \mu_y \]

ou

\[ \mathcal{H}_0 : Med(P) = Med(Q) \quad \text{contre} \quad \mathcal{H}_1 : Med(P) \neq Med(Q) \]


\section{Données appariées}

Soit :
\begin{itemize}
\item $n_1 = n_2 = n $ 
\item Les valeurs $X_i$ et $Y_i$ correspondent au même individu $i$
\end{itemize}

\begin{exemple}
Pression artelliele de patients avant/apres une prise de médicament
\end{exemple}

On se ramene au cadre de test sur 1 echantillon en considerant $D_i = X_i - Y_i$

On teste:

\[ \mathcal{H}_0 : \mu_d = 0 \quad \text{contre} \quad \mathcal{H}_1 : \mu_d \neq 0 \]

où $\mu_D$ l'esperance de $D$, donc $\mu_D = \mu_x - \mu_u$

ou

\[ \mathcal{H}_0 : Med(D) = Med(X-Y) = 0 \quad \text{contre} \quad \mathcal{H}_1 : Med(D) \neq 0 \]

\begin{commentaire}
\textbf{Attention : } $Med(X-Y) \neq Med(X) - Med(Y)$
\end{commentaire}

\begin{itemize}
\item Test de Student
\item Test du signe
\item Test de Wilconson des rangs signes
\end{itemize}

\begin{commentaire}
\textbf{Sur} \texttt{R} : 
\begin{itemize}
\item Soit on indique \texttt{paired = T} et on donne les $2$ echantillon
\item Soit on utilise le tests sur $1$ Echantillon avec les $D_i$
\end{itemize}
\end{commentaire}

\section{Données non appariées}

\subsection{Test de Student / Welch}

On note $\sigma^2_X = \mathbb{V}[P]$ et $\sigma^2_Y = \mathbb{V}[Q]$.

$\frac{\sqrt{n}(\bar{X}_{n_1} - \bar{Y}_{n_2} }{\hat{\sigma}}$ aura un bon comportement sous $\mathcal{H}_0$

\begin{exemple}
\begin{itemize}
\item $P =\mathcal{N}(\mu_x , \sigma^2)$
\item $Q =\mathcal{N}(\mu_y , \sigma^2)$
\end{itemize}

On a alors: $\bar{X}_{n_1} - \bar{Y}_{n_2} \sim \mathcal{N}(\mu_x - \mu_y , \frac{\sigma^2}{n_1} + \frac{\sigma^2}{n_2} )$

On a \[ \hat{\sigma}^2_p = \frac{(n_1 - 1) \hat{\sigma}_x^2 + (n_2-1) \hat{\sigma}^2_Y}{n_1 + n_2 -2} = w \hat{\sigma}_X^2 + (1-w) \hat{\sigma}_Y^2 \]

où $w=\frac{n_1 - 1 }{n_1 + n_2 - 2} \in ]0,1[$ 

où $\hat{\sigma}^2_X = \frac{1}{n_1 -1} \sum_{i=1}^n (X_i - \bar{X}_{n1}$ et de même pour $\hat{\sigma}^2_Y$

$\hat{\sigma}^2_p$ estimateurs consistant de $\sigma^2$

Alors

\[ T_{n_1,n_2} = \frac{\bar{X}_{n_1} - \bar{Y}_{n_2}}{\hat{\sigma}_p \sqrt{\frac{1}{n_1} + \frac{1}{n_2}} }\sim \mathcal{C}(n_1 + n_2 -2\]

sous $\mathcal{H}_0$

\subsubsection{Cas generale}

Soit $n = (n_1, n_2)$. $n \to \infty$ signifie $n_1 \to \infty$ et $n_2 \to \infty$.

Soit $$ et $$ "statistique de Welch. Alors

\[ T_N \xrightarrow{loi sous \mathcal{H_0}} \mathcal{N}(0,1)\]

On a aussi lorsque $\hat{\sigma}^2-n = \frac{\hat{\sigma}^2_x}{n_1} + \frac{\hat{\sigma}^2_y}{n_2}$ et $T_n = \frac{}{}$

\[ loi(T_n) \approx{sous \mathcal{H}_0} \mathcal{X} (\ni) \]

où $\ni$ explicite (formule compliquée).

\begin{commentaire}
\textbf{Sur } \texttt{R} : \texttt{t.test(x,y)}
\end{commentaire}

\begin{remarque}
\begin{itemize}
\item Si on indique égalité des variances, test de Student
\item En pratique, on ne sait jamais si le variances sont egales ou non. On utilise toujours le test de Welch. (meme dans le cas d'egalité de variances
\end{itemize}
\end{remarque}

\subsection{Test de Wilsoxn-Mann-Whitney}

On veut tester

\[ \mathcal{H}_0 : Med(X-Y) = 0 \quad \mathcal{H}_1 : Med(X-Y) \neq 0 \]

On suppose $P$ et $Q$ ne sont differentes que par une translation.

\textbf{Statistique:}

Soit $C = (X_1 , ... , X_{n_1}, Y_1  , ... , Y_{n_2} ) $. Soit $R(X,C)$ le vexteur contenant les rangs associés à $X$ dans $C$. De meme pour $R(Y,C)$.

On definit:

\[ W_\alpha = \sum_{i=1}^{n_1} R_i(X,C)\]

\begin{exemple}
Soit:
\begin{itemize}
\item $x = (4,6,2)$
\item $y = (1,3,8,9)$
\item $c = (4,6,2,1,3,8,9)$
\end{itemize}

Alors $r(x) = (4,5,2,1,3,6,7)$. Donc:

\[ w_X = 4 + 5 + 2 = 11 \]
\end{exemple}

Autre aprroche

\[ U_n = \sum_{i=1}^{n_1} \sum_{j=1}^{n_2} \mathbb{1}_{X_i} > Y_j \]

On a

\[ U_n = W_X - \frac{n_1 ( n_1 +1)}{n_2}\]

\textbf{Résultat :} Sous $\mathcal{H_0}$, la loi de $W_X$ est connue et ne dépend pas de $P$ ni de $Q$. On a aussi un resultat asymptotique lorsque $n \to \infty$

\begin{commentaire}
\textbf{Sur} \texttt{R} : \texttt{wilcox.test(x,y)}
\end{commentaire}

\textbf{Resultat :} sans l'hypothese "$P$ et $Q$ sont efales à une translation pres" le test de WilconxonMW test

\[ 
\begin{cases}
\mathcal{H}_0 : P = Q \\
\mathcal{H}_1 : P \neq Q
\end{cases}
\]

On peut le corriger asymptotiquement pour qu'il teste correctement $Med(X-Y) = 0$ contre $Med(X-Y) \neq 0$ (Voir exercice 4 TD2)





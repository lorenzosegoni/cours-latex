\chapter{Le Modèle SIS}

\section{Introuduction}

Le modèle épidémiologique \textit{SIS} (pour \textbf{Susceptible–Infecté–Susceptible}) décrit la propagation d’une maladie infectieuse dans une population fermée, sous l’hypothèse que les individus infectés peuvent redevenir susceptibles après leur guérison. 

\bigskip

\begin{hypothese}
Les hypothèses principales du modèle sont :
\begin{enumerate}
  \item La population totale est \textbf{constante} et vaut $N$, avec $N$ \textbf{grand}. Elle se divise en deux compartiments :
  \begin{itemize}
  \item $S(t) : \text{nombre d’individus \textit{susceptibles} à l’instant } t,$
  \item $I(t) : \text{nombre d’individus \textit{infectés} à l’instant } t, $
  \end{itemize}
  avec $S(t) + I(t) = N$ pour tout $t \geq 0$.
  \item La \textit{transmission de l’infection est proportionnelle aux contacts entre susceptibles et infectés}, avec un taux $\lambda > 0$.
  \item Les \textit{individus infectés guérissent et redeviennent susceptibles}, avec un taux de guérison $\mu > 0$.
\end{enumerate}
\end{hypothese}

\bigskip

Ce modèle ne procure pas d'immunité (exemple : la \textbf{tuberculose}).

\section{Le modèle}

On a ce modèle:

\begin{figure}[h!]
\centering
\includegraphics[width=0.5\textwidth]{SIS/Image/modele_sis.jpg}
\caption{Schéma modéle SIS}
\end{figure}

\newpage

\begin{definition}
Le système d'équations différentielles est :

\begin{equation}
\label{eq:sis_system}
\begin{cases}
        \frac{dS(t)}{dt} = -\lambda SI + \gamma I, \\
        \frac{dI(t)}{dt} = \lambda SI - \gamma I  \\
        S(0) + I(0) = S_0 + I_0 = N
\end{cases}
\end{equation}

où 

\begin{itemize}
\item $\lambda$ est le \textit{taux d'infection} (ou de contact) (\textbf{par unité de temps et par individu}). 
\item $\gamma$ représente le \textit{taux de guérison} \textbf{par unité de temps et par individu infectieux}.
\end{itemize}
\end{definition}

Puisque il y a pas de changement de population (population fermé) et on a pas de naissance et de mort, on a que $S+I$ est constante et égale à $N$


\begin{proposition}
En remplaçant $S$ par $N-I$ le système différentiel (\ref{eq:sis_system}) se réduit à l'équation différentielle suivante :

\begin{equation}
    \frac{dI(t)}{dt} = \lambda \left( \frac{ \lambda N - \gamma}{\lambda} - I \right) I , \quad \text{avec } I(0) = I_0.
    \label{eq:sis_reduced}
\end{equation}

Là encore, (\ref{eq:sis_reduced}) est une équation différentielle logistique si le terme $\lambda N / \gamma - 1 \neq 0$, donc
\begin{align*}
    I(t) =
    \begin{cases}
        \dfrac{\lambda I_0 e^{(\lambda N - \gamma) t}}{I_0 (\lambda N - \gamma) ( e^{(\lambda N - \gamma) t} - 1 ) + (\lambda N - \gamma)} & \text{si } \lambda N - \gamma \neq 0, \\
        \dfrac{I_0}{1 + \lambda I_0 t} & \text{sinon.}
    \end{cases}
    \label{eq:sis_solution}
\end{align*}

\end{proposition}

\begin{proof}

\textbf{Réduction à une équation pour $I(t)$.} 

En remplaçant $S$ par $N - I$ dans le système différentiel (\ref{eq:sis_system}), on obtient :

\begin{align*}
\frac{dI(t)}{dt} = I'(t) &= \lambda S(t) I(t) - \gamma I(t) \\
&= \lambda (N - I(t)) I(t) - \gamma I(t) \\
&= \lambda N I(t) - \lambda I^2(t) - \gamma I(t) \\
&= (\lambda N - \gamma) I(t) - \lambda I^2(t) \\
&= \lambda \left( \frac{\lambda N - \gamma}{\lambda} - I(t) \right) I(t)
\end{align*}

Posons $K = \frac{\lambda N - \gamma}{\lambda}$. L'équation se réécrit :

\[ \boxed{ I'(t) = \lambda (K - I(t)) I(t),} \quad \text{avec } I(0) = I_0 \]

\bigskip
\textbf{Cas 1 : $\lambda N - \gamma \neq 0$ (équation logistique)}

Dans ce cas, $K \neq 0$ et l'équation est une équation différentielle de type logistique. Nous allons la résoudre par séparation des variables.

\smallskip
\textit{Séparation des variables.} L'équation se réécrit :

\[ \frac{dI}{I(K - I)} = \lambda \, dt \]

\smallskip

\textit{Décomposition en éléments simples.} Pour intégrer le membre de gauche, décomposons en éléments simples :

\[ \frac{1}{I(K - I)} = \frac{A}{I} + \frac{B}{K - I} \]

En multipliant par $I(K - I)$ :

\[ 1 = A(K - I) + BI \]

En identifiant les coefficients :
\begin{itemize}
\item Pour $I = 0$ : $1 = AK$, donc $A = \frac{1}{K}$
\item Pour $I = K$ : $1 = BK$, donc $B = \frac{1}{K}$
\end{itemize}

D'où :

\[ \frac{1}{I(K - I)} = \frac{1}{K} \left( \frac{1}{I} + \frac{1}{K - I} \right) \]

\smallskip
\textit{Intégration.} En intégrant des deux côtés :

\begin{align*}
\int \frac{1}{K} \left( \frac{1}{I} + \frac{1}{K - I} \right) dI &= \int \lambda \, dt \\
\frac{1}{K} \left[ \ln|I| - \ln|K - I| \right] &= \lambda t + C_1 \\
\frac{1}{K} \ln\left|\frac{I}{K - I}\right| &= \lambda t + C_1 \\
\ln\left|\frac{I}{K - I}\right| &= K\lambda t + C_2
\end{align*}

où $C_2 = KC_1$ est une nouvelle constante d'intégration.

En prenant l'exponentielle :

\[ \frac{I}{K - I} = C e^{K\lambda t} \]

où $C = e^{C_2}$ est une constante positive.

\smallskip
\textit{Résolution pour $I(t)$.} En résolvant pour $I$ :

\begin{align*}
I &= C(K - I) e^{K\lambda t} \\
I &= CK e^{K\lambda t} - CI e^{K\lambda t} \\
I + CI e^{K\lambda t} &= CK e^{K\lambda t} \\
I(1 + C e^{K\lambda t}) &= CK e^{K\lambda t} \\
I(t) &= \frac{CK e^{K\lambda t}}{1 + C e^{K\lambda t}}
\end{align*}

\smallskip
\textit{Détermination de la constante $C$.} À $t = 0$ :

\[ I_0 = \frac{CK}{1 + C} \]

En résolvant pour $C$ :

\begin{align*}
I_0(1 + C) &= CK \\
I_0 + I_0 C &= CK \\
I_0 &= CK - I_0 C \\
I_0 &= C(K - I_0) \\
C &= \frac{I_0}{K - I_0}
\end{align*}

\smallskip
\textit{Expression finale.} En substituant $C$ et $K = \frac{\lambda N - \gamma}{\lambda}$ :

\begin{align*}
I(t) &= \frac{\frac{I_0}{K - I_0} \cdot K e^{K\lambda t}}{1 + \frac{I_0}{K - I_0} e^{K\lambda t}} \\
&= \frac{I_0 K e^{K\lambda t}}{K - I_0 + I_0 e^{K\lambda t}} \\
&= \frac{K I_0 e^{K\lambda t}}{(K - I_0)(1 - e^{K\lambda t}) + K}
\end{align*}

En remplaçant $K\lambda = \lambda N - \gamma$ :

\begin{align*}
I(t) &= \frac{\frac{\lambda N - \gamma}{\lambda} \cdot I_0 e^{(\lambda N - \gamma) t}}{\left(\frac{\lambda N - \gamma}{\lambda} - I_0\right)(1 - e^{(\lambda N - \gamma) t}) + \frac{\lambda N - \gamma}{\lambda}}
\end{align*}

En multipliant numérateur et dénominateur par $\lambda$ et en réarrangeant :

\[ I(t) = \frac{(\lambda N - \gamma) I_0 e^{(\lambda N - \gamma) t}}{I_0 (\lambda N - \gamma) (e^{(\lambda N - \gamma) t} - 1) + (\lambda N - \gamma)} \]

Cette expression peut aussi s'écrire en factorisant :

\[ I(t) = \frac{\lambda I_0 e^{(\lambda N - \gamma) t}}{I_0 (\lambda N - \gamma) (e^{(\lambda N - \gamma) t} - 1) + (\lambda N - \gamma)} \]

\bigskip
\textbf{Cas 2 : $\lambda N - \gamma = 0$ (cas dégénéré)}

Dans ce cas, $K = 0$ et l'équation d'avant devient :

\[ I'(t) = -\lambda I^2(t) \]

\smallskip
\textit{Changement de variable.} Posons $Y(t) = \frac{1}{I(t)}$, qui est bien défini car $I(t) > 0$ pour tout $t \geq 0$.

En dérivant :

\begin{align*}
Y'(t) &= -\frac{I'(t)}{I^2(t)} \\
&= -\frac{-\lambda I^2(t)}{I^2(t)} \\
&= \lambda
\end{align*}

Ceci est une équation différentielle élémentaire dont la solution générale est :

\[ Y(t) = \lambda t + C \]

\smallskip
\textit{Détermination de la constante.} À $t = 0$ :

\[ Y(0) = \frac{1}{I_0} = C \]

Donc :

\[ Y(t) = \lambda t + \frac{1}{I_0} \]

\smallskip
\textit{Expression de $I(t)$.} En revenant à $I(t) = \frac{1}{Y(t)}$ :

\[ I(t) = \frac{1}{\lambda t + \frac{1}{I_0}} = \frac{I_0}{1 + \lambda I_0 t} \]

\bigskip
\textbf{Conclusion}

En combinant les deux cas, on obtient la solution complète :

\[ I(t) =
\begin{cases}
\dfrac{\lambda I_0 e^{(\lambda N - \gamma) t}}{I_0 (\lambda N - \gamma) (e^{(\lambda N - \gamma) t} - 1) + (\lambda N - \gamma)} & \text{si } \lambda N - \gamma \neq 0, \\[0.5em]
\dfrac{I_0}{1 + \lambda I_0 t} & \text{si } \lambda N - \gamma = 0.
\end{cases} \]

\end{proof}

\begin{commentaire}
Lorsque $t$ tend vers $+\infty$, 

\[ I(t) \to \begin{cases}
        N - \frac{\gamma}{\lambda} & \text{si } \lambda N - \gamma > 0,\\
        0 & \text{sinon.}
    \end{cases}
\]
\end{commentaire}


\begin{proof}
Les points d'équilibre du système SIS satisfont $I'(t) = 0$, soit :
\[ \lambda(N - I)I - \gamma I = 0 \quad \Leftrightarrow \quad I[\lambda(N - I) - \gamma] = 0 \]

Donc soit $I^* = 0$, soit $I^* = N - \frac{\gamma}{\lambda}$.

\bigskip
\textbf{Cas 1 : $\lambda N - \gamma > 0$}

Dans ce cas, $I^* = N - \frac{\gamma}{\lambda} > 0$ est un équilibre positif. 

De la solution explicite, en divisant numérateur et dénominateur par $e^{(\lambda N - \gamma)t}$ :
\[ I(t) = \frac{\lambda I_0}{I_0(\lambda N - \gamma)\left(1 - e^{-(\lambda N - \gamma)t}\right) + (\lambda N - \gamma)e^{-(\lambda N - \gamma)t}} \]

Lorsque $t \to +\infty$, comme $\lambda N - \gamma > 0$, on a $e^{-(\lambda N - \gamma)t} \to 0$, d'où :

\begin{align*}
 \lim_{t \to +\infty} I(t) &= \frac{\lambda I_0}{I_0(\lambda N - \gamma)} \\
 &= \frac{\lambda}{\lambda N - \gamma} \\
 &= \frac{\lambda N - \gamma + \gamma}{\lambda(\lambda N - \gamma)} \cdot \lambda \\
 &= N - \frac{\gamma}{\lambda} 
 \end{align*}

\bigskip
\textbf{Cas 2 : $\lambda N - \gamma \leq 0$}

Si $\lambda N - \gamma < 0$, alors $e^{(\lambda N - \gamma)t} \to 0$ quand $t \to +\infty$. Le numérateur de $I(t)$ tend vers $0$ tandis que le dénominateur reste strictement positif, donc $I(t) \to 0$.

Si $\lambda N - \gamma = 0$, alors $I(t) = \frac{I_0}{1 + \lambda I_0 t} \to 0$ quand $t \to +\infty$.

\end{proof}

\bigskip

\begin{notation}
Soit $\mathcal{R}_0 = \dfrac{\lambda N}{\gamma}$. On appelle $\mathcal{R}_0$ le \textbf{nombre de reproduction de base}. Alors 

\[ I(t) \xrightarrow{ t \to + \infty } \begin{cases}
        N - \frac{\gamma}{\lambda} & \text{si } \mathcal{R}_0 > 1,\\
        0 & \text{si } \mathcal{R}_0 \leq 1.
    \end{cases}
\]


Les conditions $\mathcal{R}_0 > 1$ et $\mathcal{R}_0 \leq 1$ conduisent à des comportements différents du modèle. Il y a donc \textbf{disparition de l'infection} si $\mathcal{R}_0 \leq 1$ et \textbf{maintien d'un état endémique} dans le cas contraire.
\end{notation}

\section{Recherche des points d'équilibre du modèle}

\begin{definition}
On obtient les deux points d'équilibre suivants :

\[ (S^*_1 = N , I^*_1 = 0) \quad \text{et} \quad (S^*_2 = \frac{\gamma}{\lambda}, I^*_2 = N - \frac{\gamma}{\lambda}) \]
\end{definition}

\begin{itemize}
\item Le point $(S^*_1, I^*_1)$ est un point d'équilibre d'\textbf{absence d'épidémie} car il n'y a pas de propagation de l'épidémie s'il n'y a pas au moins un individu infectieux.
\item Le second point $(S^*_2, I^*_2)$ est le point d'équilibre (\textbf{endémique}) vers lequel la trajectoire va tendre lorsque $\mathcal{R}_0 > 1$. C'est donc un \textit{équilibre asymptotiquement stable}.
\end{itemize}


\begin{figure}[h!]
\centering
\includegraphics[width=0.45\textwidth]{SIS/Image/equi_sis_grd.jpg}
\hfill
\includegraphics[width=0.45\textwidth]{SIS/Image/equi_sis_pet.jpg}
\caption{Comportement asymptotique du modèle SIS. À gauche : cas $\lambda N > \gamma$ (régime endémique). À droite : cas $\lambda N < \gamma$ (régime d'extinction).}
\label{fig:equilibres_sis}
\end{figure}
\smallskip

\newpage

\section{Interprétation de la condition $\mathcal{R}_0 > 1$}

$\mathcal{R}_0 > 1$ indique que chaque personne infectieuse doit avoir au moins un contact infectant (qui puisse transmettre l'infection) pendant sa période moyenne d'infectiosité pour que la maladie se maintienne à un \textbf{niveau endémique}.

\paragraph{Le déstin des individus initialement infectieux} Pour tout $t \geq 0$, $J(t)$ est le nombre d'individus \textbf{initialement infectieux} à l'instant $t$ \textbf{et qui le sont encore à l'instant} $t$. Ainsi $J(t)$ vérifie l'équation différentielle

\begin{equation}
    \frac{dJ(t)}{dt} = -\gamma J(t), \quad J(0) = I_0.
    \label{eq:Jt}
\end{equation}

Donc, pour tout $t \geq 0$, $J(t) = I_0 e^{-\gamma t}$ et $e^{-\gamma t}$ approche une probabilité (approche fréquentiste) : la probabilité qu'un individu initialement infecté le soit toujours à l'instant $t$.

\bigskip
\paragraph{Loi de la durée d'infection} Soit $T$ la variable aléatoire \textbf{durée d'infection}, c.-à-d. « le temps entre le début et la fin de l'infection pour un individu malade ». Alors l'événement $(T > t)$ signifie que l'individu est encore malade au temps $t$, donc $P(T > t) = e^{-\gamma t}$.

En d'autres termes, on voit que $T$ suit une loi exponentielle d'espérance $\dfrac{1}{\gamma}$ et on peut donc interpréter $\dfrac{1}{\gamma}$ comme une « \textbf{période moyenne d'infectiosité} ».

\bigskip
\paragraph{Autre variables importantes}

\begin{itemize}
\item $\lambda N$ est le nombre moyen de contacts infectants, par unité de temps causés par un individu infectieux.
\item 
$ \mathcal{R}_0 = \frac{\lambda N}{\gamma} $ est le nombre moyen de contacts infectants effectués par un individu infectieux durant sa période d'infectiosité. $\mathcal{R}_0$ est appelé \textbf{nombre de reproduction de base}.
\end{itemize}

\section{Représentation dans le plan de la trajectoire}

\begin{figure}[h!]
\centering
\includegraphics[width=0.6\textwidth]{SIS/Image/trajectoire_sis.jpg}
%\caption{Schéma modéle SI}
\end{figure}

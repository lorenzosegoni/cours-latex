\chapter{Le Modèle \textit{SI}}

\section{Introduction}

L'infection se propage par \textbf{contact} entre les membres d'une communauté mais au travers de laquelle, il n'y a pas de «retrait» (morts, isolés, ...).

\smallskip

\textbf{À la fin, tous les individus susceptibles deviennent infectés.}

\smallskip

Ces hypothèses sont réductrices dans la plupart des situations mais peuvent-être approximativement appliquées dans les situations où :

\begin{hypothese}
\begin{enumerate}
\item la maladie est \textbf{hautement infectieuse} mais \textbf{pas} \textit{suffisamment grave} pour les cas de retraits par mort ou isolation,
\item \textbf{aucun infecté ne guérit} de la maladie durant la partie majeure de l'épidémie,
\end{enumerate}
\end{hypothese}

C'est la cas de certaines infections peu graves de l'appareil respiratoire.

Il faut remarqué que pour toute maladie, le \textbf{modèle SI est vrai en court terme}.

\section{Le modèle}

On peut rapresenter ce problème de cette façon:

\begin{figure}[h!]
\centering
\includegraphics[width=0.45\textwidth]{SI/Image/modele_si.jpg}
\caption{Schéma modéle SI}
\end{figure}

\newpage

\begin{defi}
On peut donc en déduire le système différentielle;

\begin{equation}
\label{eq:si_system}
\begin{cases}
\frac{dS(t)}{dt} = -\lambda SI, \\
\frac{dI(t)}{dt} = \lambda SI \\
\forall t , \quad S(t)+I(t)=N
\end{cases}
\end{equation}
\end{defi}

à cause de l'infection, $S$ décroît et $I$ croît. Le paramètre $\lambda$ est la constante de proportionnalité, c'est le taux d'infection (ou de contact) (par unité de temps et par individu).

\bigskip

\begin{hypothese}
On prend \textbf{l'hypothèse d'une grande population}.
\end{hypothese}

\smallskip

On sait que pour tout t, $N = S(t)+I(t)$ (c'est à dire une population constante), car on a un même flux entrant et sortant.

\bigskip

\begin{defi}

Pour résoudre ce système différentielle, on va donc le réduire en une seule équation différentielle:

\begin{equation}
\boxed{ \frac{dI(t)}{dt} = \lambda ( N - I(t) )I(t) }
    \label{eq:si_equa_diff}
\end{equation}

La résolution de ce modèle nous donne:

\begin{equation}
    \forall t \geq 0, \quad \boxed{ I(t) =\frac{N I_0}{I_0 + S_0 e^{-\lambda N t}} }= \frac{I_0 N e^{\lambda N t}}{S_0 + I_0 e^{\lambda N t}}
    \label{eq:solution_si_I(t)}
\end{equation}

et

\begin{equation}
    \forall t \geq 0, \quad \boxed{ S(t) =\frac{N S_0}{S_0+I_0 e^{\lambda N t}} }
    \label{eq:solution_si_S(t)}
\end{equation}

\end{defi}

\begin{proof}

\textbf{Existence et unicité de la solution.} On suppose que les hypothèses du théorème de Cauchy–Lipschitz sont satisfaites (continuité et condition de Lipschitz sur les fonctions du système).  

Ainsi, pour toute condition initiale donnée, il existe une solution unique \((S(t),I(t))\) définie sur \([0,+\infty[\).

\bigskip

\textbf{Passage du discret au continu.} En discrétisant le système avec un pas $\Delta t > 0$, on écrit

\[ \Delta S(t) = S(t+\Delta t)-S(t) = -\lambda S(t) I(t) \Delta t. \]

Sous l’hypothèse de grande population (paramètre $\lambda$ petit devant $S(t)$), on identifie

\[ \frac{\Delta S(t)}{\Delta t} \approx \frac{dS}{dt}. \]

De même pour $I$, ce qui justifie l’écriture du système différentiel continu.

\bigskip

\textbf{Réduction à une équation pour $I(t)$.} Comme $S(t)+I(t)=N$ est constant, $\forall t \geq 0$

\begin{align*}
\frac{dI(t)}{dt} = I'(t) &= \lambda S(t)\, I(t) \\
&= \lambda \bigl(N - I(t)\bigr) I(t)  \\
&= - \lambda I^2(t) + \lambda N I(t) \\
&=  \boxed{ \lambda ( N - I(t) )I(t).}
\end{align*}

\bigskip

\textbf{Changement de variable (linéarisation).} On pose:

\[ Y(t) = \frac{1}{I(t)}. \]

Ceci est bien défini car $I(t)>0$ pour tout $t\geq 0$ (en effet $I(0)=I_0>0$ et $I'(t)\geq 0$).  

On différencie, en remplaçant $I'(t)$, on obtient :

\begin{align*}
Y'(t) = \left( \frac{1}{I(t)} \right)' &= -\frac{I'(t)}{I^2(t)} \\
&= \frac{\lambda I^2(t)}{I^2(t)} -  \frac{\lambda N I(t)}{I^2(t)}\\
&= \lambda - \lambda N \frac{1}{I(t)} \\
&= -\lambda N Y(t) + \lambda.
\end{align*}

On a obtenu l’équation différentielle linéaire suivante :

\[ Y'(t) = -\lambda N Y(t) + \lambda. \]

Cette équation est de la forme générale

\[ Y'(t) + a Y(t) = b, \]

où $a = \lambda N$ et $b = \lambda$ sont des constantes positives.

\bigskip

\textbf{Résolution de l’équation homogène}

On commence par résoudre l’équation homogène associée, c’est-à-dire en prenant $b=0$ :

\[ Y_h'(t) = -\lambda N Y_h(t). \]

C’est une équation différentielle linéaire du premier ordre à coefficients constants.  

La méthode standard donne :

\[ Y_h(t) = K e^{-\lambda N t}, \qquad K \in \mathbb{R}. \]

\textbf{Recherche d’une solution particulière}

On cherche maintenant une solution particulière de l’équation complète

\[ Y'(t) = -\lambda N Y(t) + \lambda. \]

La méthode du \textbf{facteur intégrant} consiste à multiplier l’équation par $e^{\lambda N t}$ afin de transformer le membre de gauche en une dérivée d’un produit.  

En effet :

\[ \bigl( Y(t) e^{\lambda N t} \bigr)' = Y'(t) e^{\lambda N t} + \lambda N Y(t) e^{\lambda N t}. \]

Or, l’équation initiale donne

\[ Y'(t) + \lambda N Y(t) = \lambda, \]

donc en multipliant par $e^{\lambda N t}$ :

\[ \bigl( Y(t) e^{\lambda N t} \bigr)' = \lambda e^{\lambda N t}. \]

\smallskip

On peut maintenant intégrer des deux côtés :

\begin{align*}
Y(t) e^{\lambda N t} &= \int \lambda e^{\lambda N t}\,dt \\
&= \frac{\lambda}{\lambda N} e^{\lambda N t} + C \\
&= \frac{1}{N} e^{\lambda N t} + C. 
\end{align*}

En divisant par $e^{\lambda N t}$, on obtient la solution générale :
\[
Y(t) = \frac{1}{N} + C e^{-\lambda N t}.
\]

\bigskip

\textbf{Utilisation de la condition initiale.} À $t=0$, on a

\[ Y(0) = Y_0 = \frac{1}{I_0} \]

Et avec l'expression obtenu avant on a $Y(0)= \frac{1}{N} + C$. Donc on obtient

\[ \frac{1}{I_0} = \frac{1}{N} + C \quad \Longrightarrow \quad
C = \frac{1}{I_0} - \frac{1}{N} = \frac{N-I_0}{N I_0} = \frac{S_0}{N I_0}. \]

Ainsi

\[ Y(t) = \frac{1}{N} + \frac{S_0}{N I_0} e^{-\lambda N t}. \]

\bigskip

\textbf{Expression finale de la solution} En revenant à $I(t)=1/Y(t)$ :

\[ \boxed{ I(t) = \frac{N I_0}{I_0 + S_0 e^{-\lambda N t}}. } \]

En multipliant numérateur et dénominateur par $e^{\lambda N t}$ :

\[ \boxed{ I(t) = \frac{N I_0 e^{\lambda N t}}{S_0 + I_0 e^{\lambda N t}}. } \]

\bigskip

\textbf{Expression explicite de $S(t)$.} En utilisant la conservation de la population $S(t)+I(t)=N$, on obtient

\[ S(t)=N-I(t)=N-\frac{N I_0 e^{\lambda N t}}{S_0+I_0 e^{\lambda N t}}. \]

En mettant au même dénominateur :

\[ S(t)=\frac{N\bigl(S_0+I_0 e^{\lambda N t}\bigr)-N I_0 e^{\lambda N t}}{S_0+I_0 e^{\lambda N t}}= \boxed{ \frac{N S_0}{S_0+I_0 e^{\lambda N t}}. } \]

\end{proof}

\commentaire{
\textbf{ Comportement asymptotique }

\medskip

Quand $t \to +\infty$, $e^{-\lambda N t} \to 0$ donc

\[ I(t) =\frac{N I_0}{I_0 + S_0 e^{-\lambda N t}}  \xrightarrow{t \to \infty} \frac{N I_0}{I_0} = N. \]

À long terme, toute la population est infectée, ce qui correspond aux hypothèses du modèle SI.
}


\section{Points d'équilibre}

\begin{defi}
Un \textbf{point d'équilibre} du système (\ref{eq:si_system}) est un point $(S^*, I^*)$ particulier de ce système, donc une solution du système (\ref{eq:si_system}) obtenue en posant $\frac{S'(t)}{dt} = 0$ et $\frac{I'(t)}{dt} = 0$.

\bigskip

Donc on a:

\[ \boxed{
\begin{cases}
\frac{S'(t)}{dt} = 0 \\
\frac{I'(t)}{dt} = 0
\end{cases}
\Leftrightarrow
\begin{cases}
I'(t) = \lambda S(t)I(t), \\
\lambda S(t)I(t) = 0
\end{cases}
\rightsquigarrow (S^*, I^*) \left( \sim (S_\infty,I_\infty) \right) }\]
\end{defi}


C'est un point où il n'y a \textbf{ni augmentation, ni diminution} (\textit{pas de variations au cours du temps}). Si on part de ce point, le système reste dans cet état. On dit aussi que la trajectoire (ensemble des couples $(S(t), I(t) \geq 0$) qui est ici une \textit{courbe plane est constante}.

\bigskip

La recherche des équilibres est très importante, car dans tous les cas que l'on va examiner dans ce cours, les trajectoires du système vont tendre (asymptotiquement, $t \to +\infty$) vers l'un des points d'équilibre, et celui-ci sera dit \textbf{asymptotiquement stable}.

\subsection{Recherche des points d'équilibre pour le modèle $SI$}

On obtient les deux points d'équilibre suivants :

\[ \boxed{ (S^*_1 = N, I^*_1 = 0) \quad \text{et} \quad (S^*_2 = 0, I^*_2 = N) }\]


\begin{itemize}
\item Le point $(S^*_1, I^*_1)$ est appelé point de \textbf{non épidémie} (ou \textbf{d'absence d'épidémie}) car il n'y a pas de propagation de l'épidémie s'il n'y a pas au moins un individu infectieux.
\item Le second point $(S^*_2, I^*_2)$ est justement le point vers lequel la trajectoire va tendre, on dit que ce point d'équilibre est \textbf{asymptotiquement stable}.
\end{itemize}

















\section{Représentation dans le plan de la trajectoire}

\begin{figure}[h]
\centering
\includegraphics[width=0.5\textwidth]{SI/Image/trajectoire_si.jpg}
\caption{Représentation du modèle SI dans le plan $(S,I)$.}
\end{figure}

\newpage

La figure ci-dessus illustre les trajectoires possibles du modèle \textit{SI} dans le plan des variables $(S,I)$.

\bigskip

L’équation de conservation $S(t) + I(t) = N$ implique que l’évolution du système se fait uniquement sur la \textbf{droite} :

\[ S + I = N. \]

Ainsi, le \textbf{triangle des états possibles} $(S,I) \geq 0$ est réduit à cette droite, ce qui signifie que la dynamique peut être représentée par un \textbf{seul degré de liberté }(par exemple $I(t)$).

\bigskip

Les deux points extrêmes de la droite correspondent aux situations limites :

\begin{itemize}
\item $ (S_1^*, I_1^*) = (N, 0) \quad \text{(aucun infecté, \textit{tout le monde susceptible})}$
\item $ (S_2^*, I_2^*) = (0, N) \quad \text{(\textit{toute la population infectée})}. $
\end{itemize}

\bigskip

Au cours du temps, la trajectoire se déplace de $(S_0, I_0)$ vers $(0, N)$ : 
le nombre de susceptibles décroît continûment tandis que celui des infectés augmente.  
Le \textbf{triangle intérieur} du graphique est vide, car les solutions physiques se trouvent uniquement sur la droite $S+I=N$.

\bigskip

Cette représentation géométrique est utile pour \textit{visualiser la conservation de la population totale} et pour \textit{interpréter graphiquement la dynamique épidémique}.

\section{La courbe d’épidémie : incidence}

En pratique, les données recueillies ne concernent pas directement le nombre de personnes infectées $I(t)$, mais le nombre de \textbf{nouveaux cas d’infections} par unité de temps (jour, semaine, mois, etc.). 

\bigskip

\begin{defi}
On appelle cette quantité \textbf{incidence}, notée $f(t)$, et définie par :

\[ f(t) = \frac{dI(t)}{dt}. \]

On obtient ainsi la \textbf{courbe d’épidémie}, qui décrit l’évolution du nombre de nouvelles infections au cours du temps.
\end{defi}


\bigskip

\begin{prop}
À partir de la solution de $I(t)$, on calcule :

\begin{equation}
    f(t) = \frac{dI(t)}{dt} 
    = \frac{\lambda I_0 S_0 N^2 e^{\lambda N t}}{(S_0 + I_0 e^{\lambda N t})^2}.
    \label{eq:incidence}
\end{equation}
\end{prop}

\begin{proof}
\begin{align*}
I'(t) &= \left( \frac{I_0 N e^{\lambda N t }}{S_0 + I_0 e^{\lambda N t}} \right)' \\
%&= \frac{I_0 N e^{\lambda N t} \cdot I_0 \lambda N e^{\lambda N t} + I_0 N  \lambda N e^{\lambda N t} \cdot (S_0 + I_0 e^{\lambda N t} ) }{ ( S_0 + I_0 e^{\lambda N t} )^2 } \\
&=  \frac{ I_0 N  \lambda N e^{\lambda N t} \cdot (S_0 + I_0 e^{\lambda N t} ) - I_0 N e^{\lambda N t} \cdot I_0 \lambda N e^{\lambda N t}} { ( S_0 + I_0 e^{\lambda N t} )^2 } \\
&= \frac{ \lambda I_0 N^2 e^{\lambda N t} \cdot (S_0 + I_0 e^{\lambda N t} ) - \lambda I_0 N^2 e^{\lambda N t} \cdot I_0 e^{\lambda N t} } { ( S_0 + I_0 e^{\lambda N t} )^2 } \\
&= \frac{ \lambda I_0 N^2 e^{\lambda N t} \cdot (S_0 + I_0 e^{\lambda N t} - I_0 e^{\lambda N t} ) } { ( S_0 + I_0 e^{\lambda N t} )^2 } \\
&=\frac{\lambda I_0 S_0 N^2 e^{\lambda N t}}{(S_0 + I_0 e^{\lambda N t})^2}.
\end{align*}
\end{proof}

\begin{figure}[h!]
\centering
\includegraphics[width=0.6\textwidth]{SI/Image/c_epidemie_si.jpg}
\caption{Courbe d’épidémie obtenue dans le modèle SI.}
\end{figure}

\subsubsection*{Propriétés de la courbe d’épidémie}

\begin{itemize}
    \item La courbe est \textbf{unimodale} : elle présente un \textit{seul maximum} (forme en cloche).
    \item On peut calculer explicitement le temps $t^*$ où \textit{l’incidence est maximale}.  
\end{itemize}

\paragraph{Calcul du maximum.}  
Le maximum est atteint lorsque $f'(t)=0$. Après dérivation et simplification, on trouve que cela se produit lorsque $I(t)=\tfrac{N}{2}$.  

\begin{prop}
En reportant dans l’expression de $I(t)$, on obtient :
\[
t^* = \frac{\log\!\left(\tfrac{S_0}{I_0}\right)}{\lambda N}.
\]

La valeur maximale de l’incidence est alors donnée par :
\[
f(t^*) = \frac{\lambda N^2}{4}, \qquad I(t^*) = \frac{N}{2}.
\]
\end{prop}

\begin{proof}
a
\end{proof}

\paragraph{Interprétation.}  
L’incidence croît rapidement depuis $0$, atteint un maximum au temps $t^*$, puis décroît progressivement vers $0$. Cela traduit le fait que, dans le modèle \textit{SI}, l’épidémie démarre rapidement, atteint un pic, puis finit par infecter l’ensemble de la population à long terme.

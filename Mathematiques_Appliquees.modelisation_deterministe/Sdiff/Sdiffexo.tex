\chapter*{Exercice}

\exo

\begin{enonce}
Chercher tous les points d'équilibre du système d'équations différentielles:

\[
\begin{cases}
 \frac{dx}{dt}=(x-1)(y-1) \\
\frac{dy}{dt}=(x+1)(y+1)
\end{cases}
\]

\end{enonce}

\begin{correction}
On cherche les points d'équilibre 

\[
\begin{cases}
x' = (x-1)(y-1) \\
y' = (x+1)(y+1)
\end{cases}
\]

On a donc au point d'équilibre $P_{eq} = (x_{eq},y_{eq}) \in \mathbb{R}^2$ si et seulement si,

\[
\begin{cases}
(x_{eq}-1)(y_{eq}-1) =0 \\
(x_{eq}+1)(y_{eq}+1) =0
\end{cases}
\]

Donc on obtient que les points possibles sont $(1,-1)$ ou $(-1,1)$.

On vérifie

\[ (1,-1) : \begin{cases}
x' = (1-1)(-1-1) = 0 \\
y' = (1+1)(-1+1) = 0
\end{cases} \quad
(-1,1) : \begin{cases}
x' = (-1-1)(1-1) = 0 \\
y' = (-1+1)(1+1) = 0
\end{cases} 
\]

On a donc les points d'équilibres $P_{eq}^{(1)} = (1,-1)$ et $P_{eq}^{(2)} = (-1,1)$.
\end{correction}



\exo

\begin{enonce}
Soit $A = \begin{pmatrix} 2 & -3 & 0 \\ 0 & -6 & -2 \\ -6 & 0 & -3 \end{pmatrix}$. Montrer que la solution nulle de l'équation différentielle $\dot{x} = Ax$ est instable.
\end{enonce}
\begin{correction}
On a que:

\begin{align*}
det(A - \lambda) &= \begin{vmatrix} 2 - \lambda & -3 & 0 \\ 0 & -6 - \lambda & -2 \\ -6 & 0 & -3 - \lambda \end{vmatrix} \\
&= (2- \lambda) \cdot \begin{vmatrix}  -6 - \lambda & -2 \\ 0 & -3 - \lambda \end{vmatrix}  -(-3)- \cdot \begin{vmatrix} 0 & -2 \\ -6 & -3 - \lambda \end{vmatrix}  + 0 \cdot \begin{vmatrix} 0 & -6 - \lambda \\ -6 & 0 \end{vmatrix}  \\
&= (2 - \lambda) \left[ (6 + \lambda)(3+ \lambda) \right] +3 [-12] \\
&= (2 - \lambda) [18 +9\lambda + \lambda^2] -36 \\
&= -\lambda^3 -7\lambda^2 \\
&= -\lambda^2 (\lambda +7)
\end{align*}

Donc on a $Sp(A)= \{ -7 , 0\}$ avec $0$ de multiplicité $2$.

\end{correction}


\exo

\begin{enonce}
Déterminer pour chacun des systèmes diférentiels suivants, les points d'équilibre ainsi que leur comportement:

\ques

\[
\begin{cases}
x' = xy-y \\
y' = xy-x
\end{cases}
\]

\ques

\[
\begin{cases}
x' = x-xy \\
y' = y-x^2
\end{cases}
\]

\ques

\[
\begin{cases}
x' = 3x-2y \\
y' = 4x+y
\end{cases}
\]

Résoudre le système linéaire.
\end{enonce}

\begin{correction}
\ques On cherche les points d'équilibre 

\[
\begin{cases}
x' = xy-y \\
y' = xy-x
\end{cases}
\]

On a donc au point d'équilibre $P_{eq} = (x_{eq},y_{eq}) \in \mathbb{R}^2$ si et seulement si,

\[
\begin{cases}
x_{eq}y_{eq}-y_{eq} = 0 \\
x_{eq}y_{eq}-x_{eq} = 0
\end{cases}
\Rightarrow
\begin{cases}
y_{eq}(x_{eq}-1) = 0 \\
x_{eq}(y_{eq}-1) = 0
\end{cases}
\]

Donc on obtient que les points possibles sont $(1,0)$ ou $(0,1)$ ou $(0,0)$ ou $(1,1)$

On vérifie

\[ (1,0) : \begin{cases}
x' = 1 \cdot 0 - 0 = 0 \\
y' = 1 \cdot 0 - 1 = -1
\end{cases} \quad
(0,1) : \begin{cases}
x' = 0 \cdot 1 - 1 = -1 \\
y' = 0 \cdot 1 - 0 = 0
\end{cases} 
\]
\[
(0,0) : \begin{cases}
x' = 0 \cdot 0 - 0 = 0 \\
y' = 0 \cdot 0 - 0 = 0
\end{cases} \quad
(1,1) : \begin{cases}
x' = 1 \cdot 1 - 1 = 1 \\
y' = 1 \cdot 1 - 1 = 1
\end{cases} 
\]

On a donc les points d'équilibres $P_{eq}^{(1)} = (0,0)$ et $P_{eq}^{(2)} = (1,1)$.

-----------------------------------------------------------------------------------------------------------

Soit $A(P(x,y))$ la matrice jacobienne du système linéaire au tour du point $P(x,y)$, alors:

\[
A(P(x,y))=
\begin{pmatrix}
\frac{\partial x'}{\partial x} & \frac{\partial x'}{\partial y} \\
\frac{\partial y'}{\partial x} & \frac{\partial y'}{\partial y}
\end{pmatrix}=
\begin{pmatrix}
y & x-1\\
y-1 & x
\end{pmatrix}
\]

\begin{enumerate}
\item On regarde $P_{eq}^{(1)} = (0,0)$.

\[ A_1 = A(P_{eq}^{(1)}) = \begin{pmatrix}
0 & -1 \\
-1 & 0
\end{pmatrix} \]

On a le système d'équilibre $X' = A_1 X$. Le point d'équilibre de ce système est $Q_{eq}^{(1)} = (0,0)$. 

On regarde le spectre de $A_1$:

\[ \chi_{\lambda}(A_1) = A_1 - \lambda I_2 = \lambda^2 - Tr(A_1) \lambda + det(A_1) = \lambda^2 - 0 \lambda -1 =0 \Rightarrow Sp(A_1) = \{ -1 , +1 \} \]

On constate que $\forall \lambda \in Sp(A_1) , \lambda = \{ -1 , +1 \}$, on a $Re(\lambda) \neq 0$ Donc $Q_{eq}^{(1)}$ est hyperbolique pour le $X' = A_1 X$. Par hyperbolité le point d'équilibre $P_{eq}^{(1)}$ est donc hyperbolique.

Pour $Q_{eq}^{(1)}$, on a $\lambda_1 = -1$ et $ \lambda _2 = 1 > 0$, par le théorème de la stabilité des système lineaire permet de qualifier que $Q_{eq}^{(1)}$ est instable et par hyperbolité, on conclut que $P_{eq}^{(1)}$ est instable.
\item On regarde $P_{eq}^{(2)} = (1,1)$.
\[ A_1 = A(P_{eq}^{(2)}) = \begin{pmatrix}
1 & 0 \\
0 & 1
\end{pmatrix} = I_2 \]

On a le système d'équilibre $X' = A_2 X$. Le point d'équilibre de ce système est $Q_{eq}^{(2)} = (0,0)$. 

On regarde le spectre de $A_2$:

\[ \chi_{\lambda}(A_2) = A_2 - \lambda I_2 =0 \Rightarrow Sp(A_2) = \{ +1 \} \quad \text{avec multiplicité $2$} \]

On constate que $\forall \lambda \in Sp(A_2), \lambda = \{ +1 \}$, on a $Re(\lambda) \neq 0$ Donc $Q_{eq}^{(2)}$ est hyperbolique pour le $X' = A_2 X$. Par hyperbolité le point d'équilibre $P_{eq}^{(2)}$ est donc hyperbolique.

Pour $Q_{eq}^{(2)}$, on a $\lambda_1 = 1 >0$ et $\lambda_2 = 1 >0$, par le théorème de la stabilité des système lineaire permet de qualifier que $Q_{eq}^{(2)}$ est instable et par hyperbolité, on conclut que $P_{eq}^{(2)}$ est instable.
\end{enumerate}

\ques On cherche les points d'équilibre 

\[
\begin{cases}
x' = x-xy \\
y' = y-x^2
\end{cases}
\]

On a donc au point d'équilibre $P_{eq} = (x_{eq},y_{eq}) \in \mathbb{R}^2$ si et seulement si,

\[
\begin{cases}
x_{eq}-x_{eq}y_{eq} = 0 \\
y_{eq}-x_{eq}^2 = 0
\end{cases}
\Rightarrow
\begin{cases}
x_{eq}(1- y_{eq}) = 0 \\
y_{eq} = x_{eq}^2
\end{cases}
\]

Donc on obtient que les points possibles sont $(0,0)$ ou $(1,1)$ ou $(-1,1)$ ou $(1,1)$

On vérifie

\[ (0,0) : \begin{cases}
x' = 0 \cdot 0 - 0 = 0 \\
y' = 1  - 0 = 0
\end{cases} \quad
(1,1) : \begin{cases}
x' = 1 \cdot 1 - 1 = 0 \\
y' = 1 - 1 = 0
\end{cases} 
\]
\[
(-1,1) : \begin{cases}
x' = -1 \cdot 1 +1 = 0 \\
y' = 1 - 1 = 0
\end{cases}
\]

On a donc les points d'équilibres $P_{eq}^{(1)} = (0,0)$ et $P_{eq}^{(2)} = (1,1)$ et $P_{eq}^{(3)} = (-1,1)$ .

--------------------------------------------------------------------------------------------------------

Soit $A(P(x,y))$ la matrice jacobienne du système linéaire au tour du point $P(x,y)$, alors:

\[
A(P(x,y))=
\begin{pmatrix}
\frac{\partial x'}{\partial x} & \frac{\partial x'}{\partial y} \\
\frac{\partial y'}{\partial x} & \frac{\partial y'}{\partial y}
\end{pmatrix}=
\begin{pmatrix}
1-y & -x\\
-2x & 1
\end{pmatrix}
\]

\begin{enumerate}
\item On regarde $P_{eq}^{(1)} = (0,0)$.

\[ A_1 = A(P_{eq}^{(1)}) = \begin{pmatrix}
1 & 0 \\
0 & 1
\end{pmatrix} = I_2 \]

On a le système d'équilibre $X' = A_1 X$. Le point d'équilibre de ce système est $Q_{eq}^{(1)} = (0,0)$. 

On regarde le spectre de $A_1$:

\[ \chi_{\lambda}(A_1) = A_1 - \lambda I_2 =0 \Rightarrow Sp(A_1) = \{ +1 \} \quad \text{avec multiplicité $2$} \]

On constate que $\forall \lambda \in Sp(A_1) , \lambda = \{ +1 \}$, on a $Re(\lambda) \neq 0$ Donc $Q_{eq}^{(1)}$ est hyperbolique pour le $X' = A_1 X$. Par hyperbolité le point d'équilibre $P_{eq}^{(1)}$ est donc hyperbolique.

Pour $Q_{eq}^{(1)}$, on a $\lambda_1 = 1 >1$ et $ \lambda _2 = 1 > 0$, par le théorème de la stabilité des système lineaire permet de qualifier que $Q_{eq}^{(1)}$ est instable et par hyperbolité, on conclut que $P_{eq}^{(1)}$ est instable.
\item On regarde $P_{eq}^{(2)} = (1,1)$.
\[ A_1 = A(P_{eq}^{(2)}) = \begin{pmatrix}
0 & -1 \\
-2 & 1
\end{pmatrix} = I_2 \]

On a le système d'équilibre $X' = A_2 X$. Le point d'équilibre de ce système est $Q_{eq}^{(2)} = (0,0)$. 

On regarde le spectre de $A_2$:

\[ \chi_{\lambda}(A_2) = A_2 - \lambda I_2 =\lambda^2 - Tr(A_2) \lambda + det(A_2)= \lambda^2 - \lambda -1 = 0 \Rightarrow Sp(A_2) = \{ -1 , 2  \}  \]

On constate que $\forall \lambda \in Sp(A_2) , \lambda=  \{ -1 , 2  \}$, on a $Re(\lambda) \neq 0$ Donc $Q_{eq}^{(2)}$ est hyperbolique pour le $X' = A_2 X$. Par hyperbolité le point d'équilibre $P_{eq}^{(2)}$ est donc hyperbolique.

Pour $Q_{eq}^{(2)}$, on a $\lambda_1 = -1 < 0$ et $\lambda_2 =  2 >0$, par le théorème de la stabilité des système lineaire permet de qualifier que $Q_{eq}^{(2)}$ est instable et par hyperbolité, on conclut que $P_{eq}^{(2)}$ est instable.




\item On regarde $P_{eq}^{(3)} = (-1,1)$.
\[ A_1 = A(P_{eq}^{(3)}) = \begin{pmatrix}
0 & 1 \\
2 & 1
\end{pmatrix} = I_2 \]

On a le système d'équilibre $X' = A_3 X$. Le point d'équilibre de ce système est $Q_{eq}^{(3)} = (0,0)$. 

On regarde le spectre de $A_3$:

\[ \chi_{\lambda}(A_3) = A_3 - \lambda I_2 =\lambda^2 - Tr(A_3) \lambda + det(A_3)= \lambda^2 - \lambda -2 = 0 \Rightarrow Sp(A_3) = \{ -1,2 \}  \]

On constate que $\forall \lambda \in Sp(A_3) =   \{ -1,2 \} $, on a $Re(\lambda) \neq 0$ Donc $Q_{eq}^{(3)}$ est hyperbolique pour le $X' = A_3 X$. Par hyperbolité le point d'équilibre $P_{eq}^{(3)}$ est donc hyperbolique.

Pour $Q_{eq}^{(3)}$, on a $Re(\lambda_1) = -1 $ et $Re(\lambda_2) =  2 >0$, par le théorème de la stabilité des système lineaire permet de qualifier que $Q_{eq}^{(3)}$ est instable et par hyperbolité, on conclut que $P_{eq}^{(3)}$ est instable.
\end{enumerate}

\ques On cherche les points d'équilibre 

\[
\begin{cases}
x' = 3x-2y \\
y' = 4x+y
\end{cases}
\]

On a donc au point d'équilibre $P_{eq} = (x_{eq},y_{eq}) \in \mathbb{R}^2$ si et seulement si,

\[
\begin{cases}
3x_{eq}-2y_{eq} = 0 \\
4 x_{eq}+y_{eq} = 0
\end{cases}
\Rightarrow
\begin{cases}
x_{eq}= \frac{2}{3} y_{eq} \\
4 \cdot \frac{2}{3} y_{eq} +y_{eq} = \frac{11}{3} y_{eq} = 0
\end{cases}
\]

Donc on obtient que les points possibles sont $(0,0)$ .

On vérifie

\[ (0,0) : \begin{cases}
x' = 3 \cdot 0 - 2 \cdot 0 = 0 \\
y' = 4 \cdot 0  - 1 \cdot 0 = 0
\end{cases} \]

On a donc les points d'équilibres $P_{eq} = (0,0)$ .

---------------------------------------------------------------------------------------------------

Soit $A(P(x,y))$ la matrice jacobienne du système linéaire au tour du point $P(x,y)$, alors:

\[
A(P(x,y))=
\begin{pmatrix}
\frac{\partial x'}{\partial x} & \frac{\partial x'}{\partial y} \\
\frac{\partial y'}{\partial x} & \frac{\partial y'}{\partial y}
\end{pmatrix}=
\begin{pmatrix}
3 & -2 \\
4 & 1
\end{pmatrix} = A
\]

On regarde le spectre:

\[ \chi_{\lambda}(A) = A - \lambda I_2 = \lambda^2 - Tr(A) \lambda + det(A) =  \lambda^2 - 4 \lambda +11  = 0 \]

\[ \Rightarrow Sp(A) = \{ 2-i \sqrt{7} , 2+i\sqrt{7} \}  \]

Puisque $\forall \lambda \in Sp(A) , Re(\lambda) = 2 > 0 $ On a des points d'équilibre hyperbolique instable. 

-----------------------------------------------------------------------------------------------------------

On peut écrire l'equation: 

\[
\begin{cases}
x' = 3x-2y \\
y' = 4x+y
\end{cases} \Rightarrow
X' = \begin{pmatrix}
3 & -2 \\ 
4 & 1
\end{pmatrix}
X = AX \quad \text{ avec } X = \begin{pmatrix} x \\ y \end{pmatrix}
\]

Comme $Sp(A)=\{ 2-i \sqrt{7} , 2+i\sqrt{7} \}$, on a que:


\[ A - Re(\lambda) I_2 = N \Rightarrow N = \begin{pmatrix} 1 & -2 \\ 4 & -1 \end{pmatrix} \]

De plus on remarque que:

\[ N^2 = = \begin{pmatrix} -7 & 0 \\ 0 & -7 \end{pmatrix} = - Im^2(\lambda) I_2 \]

Donc on a:

\begin{align*}
e^{At} &= e^{(2 I_2 + N) t} \\
&= e^{2t} e^{Nt} \\
&= e^{2t} \cdot \left( \cos(\sqrt{7} t) I_2 + \frac{\sin(\sqrt{7} t)}{\sqrt{7}} N \right) \\
&= e^{2t} \begin{pmatrix} \cos(\sqrt{7} t) + \frac{1}{\sqrt{7}} \sin(\sqrt{7} t ) & -\frac{2}{\sqrt{7}} \sin(\sqrt{7} t) \\ 
\frac{4}{\sqrt{7}} \sin(\sqrt{7} t) &  \cos(\sqrt{7} t) - \frac{1}{\sqrt{7}} \sin(\sqrt{7} t) \end{pmatrix}
\end{align*}

Puisque la solution générale est $X(t) = e^{At} X_0$:

\[ \begin{cases}
x(t) = e^{2t} \left( ( \cos(\sqrt{7} t) + \frac{1}{\sqrt{7}} \sin(\sqrt{7} t) ) x_0 - \frac{2}{\sqrt{7}} \sin(\sqrt{7} t) y_0  \right) \\
y(t) = e^{2t} \left( \frac{4}{\sqrt{7}} \sin(\sqrt{7} t) x_0 + ( \cos(\sqrt{7} t) - \frac{1}{\sqrt{7}} \sin(\sqrt{7} t) ) y_0 \right)
\end{cases} \]

\end{correction}


\exo

\begin{enonce}
Soit le système différentiel suivants:

\[
\begin{cases}
x'(t) = x(4-x-2y) \\
y'(t) = y(x-y-1) \\
x(0) \geq 0 \quad , \quad y(0)\geq 0
\end{cases}
\]

\ques Déterminer les ponts d'équilibre de ce système, puis la matrice jacobienne du système linéarité en chacun de ces points.

\ques Pour chacun de ces points d'équilibre, dire s'il est hyperbolique ou non.

\ques Pour chaque point d'équilibre hyperbolique, étudier sa stabilité.

\end{enonce}

\begin{correction}
\ques Détermination des points d'équilibre

Les points d'équilibre sont les solutions du système :

\[
\begin{cases}
x(4-x-2y) = 0 \\
y(x-y-1) = 0
\end{cases}
\]

\begin{itemize}
\item Cas 1 : $x = 0$. La deuxième équation donne : $y(-y-1) = 0$

Soit $y = 0$ ou $y = -1$.Puisque $y(0) \geq 0$, on obtient le point $\boxed{E_1 = (0, 0)}$
\item Cas 2 : $y = 0$ et $x \neq 0$. La première équation donne : $x(4-x) = 0$

Soit $x = 4$ ou $x = 0$. On obtient le point $\boxed{E_2 = (4, 0)}$
\item Cas 3 : $x \neq 0$ et $y \neq 0$.

De la première équation : $4 - x - 2y = 0 \Rightarrow x = 4 - 2y$.

De la deuxième équation : $x - y - 1 = 0 \Rightarrow x = y + 1$.

En égalisant : $4 - 2y = y + 1 \Rightarrow 3 = 3y \Rightarrow y = 1 \Rightarrow x = 1 + 1 = 2$.

On obtient le point $\boxed{E_3 = (2, 1)}$
\end{itemize}

Le système s'écrit sous la forme :

\[
\begin{cases}
f(x,y) = x(4-x-2y) = 4x - x^2 - 2xy \\
g(x,y) = y(x-y-1) = xy - y^2 - y
\end{cases}
\]

Les dérivées partielles sont :

\[ \frac{\partial f}{\partial x} = 4 - 2x - 2y, \quad \frac{\partial f}{\partial y} = -2x \]
\[ \frac{\partial g}{\partial x} = y, \quad \frac{\partial g}{\partial y} = x - 2y - 1 \]

La matrice jacobienne est :

\[ J(x,y) = \begin{pmatrix} 4 - 2x - 2y & -2x \\ y & x - 2y - 1 \end{pmatrix}\]

\begin{itemize}
\item Au point $E_1 = (0, 0)$ :

\[ J(0,0) = \begin{pmatrix} 4 & 0 \\ 0 & -1 \end{pmatrix} \]
\item Au point $E_2 = (4, 0)$ :

\[ J(4,0) = \begin{pmatrix} 4 - 8 - 0 & -8 \\ 0 & 4 - 0 - 1 \end{pmatrix} = \begin{pmatrix} -4 & -8 \\ 0 & 3 \end{pmatrix} \]
\item Au point $E_3 = (2, 1)$ :

\[ J(2,1) = \begin{pmatrix} 4 - 4 - 2 & -4 \\ 1 & 2 - 2 - 1 \end{pmatrix} = \begin{pmatrix} -2 & -4 \\ 1 & -1 \end{pmatrix} \]
\end{itemize}

\ques Un point d'équilibre est hyperbolique si aucune valeur propre de la jacobienne n'a une partie réelle nulle.

\begin{itemize}
\item Point $E_1 = (0, 0)$

Les valeurs propres de $J(0,0) = \begin{pmatrix} 4 & 0 \\ 0 & -1 \end{pmatrix}$ sont :

\[ \lambda_1 = 4, \quad \lambda_2 = -1 \]

Aucune valeur propre n'a une partie réelle nulle, donc $E_1$ est hyperbolique.
\item Point $E_2 = (4, 0)$

Les valeurs propres de $J(4,0) = \begin{pmatrix} -4 & -8 \\ 0 & 3 \end{pmatrix}$ sont :

\[ \lambda_1 = -4, \quad \lambda_2 = 3 \]

Aucune valeur propre n'a une partie réelle nulle. $E_2$ est hyperbolique.
\item Point $E_3 = (2, 1)$. Le polynôme caractéristique de $J(2,1) = \begin{pmatrix} -2 & -4 \\ 1 & -1 \end{pmatrix}$ est $= \lambda^2 + 3\lambda + 6$.

Le discriminant est $\Delta = 9 - 24 = -15 < 0$. Les valeurs propres sont complexes conjuguées :

\[ \lambda = \frac{-3 \pm i\sqrt{15}}{2} \]

La partie réelle est $-\frac{3}{2} \neq 0$. $E_3$ est hyperbolique.
\end{itemize}

\ques

\begin{itemize}
\item Point $E_1 = (0, 0)$. Valeurs propres : $\lambda_1 = 4 > 0$ et $\lambda_2 = -1 < 0$

Il y a une valeur propre positive et une négative. $E_1$ est un point selle (instable).
\item Point $E_2 = (4, 0)$. Valeurs propres : $\lambda_1 = -4 < 0$ et $\lambda_2 = 3 > 0$

Il y a une valeur propre positive et une négative. $E_2$ est un point selle (instable).
\item Point $E_3 = (2, 1)$. Valeurs propres : $\lambda = \frac{-3 \pm i\sqrt{15}}{2}$

Les deux valeurs propres ont une partie réelle $\text{Re}(\lambda) = -\frac{3}{2} < 0$. $E_3$ est asymptotiquement stable.
\end{itemize}
\end{correction} 

\exo

\begin{enonce}
Soient $D = \{ (u,v) \in \mathbb{R}^2 , \quad u \geq 0, v \geq 0 \} $, le premier quadrant de $\mathbb{R}^2$ et le système différentiel suivant :

\[ 
\begin{cases} 
x' = -axy-bx+c\\
y' = axy-dy \\
a >0 \quad b>0 \quad d>0 \\
c \geq 0 \\
( x(0) , y(0) ) \in D
\end{cases} 
\]

\ques Déterminer les points d'équilibre de ce système.
\ques Ces points d'équilibre existe-t-ils toujours dans $D$? Sinon donner une condition nécessaire d'existence dans $D$. Lorsque l'existence dans $D$ de l'un d'entre eux n'est pas toujours assurée, exprimer cette condition (d'existence dans $D$) à l'aide d'une relation faisant intervenir une quantité que l'on notera $\mathcal{R}_0$.
\ques Pour chaque point d'équilibre, quand il existe dans $D$, dire s'il est hyperbolique ou non.
\ques Pour chaque point d'équilibre hyperbolique de $D$ étudier sa stabilité.
\ques Ces points d'équilibre lorsqu'ils existent dans $D$ peuvent-ils être conjointement stables?
\ques Etudier le cas $c=0$.

\end{enonce}
\begin{correction}

\ques Les points d'équilibre sont les solutions du système :

\[
\begin{cases}
-axy - bx + c = 0 \\
axy - dy = 0
\end{cases}
\]

En ajoutant les deux équations :

\[ -bx - dy + c = 0 \quad \Rightarrow \quad bx + dy = c \quad \quad (*) \]

De la deuxième équation : $axy = dy$

\begin{itemize}
\item Cas 1 : $y = 0$. De l'équation $(*)$ : $bx = c \Rightarrow x = \frac{c}{b}$.

Point d'équilibre : $\boxed{E_1 = \left(\frac{c}{b}, 0\right)}$
\item Cas 2 : $y \neq 0$. De la deuxième équation : $ax = d \Rightarrow x = \frac{d}{a}$

De l'équation $(*)$ : $b \cdot \frac{d}{a} + dy = c \Rightarrow dy = c - \frac{bd}{a} \Rightarrow y = \frac{c}{d} - \frac{b}{a}$

Point d'équilibre : $\boxed{E_2 = \left(\frac{d}{a}, \frac{c}{d} - \frac{b}{a}\right)}$
\end{itemize}

\ques Le domaine $D = \{(x,y) \in \mathbb{R}^2 : x \geq 0, y \geq 0\}$ est le premier quadrant.

\begin{itemize}
\item Point $E_1 = \left(\frac{c}{b}, 0\right)$

Comme $b > 0$ et $c \geq 0$, on a $\frac{c}{b} \geq 0$.

$E_1$ existe toujours dans $D$.
\item Point $E_2 = \left(\frac{d}{a}, \frac{ac - bd}{ad}\right)$

La première coordonnée : $\frac{d}{a} > 0$ (toujours positive) eta deuxième coordonnée : $y = \frac{ac - bd}{ad}$ doit satisfaire $y \geq 0$

$$\frac{ac - bd}{ad} \geq 0 \quad \Rightarrow \quad ac - bd \geq 0 \quad \Rightarrow \quad ac \geq bd$$

Condition d'existence de $E_2$ dans $D$ : $\boxed{ac \geq bd}$
\end{itemize}

On définit :

$$\boxed{\mathcal{R}_0 = \frac{ac}{bd}}$$

La condition d'existence de $E_2$ dans $D$ s'écrit :

\[ \boxed{\mathcal{R}_0 \geq 1} \]

Interpretation :

\begin{itemize}
\item Si $\mathcal{R}_0 < 1$ : seul $E_1$ existe dans $D$
\item Si $\mathcal{R}_0 \geq 1$ : les deux points $E_1$ et $E_2$ coexistent dans $D$
\item Si $\mathcal{R}_0 = 1$ : le point $E_2$ se réduit à $E_1$ (cas limite)
\end{itemize}

\ques Calcul de la matrice jacobienne

Le système s'écrit :

\[
\begin{cases}
f(x,y) = -axy - bx + c \\
g(x,y) = axy - dy
\end{cases}
\]

Les dérivées partielles sont :

\begin{itemize}
\item $\frac{\partial f}{\partial x} = -ay - b, \quad \frac{\partial f}{\partial y} = -ax$
\item $\frac{\partial g}{\partial x} = ay, \quad \frac{\partial g}{\partial y} = ax - d$
\end{itemize}

La matrice jacobienne est :

\[ 
J(x,y) = \begin{pmatrix} -ay - b & -ax \\ ay & ax - d \end{pmatrix}
\]

\bigskip

Point $E_1 = \left(\frac{c}{b}, 0\right)$

\[ 
J(E_1) = \begin{pmatrix} -b & -\frac{ac}{b} \\ 0 & \frac{ac}{b} - d \end{pmatrix}
\]

Les valeurs propres sont les éléments diagonaux (matrice triangulaire) :

\[ \lambda_1 = -b, \quad \lambda_2 = \frac{ac}{b} - d = \frac{ac - bd}{b} \]

\begin{itemize}
\item $\lambda_1 = -b < 0$ toujours
\item $\lambda_2 = \frac{ac - bd}{b}$
\end{itemize}

Pour que $E_1$ soit hyperbolique, il faut $\lambda_2 \neq 0$, c'est-à-dire $ac \neq bd$, ou $\mathcal{R}_0 \neq 1$.

\begin{itemize}
\item Si $\mathcal{R}_0 \neq 1$ : $E_1$ est hyperbolique
\item Si $\mathcal{R}_0 = 1$ : $E_1$ est non-hyperbolique (bifurcation)
\end{itemize}

\bigskip

Point $E_2 = \left(\frac{d}{a}, \frac{ac - bd}{ad}\right)$ (existe quand $\mathcal{R}_0 \geq 1$)

\begin{align*}
J(E_2) &= \begin{pmatrix} -a \cdot \frac{ac-bd}{ad} - b & -a \cdot \frac{d}{a} \\ a \cdot \frac{ac-bd}{ad} & a \cdot \frac{d}{a} - d \end{pmatrix} \\
&= \begin{pmatrix} -\frac{ac-bd}{d} - b & -d \\ \frac{ac-bd}{d} & 0 \end{pmatrix} \\
&= \begin{pmatrix} -\frac{ac-bd+bd}{d} & -d \\ \frac{ac-bd}{d} & 0 \end{pmatrix} \\
&= \begin{pmatrix} -\frac{ac}{d} & -d \\ \frac{ac-bd}{d} & 0 \end{pmatrix} 
\end{align*}

Le polynôme caractéristique est :

\begin{align*}
\det(J - \lambda I) &= \det \begin{pmatrix} -\frac{ac}{d} - \lambda & -d \\ \frac{ac-bd}{d} & -\lambda \end{pmatrix} \\
&= \left(-\frac{ac}{d} - \lambda\right)(-\lambda) + d \cdot \frac{ac-bd}{d} \\
&= \frac{ac}{d}\lambda + \lambda^2 + ac - bd \\
&= \lambda^2 + \frac{ac}{d}\lambda + (ac - bd)
\end{align*}

Pour que $E_2$ soit hyperbolique, le discriminant doit être non-nul :

\[ \Delta = \left(\frac{ac}{d}\right)^2 - 4(ac - bd) \neq 0 = \frac{a^2c^2}{d^2} - 4ac + 4bd \]

En général, pour $\mathcal{R}_0 > 1$, on a $\Delta > 0$ et les valeurs propres sont réelles.

\bigskip 

\textbf{Conclusion :} $E_2$ est hyperbolique quand il existe dans $D$ (c'est-à-dire quand $\mathcal{R}_0 > 1$).

Pour le cas limite $\mathcal{R}_0 = 1$, les deux points coïncident en un seul point non-hyperbolique.

\ques Stabilité des points d'équilibre hyperboliques

Stabilité de $E_1$ (quand $\mathcal{R}_0 \neq 1$)

Valeurs propres : $\lambda_1 = -b < 0$ et $\lambda_2 = \frac{ac - bd}{b}$

\begin{itemize}
\item Cas 1 : Si $\mathcal{R}_0 < 1$ (i.e., $ac < bd$), alors $\lambda_2 = \frac{ac - bd}{b} < 0$

Les deux valeurs propres sont négatives, donc $E_1$ est asymptotiquement stable (puits)
\item Cas 2 : Si $\mathcal{R}_0 > 1$ (i.e., $ac > bd$), alors $\lambda_2 = \frac{ac - bd}{b} > 0$.

Il y a une valeur propre positive, donc $E_1$ est instable (point selle)
\end{itemize}

\bigskip

Stabilité de $E_2$ (quand $\mathcal{R}_0 > 1$)

Valeurs propres du polynôme $\lambda^2 + \frac{ac}{d}\lambda + (ac - bd) = 0$ :

Les valeurs propres sont : $\lambda_1, \lambda_2 = \frac{-ac \pm \sqrt{a^2c^2 - 4acd^2 + 4bd^3}}{2d}$

Comme $\frac{ac}{d} > 0$ et le produit des racines est $ac - bd > 0$, on a :

\begin{itemize}
\item La somme des racines : $-\frac{ac}{d} < 0$
\item Le produit des racines : $ac - bd > 0$
\end{itemize}

\textbf{Conclusion :} Les deux valeurs propres ont une partie réelle négative, donc $E_2$ est asymptotiquement stable (puits) quand $\mathcal{R}_0 > 1$.

\ques 

Stabilité conjointe des deux points

\textbf{Théorème :} Deux points d'équilibre hyperboliques d'un système autonome en dimension 2 ne peuvent pas être conjointement asymptotiquement stables.

\textbf{Preuve par argument topologique :} Chaque puits attire un bassin de convergence non-vide. Si deux puits coexistaient, leurs bassins seraient disjoints et recouvreraient une partie ouverte du domaine $D$. Cependant, la dynamique du système ne permet pas cette configuration en dimension 2.

Analyse du système :

\begin{itemize}
\item Quand $\mathcal{R}_0 < 1$ : $E_1$ est stable, $E_2$ n'existe pas
\item Quand $\mathcal{R}_0 = 1$ : bifurcation transcritique, les deux points fusionnent
\item Quand $\mathcal{R}_0 > 1$ : $E_1$ devient instable (selle), $E_2$ devient stable
\end{itemize}

Conclusion: les deux points d'équilibre ne peuvent pas être conjointement stables.

Il existe une bifurcation transcritique au point $\mathcal{R}_0 = 1$ où la stabilité est transférée de $E_1$ à $E_2$.

\ques Quand $c = 0$, le système devient :

\[
\begin{cases}
x' = -axy - bx = -x(ay + b) \\
y' = axy - dy = y(ax - d)
\end{cases}
\]

Points d'équilibre:

\begin{itemize}
\item On a $E_1 = \left(\frac{0}{b} , 0 \right) = (0, 0)$
\item On a $E_2 = \left(\frac{d}{a}, \frac{0 - bd}{ad}\right) = \left(\frac{d}{a}, -\frac{b}{a}\right)$
\end{itemize}

Le point $E_2$ n'existe pas dans $D$ car sa deuxième coordonnée est négative. Seul $E_1 = (0, 0)$ existe dans $D$.

\medskip

Matrice jacobienne en $E_1 = (0, 0)$

\[
J(0, 0) = \begin{pmatrix} -b & 0 \\ 0 & -d \end{pmatrix}
\]

Valeurs propres : $\lambda_1 = -b < 0$, $\lambda_2 = -d < 0$

Donc $E_1$ est hyperbolique et asymptotiquement stable (puits).

\medskip

Pour $c = 0$, on note $\mathcal{R}_0 = \frac{0}{bd} = 0 < 1$. Le système ne produit pas de comportement endogène (aucune source interne).

Propriétés :

\begin{enumerate}
\item Domaine invariant : $D$ est invariant. Si $(x(0), y(0)) \in D$, alors $(x(t), y(t)) \in D$ pour tout $t \geq 0$.
\begin{itemize}
\item De $x' = -x(ay + b) \leq 0$, $x$ est décroissante
\item De $y' = y(ax - d)$, si $x < \frac{d}{a}$, alors $y' < 0$, donc $y$ est décroissante
\end{itemize}  
\item Convergence globale :*Toute trajectoire converge vers $(0, 0)$.
\item Interprétation biologique :
\begin{itemize}
\item Sans terme source ($c = 0$), le système ne peut maintenir de population à l'équilibre
\item Les deux espèces s'éteignent exponentiellement
\item C'est le régime d'extinction
\end{itemize}   
\end{enumerate}


Conclusion :

\[ \boxed{\text{Pour } c = 0 \text{ : seul } E_1 = (0, 0) \text{ existe et est globalement asymptotiquement stable.}} \]

\end{correction}

\exo

\begin{enonce}
Soit le systéme différentiel suivant:

\[ \begin{cases}
x'(t) = - y - x \sqrt{x^2 + y^2} \\
y'(t) = x - y \sqrt{x^2 + y^2} \\
x(0) \quad. y(0) \quad \text{ donnés}
 \end{cases} \]

\ques Montrer que ce systéme admet un unique point d'équilibre $P^* = (x^*,y^*)$ de $\mathbb{R}^2$.
\ques Etudier la stabilité du point $0$, point d'équilibre du système linéarisé autour de $P^*$. Peut-on conclure quant à la stabilité de $P^*$?
\ques Soit $d(P(t),P^*)$, la distance euclidienne entre le point $P(t) = (x(t),y(t))$ solution du système donné et $P^*$. C'est une fonction numérique de la variable $t$ d'expression

\[ d(P(t),P^*) = \sqrt{ ( x(t) - x^*)^2 + ( y(t) - y^*)^2 } \]

Etudier les variations de la fonction $d$ puis en déduire la stabilité du point $P$.
\end{enonce}

\begin{correction}
Un point d'équilibre satisfait $x'(t) = 0$ et $y'(t) = 0$, soit le système :

\[
\begin{cases}
-y - x\sqrt{x^2 + y^2} = 0 \\
x - y\sqrt{x^2 + y^2} = 0
\end{cases}
\]

On obtient (et on note):

\begin{equation}\label{equa:1}
x = y\sqrt{x^2 + y^2}
\end{equation}

\begin{equation}\label{equa:2}
y = -x\sqrt{x^2 + y^2}
\end{equation}

On multiplie les deux équations

\[
xy = -xy(x^2 + y^2)
\]

Cela donne : $xy[1 + (x^2 + y^2)] = 0$. Puisque $1 + (x^2 + y^2) > 0$, on a nécessairement $xy = 0$.

\begin{itemize}
\item Si $x = 0$, alors de l'équation \ref{equa:1} : $0 = 0$ et de l'équation \ref{equa:2} : $y = 0$.
\item Si $y = 0$, alors de l'équation \ref{equa:2} : $x = 0$, donc $y = 0$.
\end{itemize}

Le seul point d'équilibre est $P^* = (0, 0)$.

\ques Calcul de la matrice jacobienne en un point $(x, y)$ :

\[
F(x,y) = \begin{pmatrix} 
-y - x\sqrt{x^2 + y^2} \\ 
x - y\sqrt{x^2 + y^2} 
\end{pmatrix}
\]

Posons $r = \sqrt{x^2 + y^2}$.

\[
\frac{\partial F_1}{\partial x} = -\frac{x^2}{r} - r \quad ; \quad \frac{\partial F_1}{\partial y} = -1 - \frac{xy}{r}
\]

\[
\frac{\partial F_2}{\partial x} = 1 - \frac{xy}{r} \quad ; \quad \frac{\partial F_2}{\partial y} = -\frac{y^2}{r} - r
\]

À l'origine $(0,0)$, nous rencontrons une forme indéterminée. En utilisant la continuité et en passant à la limite, ou directement par développement limité au voisinage de l'origine, on établit que la jacobienne s'écrit formellement comme :

\[
J_F(0,0) = \begin{pmatrix} 0 & -1 \\ 1 & 0 \end{pmatrix}
\]

Valeurs propres : 

\[ \det(J_F - \lambda I) = \lambda^2 + 1 = 0 \Rightarrow \lambda = \pm i \]

Les valeurs propres sont purement imaginaires. Le système linéarisé a un centre à l'origine. On ne peut pas conclure directement sur la stabilité de $P^*$ par la théorème de linéarisation (cas critique).

\ques On a avec le point $P^*=(0,0)$:

\[ d(P(t), P^*) = \sqrt{x(t)^2 + y(t)^2} \]

On a:

\[ \frac{d}{dt}d = \frac{x \cdot x' + y \cdot y'}{d} \]

Avec $d \neq 0$, on a :

\begin{align*}
\dot{d} &= \frac{1}{d}\left[x(-y - x\sqrt{x^2+y^2}) + y(x - y\sqrt{x^2+y^2})\right] \\
&= \frac{1}{d}\left[-xy - x^2\sqrt{x^2+y^2} + xy - y^2\sqrt{x^2+y^2}\right] \\
&= \frac{1}{d}\left[-(x^2 + y^2)\sqrt{x^2+y^2}\right] \\
&= \frac{1}{d} \cdot (-d^2 \cdot d) = -d^3
\end{align*}

Par conséquent :

\[ \boxed{\dot{d} = -d^3} \]

\textbf{Étude des variations} :

\begin{itemize}
\item $\dot{d} < 0$ pour tout $d > 0$
\item La fonction $d(t)$ est strictement décroissante pour toute solution avec $d(0) > 0$
\item $d(t)$ tend vers $0$ lorsque $t \to +\infty$
\end{itemize}

\textbf{Résolution explicite} (pour information) :

\[ \frac{d(d)}{dt} = -d^3 \Rightarrow \frac{d(d)}{d^3} = -dt \]

Intégrant : $-\frac{1}{2d^2} = -t + C$

Avec la condition initiale $d(0) = 0$ : $C = \frac{1}{2d_0^2}$

\[ d(t) = \frac{d_0}{\sqrt{1 + 2d_0^2 t}} \]

Conclusion sur la stabilité de $P^* = (0,0)$ :

\begin{itemize}
\item Pour toute condition initiale proche de l'origine, la distance $d(t)$ est décroissante
\item Pour tout $t \geq 0$ : $d(t) \to 0$ quand $t \to \infty$
\item Le point $P^* = (0,0)$ est asymptotiquement stable (au sens du voisinage)
\end{itemize}

\textbf{Remarque :} La décroissance en $-d^3$ (plutôt que $-\lambda d$ linéaire) signifie que nous avons une convergence plus rapide que la stabilité linéaire.

\end{correction}


\exo

\begin{enonce}
Soit $A = \begin{pmatrix} 0 & -3 \\ 2 & 0 \end{pmatrix}$
\ques Montrer que la solution nulle de l'équation différentielle $\dot{x} = Ax$ est stable mais pas asymptotiquement stable.
\ques Trouver l'espace vectoriel des solutions.
\end{enonce}

\begin{correction}
\ques On a le polynôme caractéristique:

\[ \chi_A(\lambda) = \lambda^2 + 6 \Leftrightarrow Sp(A) = \{ -i\sqrt{6} , i \sqrt{6} \} \]

Les valeurs propres sont purement imaginaires.

\begin{commentaire}
On peut remarque que $A^2=-6 I_2$
\end{commentaire}

La solution générale s’écrit alors :
\[
e^{At} = \cos(\omega t)\,I + \frac{\sin(\omega t)}{\omega}\,A.
\]

\smallskip
Soit $X(0)=X_0=(x_0,y_0)^T$. On a donc :
\[
X(t) = e^{At} X_0
= \Big( \cos(\omega t)\,I + \frac{\sin(\omega t)}{\omega}A \Big) X_0.
\]

En composantes :
\[
\begin{cases}
x(t) = \cos(\omega t)\,x_0 - \dfrac{3}{\omega}\sin(\omega t)\,y_0, \\[6pt]
y(t) = \dfrac{2}{\omega}\sin(\omega t)\,x_0 + \cos(\omega t)\,y_0,
\end{cases}
\qquad \text{avec } \omega = \sqrt{6}.
\]

\smallskip
\textbf{Stabilité.}  
On remarque que la quantité
\[
Q(X) = 2x^2 + 3y^2
\]
est conservée le long des trajectoires, car
\[
\frac{d}{dt} Q(X(t)) = X(t)^T (A^T M + M A) X(t) = 0,
\]
avec $M = \begin{pmatrix} 2 & 0 \\ 0 & 3 \end{pmatrix}$.

Ainsi, $Q(X(t)) = Q(X(0))$ pour tout $t$.  
Cela montre que si $X(0)$ est proche de $0$, alors $X(t)$ reste proche de $0$ pour tout $t$ :  
l’origine est donc \textbf{stable au sens de Lyapunov}.

\smallskip
\textbf{Non-asymptotique.}  
Cependant, si $X(0) \neq 0$, alors $Q(X(t)) = Q(X(0)) > 0$ pour tout $t$ :  
les trajectoires ne s’approchent jamais de l’origine.  
Les solutions sont périodiques, donc la solution nulle est \textbf{stable mais non asymptotiquement stable}.

\bigskip
\textbf{2) Espace vectoriel des solutions.}

L’espace des solutions est :
\[
\mathcal{S} = \{\, X : \mathbb{R} \to \mathbb{R}^2 \mid X(t) = e^{At}v, \; v \in \mathbb{R}^2 \,\}.
\]
C’est un espace vectoriel de dimension $2$.

En prenant $v = e_1 = (1,0)^T$ et $v = e_2 = (0,1)^T$, on obtient une base de $\mathcal{S}$ :
\[
\begin{aligned}
u_1(t) &= e^{At} e_1 =
\begin{pmatrix}
\cos(\omega t) \\[4pt]
\dfrac{2}{\omega}\sin(\omega t)
\end{pmatrix}, \\[10pt]
u_2(t) &= e^{At} e_2 =
\begin{pmatrix}
-\dfrac{3}{\omega}\sin(\omega t) \\[6pt]
\cos(\omega t)
\end{pmatrix}.
\end{aligned}
\]

Toute solution réelle s’écrit donc :
\[
X(t) = c_1\,u_1(t) + c_2\,u_2(t),
\qquad (c_1,c_2)\in\mathbb{R}^2.
\]

Les solutions sont périodiques de période :
\[
T = \frac{2\pi}{\omega} = \frac{2\pi}{\sqrt{6}}.
\]

\textbf{Conclusion :} l’origine est un \emph{centre} — stable mais non asymptotiquement stable —,  
et les solutions décrivent des ellipses centrées en l’origine.
\end{correction}

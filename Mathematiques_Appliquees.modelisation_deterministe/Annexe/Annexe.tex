\chapter{Annexe: Théorèmes utilisé}

\section{Théorèmes fondamentaux : des gendarmes et des accroissements finis}


\begin{theo}[des gendarmes]
Soient $f, g, h$ trois fonctions définies sur un intervalle $I$ contenant un point $a$, telles que :
\[
\forall x \in I, \quad f(x) \leq g(x) \leq h(x)
\]
et supposons que :
\[
\lim_{x \to a} f(x) = \lim_{x \to a} h(x) = \ell
\]
Alors :
\[
\lim_{x \to a} g(x) = \ell
\]
\end{theo}


\begin{proof}
L'idée est que $g(x)$ est \textit{encadrée} entre deux fonctions $f(x)$ et $h(x)$ qui ont la même limite $\ell$.  
Ainsi, pour tout $\varepsilon > 0$, il existe $\eta > 0$ tel que pour tout $x$ vérifiant $0 < |x - a| < \eta$, on a :
\[
|\ f(x) - \ell\ | < \varepsilon \quad \text{et} \quad |\ h(x) - \ell\ | < \varepsilon
\]
ce qui entraîne :
\[
\ell - \varepsilon < f(x) \leq g(x) \leq h(x) < \ell + \varepsilon
\]
d'où :
\[
|\ g(x) - \ell\ | < \varepsilon
\]
ce qui prouve que $\displaystyle \lim_{x \to a} g(x) = \ell$.
\end{proof}

\bigskip


\begin{theo}[des accroissements finis — TAF]
Soit $f : [a,b] \to \mathbb{R}$ une fonction continue sur $[a,b]$ et dérivable sur $]a,b[$.  
Alors il existe au moins un point $c \in ]a,b[$ tel que :
\[
f'(c) = \frac{f(b) - f(a)}{b - a}
\]
\end{theo}

\begin{proof}
On applique le théorème de Rolle à la fonction :
\[
\varphi(x) = f(x) - \left( \frac{f(b) - f(a)}{b - a} \right)(x - a)
\]
La fonction $\varphi$ est continue sur $[a,b]$ et dérivable sur $]a,b[$, avec :
\[
\varphi(a) = f(a) \quad \text{et} \quad \varphi(b) = f(b) - \frac{f(b) - f(a)}{b - a}(b - a) = f(a)
\]
Ainsi, $\varphi(a) = \varphi(b)$ et, par le théorème de Rolle, il existe $c \in ]a,b[$ tel que $\varphi'(c) = 0$, soit :
\[
f'(c) - \frac{f(b) - f(a)}{b - a} = 0
\]
d'où :
\[
f'(c) = \frac{f(b) - f(a)}{b - a}
\]
\end{proof}

\commentaire{
Le théorème des accroissements finis formalise l'idée intuitive qu’une fonction dérivable atteint, en au moins un point, une pente égale à la pente de la corde reliant ses extrémités.  
Il permet notamment de démontrer la croissance ou la décroissance d’une fonction à partir du signe de sa dérivée.
}



\section{Théorèmes Equation Differentielle}


\begin{theo}[Cauchy--Lipschitz]
Soient $t_0 \in \mathbb{R}$, $x_0 \in \mathbb{R}^n$, et $U \subset \mathbb{R} \times \mathbb{R}^n$ un ouvert contenant $(t_0,x_0)$. 
Soit $f : U \to \mathbb{R}^n$ une application continue, localement lipschitzienne par rapport à la variable $x$. 

Alors, le problème de Cauchy
\[
\begin{cases}
x'(t) = f(t,x(t)), \\
x(t_0) = x_0,
\end{cases}
\]
admet une solution $x : I \to \mathbb{R}^n$, définie sur un intervalle $I$ contenant $t_0$. 

De plus, cette solution est unique : si $y$ est une autre solution définie sur un intervalle $J$ contenant $t_0$, alors $x(t)=y(t)$ pour tout $t \in I \cap J$.
\end{theo}

\commentaire{
Le théorème dit que si une fonction suit des hypothèses très usuelles dans le monde réelle, et si on prend un intervalle spécifique (comme $\mathbb{R}^+$), et une condition initial, le système différentielle admet une unique solution
}




















































\newpage

\section{Fonctions hyperboliques}

Dans cette annexe nous rassemblons les définitions, propriétés, identités et résultats usuels sur les fonctions hyperboliques. L'approche reste élémentaire : définitions par exponentielles, dérivées, formules d'addition, identités fondamentales, fonctions réciproques, séries de Taylor, intégrales usuelles et quelques applications.

\begin{defi}
Les \emph{fonctions hyperboliques} $\sinh,\cosh,\tanh,\operatorname{sech},\operatorname{csch},\coth$ sont définies pour tout réel $x$ par :
\[
\begin{aligned}
\sinh x &= \frac{e^x-e^{-x}}{2},\\[4pt]
\cosh x &= \frac{e^x+e^{-x}}{2},\\[4pt]
\tanh x &= \frac{\sinh x}{\cosh x}=\frac{e^x-e^{-x}}{e^x+e^{-x}}\quad(\cosh x\neq0),\\[4pt]
\operatorname{sech} x &= \frac{1}{\cosh x},\qquad
\operatorname{csch} x = \frac{1}{\sinh x}\quad(x\neq0),\\[4pt]
\coth x &= \frac{\cosh x}{\sinh x}\quad(x\neq0).
\end{aligned}
\]
\end{defi}

\begin{prop}
\begin{enumerate}
\item Domaines :
\[
\sinh,\cosh,\tanh,\operatorname{sech}\ \text{sont définies sur }\mathbb R;
\qquad \operatorname{csch},\coth\ \text{sont définies sur }\mathbb R\setminus\{0\}.
\]
\item Parité :
\[
\sinh(-x)=-\sinh x,\quad \cosh(-x)=\cosh x.
\]
\item Croissance et signes :
\[
\cosh x\ge 1\ \forall x,\qquad \sinh x \text{ est strictement croissante,}
\]
avec $\sinh 0=0$, $\cosh 0=1$. De plus $\tanh x$ est strictement croissante et $-1<\tanh x<1$ pour tout $x\in\mathbb R$.
\end{enumerate}
\end{prop}

\begin{prop}[Identité fondamentale]
Pour tout réel $x$,
\[
\boxed{\ \cosh^2 x-\sinh^2 x=1\ }\!.
\]
\end{prop}

\begin{proof}
\[
\cosh^2x-\sinh^2x
=\Big(\frac{e^x+e^{-x}}2\Big)^2-\Big(\frac{e^x-e^{-x}}2\Big)^2
=\frac{(e^{2x}+2+e^{-2x})-(e^{2x}-2+e^{-2x})}{4}=1.
\]
\end{proof}

\begin{prop}[Formules d'addition]
Pour tous réels $x,y$,
\[
\begin{aligned}
\sinh(x+y)&=\sinh x\cosh y+\cosh x\sinh y,\\
\cosh(x+y)&=\cosh x\cosh y+\sinh x\sinh y.
\end{aligned}
\]
\end{prop}

\begin{proof}
\[
\sinh(x+y)=\tfrac{e^{x+y}-e^{-x-y}}2
=\sinh x\cosh y+\cosh x\sinh y.
\]
La formule pour $\cosh(x+y)$ se démontre de façon analogue.
\end{proof}


\begin{prop}[Dérivées usuelles]
\[
\begin{aligned}
(\sinh x)'&=\cosh x,\\
(\cosh x)'&=\sinh x,\\
(\tanh x)'&=\operatorname{sech}^2 x=1-\tanh^2 x,\\
(\operatorname{sech} x)'&=-\operatorname{sech} x\,\tanh x,\\
(\operatorname{csch} x)'&=-\operatorname{csch} x\,\coth x,\\
(\coth x)'&=-\operatorname{csch}^2 x.
\end{aligned}
\]
\end{prop}

\begin{proof}
Direct par dérivation des définitions à l’aide de $e^x$.
\end{proof}

\begin{prop}[Séries de Taylor]
Pour tout $x\in\mathbb R$ :
\[
\begin{aligned}
\sinh x &= \sum_{n=0}^{\infty}\frac{x^{2n+1}}{(2n+1)!}=x+\frac{x^3}{3!}+\frac{x^5}{5!}+\cdots,\\[4pt]
\cosh x &= \sum_{n=0}^{\infty}\frac{x^{2n}}{(2n)!}=1+\frac{x^2}{2!}+\frac{x^4}{4!}+\cdots.
\end{aligned}
\]
\end{prop}

\begin{prop}[Limites et asymptotiques]
\begin{enumerate}
\item $\displaystyle \lim_{x\to+\infty}\sinh x=\lim_{x\to+\infty}\cosh x=+\infty$, et $\cosh x\sim\frac{e^{x}}{2}$, $\sinh x\sim\frac{e^{x}}{2}$ quand $x\to+\infty$.
\item $\displaystyle \lim_{x\to-\infty}\sinh x=-\infty$, $\cosh x\to+\infty$, et $\tanh x\to\pm1$ quand $x\to\pm\infty$.
\end{enumerate}
\end{prop}

\begin{prop}[Fonctions hyperboliques inverses]
\[
\begin{aligned}
\operatorname{arsinh} x&=\ln\big(x+\sqrt{x^2+1}\big),\qquad (\operatorname{arsinh} x)'=\frac{1}{\sqrt{1+x^2}},\\[4pt]
\operatorname{arcosh} x&=\ln\big(x+\sqrt{x^2-1}\big)\quad (x\ge1),\qquad (\operatorname{arcosh} x)'=\frac{1}{\sqrt{x^2-1}},\\[4pt]
\operatorname{artanh} x&=\tfrac{1}{2}\ln\!\big(\tfrac{1+x}{1-x}\big)\quad(|x|<1),\qquad (\operatorname{artanh} x)'=\frac{1}{1-x^2}.
\end{aligned}
\]
\end{prop}

\begin{prop}[Intégrales usuelles]
\[
\begin{aligned}
\int \sinh x\,dx &=\cosh x + C,\\
\int \cosh x\,dx &=\sinh x + C,\\
\int \tanh x\,dx &=\ln(\cosh x)+C,\\
\int \operatorname{sech}^2 x\,dx &=\tanh x + C,\\
\int \operatorname{sech}\,x\,dx &= 2\arctan\big(\tanh(\tfrac x2)\big)+C.
\end{aligned}
\]
\end{prop}

\begin{prop}[Identités supplémentaires]
\[
\begin{aligned}
\cosh(2x)&=\cosh^2x+\sinh^2x=2\cosh^2x-1=1+2\sinh^2x,\\
\sinh(2x)&=2\sinh x\cosh x,\\
\tanh(2x)&=\frac{2\tanh x}{1+\tanh^2 x}.
\end{aligned}
\]
\end{prop}

\begin{prop}[Solutions d'équation différentielle simple]
L'équation $y''=y$ admet la base de solutions $\{\cosh x,\sinh x\}$.  
Ainsi, la solution générale est
\[
y(x)=A\cosh x+B\sinh x,\qquad A,B\in\mathbb R.
\]
\end{prop}

\begin{prop}[Lien avec les fonctions trigonométriques]
La substitution $x\mapsto ix$ relie les fonctions trigonométriques et hyperboliques :
\[
\cos(ix)=\cosh x,\qquad \sin(ix)=i\sinh x.
\]
\end{prop}


\begin{prop}[Formules d'exponentielle utiles]
\[
e^x=\cosh x+\sinh x,\qquad e^{-x}=\cosh x-\sinh x.
\]
\end{prop}

\commentaire{
\textbf{Remarques pratiques :}
\begin{itemize}
\item La substitution $t=\tanh(x/2)$ simplifie souvent les intégrales rationnelles en $\sinh$ et $\cosh$.
\item Beaucoup d’identités s’obtiennent aisément à partir de $e^x$ et $e^{-x}$.
\item $\sinh$ est strictement croissante, $\cosh$ minimale en $0$, $\tanh$ bornée par $1$.
\end{itemize}

\vspace{6pt}
\noindent\textbf{Table récapitulative}
\[
\begin{array}{l|l}
\text{Fonction} & \text{Dérivée}\\\hline
\sinh x & \cosh x\\
\cosh x & \sinh x\\
\tanh x & 1-\tanh^2 x\\
\operatorname{sech} x & -\operatorname{sech} x\,\tanh x\\
\operatorname{csch} x & -\operatorname{csch} x\,\coth x\\
\coth x & -\operatorname{csch}^2 x
\end{array}
\]
}


\newpage

\section{Développements limités}

Les développements limités (DL) permettent d’approcher une fonction par un polynôme au voisinage d’un point, le plus souvent $0$.  
Ils sont fondamentaux pour l’analyse locale des fonctions, les approximations numériques et les calculs de limites.

\begin{defi}[Développement limité]
Soit $f$ une fonction définie dans un voisinage de $a\in\mathbb R$.  
On dit que $f$ admet un \emph{développement limité à l’ordre $n$ en $a$} si :
\[
f(x) = P_n(x-a) + o((x-a)^n),
\]
où $P_n$ est un polynôme de degré $\le n$, dit \emph{polynôme de Taylor d’ordre $n$ en $a$}.

Autrement dit :
\[
P_n(x-a) = \sum_{k=0}^n \frac{f^{(k)}(a)}{k!}(x-a)^k.
\]
\end{defi}


\begin{prop}
\begin{enumerate}
\item Si $f$ est de classe $\mathcal C^n$ au voisinage de $a$, alors elle admet un DL à l’ordre $n$ en $a$.
\item Si $f$ et $g$ admettent des DL au voisinage de $a$, alors :
\[
\begin{aligned}
(f+g)(x) &\text{ a pour DL la somme des deux DL,}\\
(fg)(x) &\text{ a pour DL le produit des deux DL, tronqué à l’ordre voulu,}\\
\frac{f}{g}(x) &\text{ a pour DL le quotient des deux DL, si $g(a)\neq 0$.}
\end{aligned}
\]
\item Le DL est \textbf{unique} : si $f(x)=P(x)+o((x-a)^n)=Q(x)+o((x-a)^n)$, alors $P=Q$.
\end{enumerate}
\end{prop}


\begin{prop}[Développements limités usuels en $0$]
On a, pour $x$ réel petit :
\[
\begin{aligned}
e^x &= 1 + x + \frac{x^2}{2!} + \frac{x^3}{3!} + \frac{x^4}{4!} + o(x^4),\\[6pt]
\ln(1+x) &= x - \frac{x^2}{2} + \frac{x^3}{3} - \frac{x^4}{4} + o(x^4)\quad(x>-1),\\[6pt]
(1+x)^\alpha &= 1 + \alpha x + \frac{\alpha(\alpha-1)}{2}x^2 + \frac{\alpha(\alpha-1)(\alpha-2)}{6}x^3 + o(x^3),\\[6pt]
\sin x &= x - \frac{x^3}{3!} + \frac{x^5}{5!} + o(x^5),\\[6pt]
\cos x &= 1 - \frac{x^2}{2!} + \frac{x^4}{4!} + o(x^4),\\[6pt]
\tan x &= x + \frac{x^3}{3} + \frac{2x^5}{15} + o(x^5),\\[6pt]
\sinh x &= x + \frac{x^3}{3!} + \frac{x^5}{5!} + o(x^5),\\[6pt]
\cosh x &= 1 + \frac{x^2}{2!} + \frac{x^4}{4!} + o(x^4),\\[6pt]
\tanh x &= x - \frac{x^3}{3} + \frac{2x^5}{15} + o(x^5),\\[6pt]
\frac{1}{1-x} &= 1 + x + x^2 + x^3 + o(x^3)\quad(|x|<1).
\end{aligned}
\]
\end{prop}

\begin{proof}
Ces développements se déduisent de la série de Taylor :
\[
f(x) = \sum_{k=0}^{n} \frac{f^{(k)}(0)}{k!}x^k + o(x^n),
\]
ou par manipulation algébrique (produit, quotient, inversion de série).
\end{proof}

\begin{prop}[Comportement du reste]
Si $f$ est de classe $\mathcal C^{n+1}$ au voisinage de $a$, on peut écrire :
\[
f(x) = P_n(x-a) + R_n(x),
\]
avec une \emph{forme de Lagrange du reste} :
\[
R_n(x) = \frac{f^{(n+1)}(\xi)}{(n+1)!}(x-a)^{n+1},\quad \xi\in(a,x).
\]
\end{prop}


\begin{prop}[DL de fonctions composées]
Si $f$ admet un DL en $a$ et $g$ admet un DL en $f(a)$, alors :
\[
g(f(x)) = g(f(a)) + g'(f(a))(f(x)-f(a)) + \cdots
\]
On procède en remplaçant le DL de $f$ dans celui de $g$ et en développant à l’ordre voulu.
\end{prop}


\commentaire{
\[
\begin{aligned}
\sin(x^2) &= x^2 - \frac{(x^2)^3}{3!} + o(x^6) = x^2 - \frac{x^6}{6} + o(x^6),\\[6pt]
e^{\sin x} &= 1 + \sin x + \frac{(\sin x)^2}{2} + o(x^2) = 1 + x + \frac{x^2}{2} - \frac{x^3}{6} + o(x^3).
\end{aligned}
\]
}

\begin{prop}[Règles pratiques de calcul de DL]
\begin{enumerate}
\item On ne garde que les termes jusqu’à l’ordre demandé.
\item On remplace successivement les fonctions par leur DL tronqué.
\item On simplifie à chaque étape (en négligeant les $o(x^n)$ inutiles).
\item On peut utiliser des logiciels de calcul formel pour vérifier (par ex. \texttt{sympy} ou \texttt{Maple}).
\end{enumerate}
\end{prop}

\begin{prop}[Applications classiques]
\begin{enumerate}
\item \textbf{Calcul de limites} : $\displaystyle \lim_{x\to0}\frac{\sin x}{x}=1$ découle du DL de $\sin x$.
\item \textbf{Études de fonctions} : signe, convexité, asymptotes locales.
\item \textbf{Approximation numérique} : $e^x\simeq 1+x+\frac{x^2}{2}$ pour $|x|\ll1$.
\end{enumerate}
\end{prop}

\vspace{0.4cm}
\noindent\textbf{Table récapitulative des développements limités usuels :}
\[
\begin{array}{l|l}
\text{Fonction} & \text{Développement limité en } 0\\\hline
e^x & 1 + x + \frac{x^2}{2} + \frac{x^3}{6} + \frac{x^4}{24} + o(x^4)\\[4pt]
\ln(1+x) & x - \frac{x^2}{2} + \frac{x^3}{3} - \frac{x^4}{4} + o(x^4)\\[4pt]
(1+x)^\alpha & 1 + \alpha x + \frac{\alpha(\alpha-1)}{2}x^2 + \frac{\alpha(\alpha-1)(\alpha-2)}{6}x^3 + o(x^3)\\[4pt]
\sin x & x - \frac{x^3}{6} + \frac{x^5}{120} + o(x^5)\\[4pt]
\cos x & 1 - \frac{x^2}{2} + \frac{x^4}{24} + o(x^4)\\[4pt]
\tan x & x + \frac{x^3}{3} + \frac{2x^5}{15} + o(x^5)\\[4pt]
\sinh x & x + \frac{x^3}{6} + \frac{x^5}{120} + o(x^5)\\[4pt]
\cosh x & 1 + \frac{x^2}{2} + \frac{x^4}{24} + o(x^4)\\[4pt]
\tanh x & x - \frac{x^3}{3} + \frac{2x^5}{15} + o(x^5)\\[4pt]
\frac{1}{1-x} & 1 + x + x^2 + x^3 + o(x^3)
\end{array}
\]


\chapter{Introduction}

\section{Exemples Historiques}

Les maladies infectieuses ont marqué l'histoire de l'humanité de manière dramatique, causant des millions de morts et transformant profondément les sociétés. De la peste noire médiévale aux pandémies contemporaines, ces fléaux continuent de représenter un défi majeur pour la santé publique mondiale. \textit{En France, les maladies infectieuses constituent aujourd'hui la troisième cause de mortalité}, illustrant l'importance persistante de cette problématique sanitaire.

\subsection{Les Grandes Épidémies Historiques}

\subsubsection{La Peste Noire (1346-1350)}

\textbf{Entre 1346 et 1350, la peste noire, probablement bubonique, a tué le quart de la population européenne.} Cette épidémie, causée par la bactérie \textit{Yersinia pestis}, a particulièrement ravagé la France avec des bilans dramatiques dans les principales villes.

\begin{figure}[h]
\centering
\includegraphics[width=0.6\textwidth]{Introduction/Image/yersinia_pestis.jpg}
\caption{Yersinia pestis : la bactérie responsable de la peste noire principalement bubonique}
\end{figure}

\textit{En France, on dénombra par exemple 120 000 morts en Provence, 45 000 à Lyon et 100 000 à Rouen.} Cette épidémie s'est répétée régulièrement pendant plus de 300 ans, notamment lors de la Grande peste de Londres en 1665-1666.

\textbf{Les inquisiteurs du Moyen Âge ont lourdement aggravé les ravages de la grande peste.} En 1233, le Pape Grégoire IX a déclaré dans une bulle pontificale que le chat, incarnant la luxure et la paresse, était un serviteur du diable. \textit{Les inquisiteurs les ont pourchassés et exterminés, ce qui a permis aux rats de pulluler dans les villes.} Lorsque les rats, vecteurs de la Peste Noire, ont débarqué en France, ils se sont multipliés faute de prédateurs et ont répandu le fléau.

\subsubsection{La Variole et les Populations Amérindiennes}

\textbf{La petite vérole ou variole, maladie très contagieuse et immunisante dont 15\% de cas sont mortels, a tué environ 2 millions d'Aztèques.}

\begin{figure}[h]
\centering
\includegraphics[width=0.6\textwidth]{Introduction/Image/variole_bangladesh.jpg}
\caption{Jeune fille du Bangladesh atteinte de variole en 1973}
\end{figure}

\subsubsection{La Grippe Espagnole (1918-1919)}

\textbf{L'épidémie de grippe de 1918-1919, dite grippe espagnole, a fait entre 50 à 100 millions de morts au total.} Cette pandémie reste l'une des plus meurtrières de l'histoire moderne.

\begin{figure}[h]
\centering
\includegraphics[width=0.6\textwidth]{Introduction/Image/virus_grippe_espagnole.jpg}
\caption{Virus reconstitué de la grippe espagnole, celui qui est le plus proche par ses effets sur l'organisme du virus H5N1 (grippe A)}
\end{figure}

\subsubsection{Autres Épidémies Historiques Marquantes}

\textbf{Le virus de la grippe de Hong Kong (pandémie de 1968-1969) a causé 1 million de morts.} Cette pandémie était causée par une souche réassortie H3N2 du virus H2N2 de la grippe A.

\begin{figure}[h]
\centering
\includegraphics[width=0.6\textwidth]{Introduction/Image/grippe_hong_kong.jpg}
\caption{Souche de l'agent responsable de la pandémie de 1968 : grippe de Hong Kong, causée par une souche réassortie H3N2 du virus H2N2 de la grippe A}
\end{figure}

\textbf{De 1918 à 1921, 25 millions de personnes en Russie furent atteintes de typhus.} Cette maladie est causée par les Rickettsies, famille de bactéries responsables du typhus. \textit{La Rickettsie sévit à l'état endémique chez les rongeurs qui lui servent d'hôte, y compris les souris et les rats, et est transmise aux humains par la morsure ou piqûre d'acariens, de puces et des poux de corps.}

\begin{figure}[h]
\centering
\includegraphics[width=0.6\textwidth]{Introduction/Image/rickettsies.jpg}
\caption{Les Rickettsies : famille de bactéries responsable du typhus}
\end{figure}

\subsection{Les Défis Sanitaires Contemporains}

\subsubsection{Bilan Actuel des Maladies Infectieuses}

Malgré les progrès de la médecine dans la lutte contre les maladies, on observe encore, par an :
\begin{itemize}
\item \textbf{Un million de morts dues à la malaria ou paludisme}
\item \textit{Plus de 100 millions de cas de malaria en Inde}
\item \textbf{Un million de morts dues à la rougeole}
\item \textbf{Deux millions de morts dues à la tuberculose}
\item \textbf{Trois millions de morts dues au V.I.H. dont 3 500 en France}, avec dix millions d'infectés par le V.I.H.
\item \textbf{Trois millions de morts dues aux maladies respiratoires}
\item \textit{Près de deux millions de morts dues à la tuberculose (liés dans un grand nombre de cas au Sida) dont 700 en France}
\end{itemize}

\subsubsection{Les Épidémies Émergentes Récentes}

Plus récemment, plusieurs virus ont suscité la préoccupation de la communauté internationale : \textit{le virus d'Ebola, l'épidémie de SRAS en 2002-2003, la dengue, le chikungunya, la fièvre Zika et le coronavirus Sars-Cov-2 (Covid-19).} L'apparition de plusieurs souches de la grippe aviaire a entraîné l'abattage forcé de millions de volailles.

\subsection{Focus sur les Épidémies Émergentes}

\subsubsection{L'Épidémie de Fièvre Hémorragique Ebola}

\begin{figure}[h]
\centering
\includegraphics[width=0.6\textwidth]{Introduction/Image/virus_ebola.jpg}
\caption{Image colorisée d'une particule virale Ebola obtenue par microscopie électronique en transmission en août 2014}
\end{figure}

\textbf{Ce virus, qui provoque des fièvres hémorragiques, tire son nom d'une rivière du Nord de l'actuelle République démocratique du Congo (ex-Zaïre), où il a été repéré pour la première fois en 1976.} \textit{Son taux de mortalité varie entre 25 et 90\% chez l'homme.}

\textbf{L'épidémie de fièvre hémorragique Ebola en Afrique de l'ouest a tué plus de 2 400 personnes sur 4 784 cas au 12 septembre 2014, selon l'OMS.} Dans les trois pays les plus touchés (Guinée, Libéria et Sierra Leone), le nombre de cas avait augmenté plus vite que la capacité à les gérer.

\textit{L'Organisation mondiale de la santé avait rappelé qu'il n'y avait plus un seul lit disponible pour traiter les patients d'Ebola au Libéria.} En résumé, l'épidémie qui a sévi en Afrique de l'Ouest en 2014 et 2015 affichait une létalité de 39,5\% au 27 mars 2016 (11 323 morts sur 28 646 cas recensés).

\subsubsection{La Maladie de Marburg}

Le lundi 9 août, l'OMS alerte sur la maladie de Marburg qui continue sa progression avec la détection du premier cas en Afrique de l'Ouest, en Guinée. \textbf{Hautement transmissible, la maladie à virus de Marburg est un cousin à peine moins meurtrier du virus Ebola, contre lequel il n'y a ni vaccin ni traitement.} Elle se manifeste par une fièvre aiguë accompagnée d'hémorragies internes et externes entraînant la mort dans 50\% des cas en moyenne.

\textit{La maladie, autrefois appelée fièvre hémorragique à virus Marburg, tire son nom de la ville allemande où elle a été identifiée pour la première fois en 1967.} Ce virus de la famille des filoviridae, comme Ebola, se transmet à l'homme par les chauves-souris frugivores (roussettes), considérées comme l'hôte naturel du virus.

\subsubsection{Le Virus Chapare}

\textbf{Ce virus est apparu en 2004 en Bolivie, dans la province de Chapare qui lui a donné son nom.} Dans un hôpital de La Paz, plusieurs personnes ont été prises en charge pour une fièvre hémorragique. \textit{Cinq personnes en contact avec ces patients ont été à leur tour infectées, dont un interne en médecine, un ambulancier et un gastro-entérologue.} Au total, trois personnes sont décédées.

\textbf{Ce qui inquiète les scientifiques, c'est qu'il s'agit de la première description de la transmission interhumaine du virus Chapare.} Les patients infectés ont souffert de fièvres, de douleurs abdominales, de vomissements et de saignements de gencives. \textit{Aucun traitement spécifique n'existe à ce jour.}

\subsection{Les Vecteurs Contemporains : Le Moustique Tigre}

\subsubsection{Présentation et Répartition}

\begin{figure}[h]
\centering
\includegraphics[width=0.6\textwidth]{Introduction/Image/moustique_tigre.jpg}
\caption{Aedes aegypti est une espèce de moustique qui est le vecteur principal de la dengue, du virus Zika, du chikungunya et de la fièvre jaune. Il en est de même d'Aedes albopictus}
\end{figure}

\textbf{D'après Santé Publique France, au 1er mai 2020, l'Aedes albopictus est présent dans 57 départements français.}

\subsubsection{Le Virus Zika}

\textbf{Proche du virus de la dengue et du chikungunya, le virus Zika a fait son apparition en novembre 2019 dans le Var (3 cas avérés) et 1 cas importé (Île-de-France) entre le 1er mai et le 13 septembre 2020.}

\textit{Le virus Zika est une maladie virale transmise à l'homme par l'intermédiaire d'une piqûre du moustique du genre Aedes.} Dans certains cas, le virus peut se transmettre par voie sexuelle. Les symptômes sont le plus souvent bénins mais \textbf{le virus peut provoquer des anomalies congénitales en cas d'infection pendant la grossesse.}

\textbf{Ce virus Zika, identifié pour la première fois en Ouganda dans la forêt du même nom en 1947, débarque dans nos contrées.} Une fois contaminé, l'hôte peut contracter de la fièvre, des douleurs musculaires, ou des éruptions cutanées. \textit{Dans les cas les plus graves, des complications neurologiques ont aussi été observées.}

\textbf{Durant l'épidémie de 2013-2014, la Polynésie française a fait état de 32 000 personnes touchées par le virus Zika.} Selon le HCSP, 72 d'entre elles avaient présenté des complications neurologiques graves, dont 42 syndromes de Guillain-Barré, qui peut déboucher sur une paralysie totale des membres et du visage.

\subsubsection{La Dengue à Paris}

\textbf{La dengue a été détectée dans le 13ème puis le 15ème arrondissement de Paris depuis fin août 2023.} Pour éviter toute propagation, une opération de démoustication a été diligentée par l'ARS Île-de-France dans la nuit du vendredi 15 au samedi 16 septembre 2023.

\textit{L'épandage de produits insecticides s'est déroulé dans le 15ème arrondissement et plus particulièrement dans le quartier Dombasle, dans un rayon de 150 mètres autour du domicile des personnes contaminées.} Les habitants ont été invités à rentrer leurs plantes d'extérieur, à bien fermer leurs fenêtres, à garder à l'intérieur leurs animaux domestiques.

\subsubsection{Nouveaux Virus Transmis par le Moustique Tigre}

\textbf{D'après un article de la revue médicale La Santé au quotidien du 26 juin 2023, le moustique tigre transmet deux nouveaux virus.} Une étude de surveillance menée dans le Grand Est a identifié deux nouveaux virus transmis par le moustique tigre : les virus West Nile (virus de Nil occidental) et Usutu.

\textbf{Le virus West Nile est connu depuis plusieurs années en France et est régulièrement mis en évidence sur le bassin méditerranéen.} \textit{Seules 20\% des personnes contaminées développent des symptômes.} Quand les symptômes sont présents, le virus se manifeste par une fièvre soudaine, des maux de tête, des courbatures voire des complications neurologiques : méningites et encéphalites.

\textit{Le virus d'Usutu se transmet par les moustiques, les moustiques tigre mais aussi par les oiseaux.} Les symptômes incluent une désorientation et une perte de coordination motrice.

\subsection{Les Coronavirus : De SARS-CoV-2 au Khosta-2}

\subsubsection{Le COVID-19}

\begin{figure}[h!]
\centering
\includegraphics[width=0.6\textwidth]{Introduction/Image/sars_cov2.jpg}
\caption{Image d'illustration de virus Sars-Cov-2 dans un organisme humain}
\end{figure}

\textbf{Le Covid-19 est la maladie infectieuse causée par le coronavirus le plus récemment découvert.} Les coronavirus sont une famille de virus qui provoquent des maladies allant d'un simple rhume à des pathologies plus sévères. \textit{Ce nouveau coronavirus et cette nouvelle maladie n'avaient encore jamais été décelés avant qu'une flambée épidémique ne soit signalée à Wuhan, en Chine, en décembre 2019.}

Le Covid-19 est un virus dangereux parce que :
\begin{itemize}
\item \textbf{Il est très contagieux : chaque personne infectée va contaminer au moins 3 personnes en l'absence de mesures de protection}
\item \textbf{Une personne contaminée mais qui ne ressent pas encore de symptômes peut contaminer d'autres personnes}
\end{itemize}

\begin{table}[h!]
\centering
\small % Réduire la taille de police
\begin{tabular}{|p{3cm}|p{3cm}|p{2cm}|p{3cm}|p{2cm}|}
\hline
Pays & Nombre de cas & + & Nombre de mort & + \\
\hline
Monde Entier & 219 M & & 4,55 M & \\
\hline
France & 6,92 M & 10 327 & 116k & 149 \\
\hline
Usa & 41,4 M & 160 k & 664 k & 2 652 \\
\hline
Indie & 32,6 M & & 437 k & \\
\hline
Brésil & 21 M & 13 406 & 588 k & 731 \\
\hline
Royaume Uni & 7,28 M & 26 251 & 134 k & 185 \\
\hline
Russie & 7,18 M & 17 837 & 194 k & 781 \\
\hline
Turquie & 6,68 M & & 60 117 & \\
\hline
Argentine & 5,23 M & 3 017 & 114 k & 176 \\
\hline
Iran & 5,08 M & & 110 k & \\
\hline
Colombie & 4,93 M & 1 435  & 126 k & 26 \\
\hline
Espagne & 4,92 M & 3 261 &  85 548 & 155\\
\hline
Italie & 4,61 M & 4 009 & 130 k & 72 \\
\hline
Indonésie & 4,17 M & 4 128 & 139 k & 250 \\
\hline
Allemagne & 4,12 M & 6 325 & 93 319 & 68 \\
\hline
Mexique & 3,53 M & & 269 k & \\
\hline
Pologne & 2,86 M & 536 & 75 433 & 8 \\
\hline
Afrique du Sud & 2,86 M & 3 699 & 85 302 & 300 \\
\hline
\end{tabular}
\caption{Sars-Cov-2 : nombre de cas et de morts (au 14/09/2021)}
\label{tab:Covid_2024}
\end{table}

\begin{table}[h!]
\centering
\small % Réduire la taille de police
\begin{tabular}{|p{5cm}|p{5cm}|p{5cm}|}
\hline
Pays & Nombre de cas & Nombre de mort  \\
\hline
Monde Entier & 775,567 M & 7,057 M  \\
\hline
UE & 185,823 M & 1,263 M  \\
\hline
Usa & 103,437 M & 1,193 M  \\
\hline
Chine & 99,373 M & 122,304 k  \\
\hline
Inde & 35,042 M & 533,623 k  \\
\hline
France & 38,997 M & 168,091 k \\
\hline
Allemagne & 38,438 M & 174,878 k  \\
\hline
Brésil & 37,512 M & 702,116 k  \\
\hline
Corée du sud & 34,572 M & 35,934 k  \\
\hline
Japon & 33,804 M & 74,693 k  \\
\hline
Italie & 26,781 M & 197,303 k  \\
\hline
Royaume Uni & 24,975 M & 232,112 k \\
\hline
Russie & 24,269 M & 403,188 k  \\
\hline
Turquie & 17,005 M & 101,419 k  \\
\hline
Espagne & 13,980M & 121,852 k  \\
\hline
Australie & 11,861 M & 25,236 k  \\
\hline
Vietnam & 11,624 M & 43,206 k  \\
\hline
Argentine & 10,101 M & 130,663 k  \\
\hline
\end{tabular}
\caption{Sars-Cov-2 : nombre de cas et de morts (au 15/09/2024)}
\label{tab:Covid_2021}
\end{table}


\subsubsection{Le Coronavirus Khosta-2}

\begin{figure}[h!]
\centering
\includegraphics[width=0.6\textwidth]{Introduction/Image/khosta2.jpg}
\caption{Image d'illustration de virus Khosta-2 dans un organisme hôte}
\end{figure}

\textbf{La découverte en 2020 d'un nouveau virus proche du Covid inquiète les scientifiques.} De la même famille que le Sars-CoV-2 et découvert sur des chauves-souris en Russie, il laisse planer l'angoisse d'une nouvelle épidémie.

\textit{D'après les experts, ce coronavirus serait non seulement capable de se répliquer chez les humains, mais il pourrait aussi contourner la protection immunitaire apportée par les actuels vaccins contre le Covid-19.} Malgré tout, les scientifiques se veulent rassurants : \textbf{le Khosta-2 semblerait ne pas provoquer de formes graves chez l'homme.}

\section{L'Émergence Accélérée de Nouvelles Maladies}

\textbf{Plus généralement, en soixante ans, plus de 350 nouvelles maladies infectieuses sont apparues : une émergence de virus qui semble s'accélérer.} Les exemples incluent le Virus Nipah, la variole du singe, et le virus Langya.

Les experts s'interrogent :
\begin{itemize}
\item \textbf{Quand une nouvelle épidémie mondiale (pandémie) va-t-elle frapper ?}
\item \textbf{Pourra-t-on y faire face ?}
\end{itemize}

\textit{Par exemple, récemment, un virus géant nommé Mollivirus sibericum vient d'être trouvé dans les sols gelés de Sibérie par une équipe de chercheurs franco-russe.} Ce type de virus peuvent être facilement pris pour des bactéries par leur taille.

\section{La Modélisation Mathématique des Épidémies}

\subsection{Développement Historique}

Devant l'ampleur des phénomènes épidémiques, \textbf{les épidémiologistes se sont demandés s'ils ne pouvaient pas utiliser des méthodes mathématiques pour rendre plus efficaces les recherches médicales.} Ils se demandèrent en particulier si, derrière l'évolution d'un processus de contagion, ne se cachaient pas des lois.

\textbf{La réponse à cette question fut positive.} Les épidémiologistes construisirent des \textit{modèles supposés traduire schématiquement certains aspects de la propagation d'une maladie contagieuse.

\subsubsection{Les Premiers Modèles}

\textbf{Les premiers modèles ont commencé au début du 20ème siècle (1906) avec un modèle pour les épidémies de rougeole.} En 1911 et 1917, on voit apparaître les premiers modèles utilisant les équations différentielles pour le paludisme.

\textbf{L'épidémiologie mathématique a marqué sa première victoire en 1927, lorsque W.O. Kermack, médecin en santé publique et A.G McKendrick, biochimiste, ont mis au point un modèle simple de la propagation des épidémies.} Ce modèle fut validé sur les épidémies de peste en Inde.

\textit{Selon ce modèle, lorsqu'un infectant venant de l'extérieur s'infiltre dans une population, la maladie peut se propager soudainement puis disparaître de façon tout aussi soudaine sans infecter toute la communauté.} C'est ce que l'on observe dans d'innombrables épidémies au cours des siècles.

\textbf{Kermack et McKendrick présentent les premiers modèles qui mettent en évidence un effet de seuil pour la propagation de l'épidémie.} Les premiers modèles stochastiques apparaissent en 1928 (Reed et Frost).

\subsubsection{Conclusion}

\textbf{Les maladies respiratoires, le Sida, les maladies diarrhéiques, la tuberculose, le paludisme et la rougeole représentent 90\% des décès par maladies infectieuses dans le monde.} Face à cette réalité, la recherche médicale et mathématique continue d'évoluer pour mieux comprendre et contrôler ces phénomènes.

\textit{À l'heure actuelle, la littérature est très abondante avec à la fois des articles mathématiques et d'autres ayant des applications directes en Santé Publique.} La modélisation mathématique des épidémies est devenue un outil indispensable pour anticiper et gérer les crises sanitaires futures.

L'histoire nous enseigne que les épidémies ont toujours accompagné l'humanité, mais les outils modernes de surveillance, de modélisation et de intervention nous donnent aujourd'hui des moyens sans précédent pour les combattre efficacement.

\subsection{La modélisation}
La \textbf{modélisation d'un phénomène} peut être réalisée de façon \textit{déterministe} ou \textit{stochastique}, et s'appuyer sur une échelle de \textit{temps continue ou discrète}.

\begin{itemize}
\item Les \textbf{modèles déterministes} en \textit{temps continu} sont particulièrement adaptés à l'étude de \textit{grandes populations}. Ils prennent la forme d'un \textit{système d'équations différentielles}, dont le comportement peut dans la plupart des cas être \textit{analysé mathématiquement} et \textit{visualisé par ordinateur}.

\item À l'inverse, les \textbf{modèles stochastiques} se révèlent plus pertinents pour de \textit{petites populations}. Leur étude est plus complexe sur le plan mathématique et leur comportement doit souvent être \textit{exploré à l'aide de simulations numériques}. L'intérêt principal réside dans la \textit{comparaison} de la \textit{robustesse des résultats} obtenus avec ceux des \textit{modèles déterministes}.
\end{itemize}

De manière générale, toute modélisation commence par la \textbf{description précise du problème à étudier}, suivie de sa \textit{mise en équations à partir d'}\textbf{hypothèses} \textit{plus ou moins simplificatrices}. Vient ensuite l'\textbf{analyse du modèle}, soit par des \textit{techniques mathématiques}, soit par \textit{simulation}, et enfin \textbf{l'interprétation du comportement observé}, en fonction des \textit{paramètres identifiés et estimés}.

\subsection{But de la modélisation}
Lorsque l'expérimentation directe n'est plus possible, les modèles deviennent un outil \textbf{indispensable}. Ils permettent tout d'abord de \textbf{mieux comprendre les évolutions observées et de comparer différents phénomènes}, comme par exemple l'évolution de plusieurs maladies. Ils offrent aussi la possibilité de mettre en parallèle \textbf{l'efficacité de différentes méthodes de contrôle}, soit de \textit{manière théorique}, soit grâce à des \textit{simulations réalisées pour diverses valeurs de paramètres et sur la base de données variées}.

Un autre intérêt majeur réside dans la capacité des modèles à orienter le \textbf{choix de stratégies optimales}, par exemple dans la \textit{distribution de vaccins en vue d'éradiquer certaines maladies}. Ils contribuent également à \textbf{identifier les informations qu'il est nécessaire de collecter ou d'enregistrer}, et permettent de \textbf{prédire l'étendue et la taille potentielles d'épidémies}. Enfin, la modélisation oblige le chercheur à \textbf{clarifier et préciser les hypothèses formulées}, ce qui constitue en soi un gain méthodologique essentiel.

\subsection{Hypothèses et but de la modélisation}

L’étude d’une épidémie s’intéresse principalement à \textbf{l’évolution des relations entre deux groupes d’individus au sein d’une population donnée} : d’une part les \textit{malades infectieux}, et d’autre part les \textit{individus susceptibles d’être contaminés}. Le but des modélisations en épidémiologie est de \textbf{prévoir l’évolution d’une maladie} afin de pouvoir ensuite \textit{proposer des thérapies adaptées}. Celles-ci peuvent prendre la forme d’un \textit{traitement curatif de masse} (administration médicamenteuse, mesures d’hygiène ou diététiques) et/ou de \textit{mesures préventives}, comme la vaccination lorsque cela est possible.

Selon Bailey \cite{bailey1975}, \textbf{l’étude d’une épidémie repose} en pratique sur \textit{l’analyse d’un échantillon}, à partir duquel on tente ensuite de construire un \textit{modèle généralisable à l’ensemble de la population}.

Pour introduire la démarche, considérons d’abord une situation simple : un \textbf{petit groupe d’individus infectés est inséré dans une large population d’individus encore sains mais susceptibles de contracter la maladie}. La question centrale est alors : \textit{que va-t-il se passer ? L’infection va-t-elle s’éteindre rapidement ou au contraire se propager ? Peut-elle aboutir à un état endémique, c’est-à-dire au maintien durable d’un sous-groupe d’individus porteurs de la maladie ?}

\subsection{Notations}

À \textit{chaque instant}, la population est subdivisée en \textbf{groupes disjoints}, chacun correspondant à un \textit{statut différent} vis-à-vis de la maladie :

\begin{itemize}
\item \textbf{S (susceptibles) }: individus qui ne sont \textbf{pas} encore malades mais qui \textit{peuvent le devenir}.

\item \textbf{I (infectieux) }: individus malades \textbf{capables de transmettre l’infection}.

\item \textbf{R (retirés) }: individus qui ont \textit{contracté la maladie} et qui \textbf{ne peuvent plus la transmettre}. Ils peuvent être considérés comme \textit{immunisés} de façon \textit{permanente, isolés, ou encore décédés}. Ce groupe n’est toutefois pertinent que pour certaines maladies.
\end{itemize}

Dans certains cas, il est nécessaire de distinguer davantage les \textbf{stades de l’infection}. On introduit alors :

\begin{itemize}
\item une \textbf{période de latence}, correspondant au temps écoulé entre le contact initial et le moment où l’individu devient contagieux,

\item une \textbf{période d’incubation}, qui est le délai entre l’infection effective et l’apparition des symptômes. tte distinction amène à définir deux sous-groupes supplémentaires, insérés entre les classes S et I, afin de mieux représenter la dynamique réelle de certaines maladies.
\end{itemize}


\begin{figure}[h!]
\centering
\includegraphics[width=15cm, height=10cm]{Introduction/Image/Structure.jpg}
\caption{Structure d'une épidemie}
\end{figure}

\begin{table}[h!]
\centering
\small % Réduire la taille de police
\begin{tabular}{|p{5cm}|p{5cm}|p{6cm}|}
\hline
\textbf{Maladie} & \textbf{Période d'incubation} & \textbf{Durée de la maladie} \\
\hline
Choléra & habituellement 2h à 5 j & en moyenne 5 j \\
\hline
Coqueluche (Whooping cough) & habituellement 5-6 j & isolation pendant les 5 premiers jours \\
\hline
Covid-19 & 4-6 j & en moyenne 15 j \\
\hline 
Dengue (dengue fever) & 8-10 j & 7 j \\
\hline
Ebola & 2-21 j & --- \\
\hline
Grippe (Influenza) & 1-3 j & limitée à 3 j dès l'apparition des symptômes \\
\hline
Diphtérie (Diphteria) & 2-5 j & 1-2 j après la prise de traitement \\
\hline
Malaria & --- & --- \\
\hline 
Méningite (meningitis) & 1-3 j & --- \\
\hline
Oreillons (Mumps) & 14-18 j & --- \\
\hline
Poliomyélite & 3-35 j (7-14 j pour la forme paralytique) & Le virus persiste dans la gorge 1 semaine après le début et reste dans les fèces 3-6 semaines \\
\hline
Rougeole (Measles) & 14-18 j & infectieuse depuis l'apparition des symptômes respiratoires jusqu'à 4 j après l'apparition des rougeurs \\
\hline
Rubéole (German measles) & 12-23 j (habituellement 16-18 j) & infectieuse de 7 j avant à 5 j après l'apparition des rougeurs \\
\hline
Variole (Chickenpox) & 10-21 j (habituellement 14-16 j) & habituellement 5-6 j \\
\hline
Variole (Smallpox) & 7-19 j (habituellement 12 j) & 8-9 j en moyenne \\
\hline
\end{tabular}
\caption{Données épidémiologiques des principales maladies infectieuses}
\label{tab:maladies}
\end{table}

\subsection{ Mécanismes de transmission de l’infection }

Le \textbf{mécanisme de transmission} est aujourd’hui bien connu pour la \textit{plupart} des maladies. 

De manière générale, les maladies d’origine \textbf{virale} – comme la \textit{grippe, la rougeole, la rubéole ou encore la varicelle} – confèrent à l’individu une \textbf{immunité durable contre une réinfection}. En revanche, de nombreuses maladies \textbf{bactériennes}, telles que la \textit{tuberculose, la méningite ou la gonorrhée,} ne confèrent pas une telle immunité : une personne peut donc être \textbf{réinfectée plusieurs fois au cours de sa vie}.

\bigskip

Certaines maladies, comme la \textbf{malaria}, ne se transmettent \textit{pas} directement d’un individu à l’autre, mais \textbf{nécessitent l’intervention d’agents vecteurs}, le plus souvent des \textit{insectes}. 

Dans ce cas, la \textbf{transmission interhumaine} ne peut avoir lieu qu’après qu’un \textbf{vecteur a contaminé un individu}, par exemple par une \textit{piqûre}, une \textit{transfusion sanguine}, ou encore par \textit{transmission de la mère à l’enfant pendant la grossesse}.

\bigskip

Les maladies transmissibles comme la \textit{rougeole, la grippe ou la tuberculose} sont étroitement liées aux conditions de la \textit{vie moderne} et à la \textit{densité des populations}. 

En épidémiologie, on distingue les \textbf{épidémies}, qui sont des \textit{manifestations soudaines d’une maladie}, et les \textbf{états endémiques}, correspondant à la \textit{présence persistante de la maladie dans une population}. 

Dans les pays dits « sous-développés », des millions de personnes meurent chaque année de la rougeole, d’infections respiratoires, de la diarrhée et d’autres affections pourtant considérées comme bénignes et facilement traitables dans les pays dits « développés ».

\bigskip

D’autres maladies comme la \textit{malaria, le typhus, le choléra, la bilharziose ou schistosomiase} (maladie provoquée par les bilharrzies, vers vivant en parasite dans l'appareil circulatoire de l'homme, et transmise par leurs oeufs) et \textit{la maladie du sommeil} (sleeping sickness en anglais) restent \textbf{fortement endémiques dans certaines régions du globe}. 

Avant de mettre en place un modèle, il est donc indispensable de disposer d’une \textbf{connaissance précise et réaliste de la biologie de la maladie étudiée}.  Il faut en particulier considérer :

\begin{itemize}
\item la durée de la période d’infectiosité 
\item la durée de la période d’incubation
\item le statut immunitaire acquis ou non après l’infection.
\end{itemize}

\bigskip

La \textbf{deuxième étape d’une modélisation} consiste à \textit{collecter les données nécessaires sur les caractéristiques démographiques, épidémiologiques et biologiques}. Cela inclut notamment les \textbf{taux de transition liés à l’infection}, mais aussi les \textbf{paramètres propres à la population étudiée}, tels que les \textit{taux de naissance et de mortalité}.

\bigskip

La troisième étape concerne le choix parcimonieux du modèle. 

\begin{figure}[h!]
\centering
\includegraphics[width=13cm, height=9cm]{Introduction/Image/Model_si_sir_seir.jpg}
\caption{Évolution du modèle SI, SIR et SEIR dans le temps}
\end{figure}

Parmi les modèles les plus utilisés en épidémiologie figurent les \textit{modèles compartimentaux}. 

\begin{itemize}
\item Le \textbf{modèle SEIR} est par exemple adapté à la \textit{rougeole}, car il prend en compte l’existence d’une \textbf{période de latence} (d’environ huit jours pour cette maladie) entre \textit{l’infection et le début de la contagiosité}. 
\item Lorsque cette période de latence n’existe \textbf{pas}, c’est-à-dire lorsque les individus deviennent \textit{infectieux immédiatement après avoir été infectés}, on préfère utiliser le \textbf{modèle SIR}, qui constitue une \textit{alternative simplifiée} (à SEIR).
\item Le \textbf{modèle S(E)IR} suggère, quant à lui, une \textit{longue période d’immunité} : c’est le cas de maladies comme la \textit{rougeole}, où il n’y a pas de transition de la classe R (retirés) vers la classe S (susceptibles). Dans les situations où cette immunité n’est que \textit{temporaire}, il faut au contraire choisir un modèle permettant le retour de R vers S, comme le \textbf{modèle S(E)IRS}. Le cas le plus simple de ce type est le \textbf{modèle SIS}, utilisé par exemple pour représenter la \textit{tuberculose}.
\end{itemize}

\subsubsection{Un exemple historique : la grande peste de Londres}

La \textit{première épidémie} probablement étudiée sous l’angle de la \textit{modélisation} fut la \textbf{grande peste de Londres (1665-1666)}. Cette épidémie causa la \textit{mort d’environ un sixième de la population} et entraîna la fermeture de l’université de Cambridge pendant deux ans. À cette époque, Isaac Newton, alors étudiant à Cambridge, se réfugia chez lui pour échapper à la contagion ; c’est durant cette période qu’il formula la loi de la gravitation universelle.

Le mécanisme typique de la peste, lorsqu’elle \textit{surgit brutalement}, est une \textbf{croissance rapide en intensité}, suivie d’une \textbf{disparition laissant néanmoins une part importante de la population non infectée}. Ce même schéma a été observé dans de nombreuses autres épidémies, qu’elles soient mortelles ou qu’elles aboutissent à une immunisation définitive des survivants.

\bigskip

Une des premières questions qui a retenu l’attention des scientifiques est de comprendre \textbf{pourquoi une maladie se développe soudainement dans une communauté}. 

L’un des grands succès fondateurs de l’épidémiologie mathématique fut la formulation, par \textbf{Kermack et McKendrick en 1927}, d’un \textit{modèle simple capable de prédire un comportement très proche de celui observé dans de nombreuses épidémies réelles}.

La \textbf{vitesse de progression} d’une épidémie dépend étroitement des \textit{caractéristiques biologiques de la maladie}. Ainsi, dans le cas de la \textbf{grippe}, la période de latence est estimée entre un et trois jours, tandis que la durée d’infection s’étend généralement de deux à trois jours. Dans ces conditions, une épidémie de grippe peut se \textit{propager à l’échelle d’une ville entière en moins de six semaines}.

\subsection{ Hypothèses pour les modèles SI, SIS et SIR} 

Pour pouvoir modéliser efficacement une épidémie, il est indispensable d’adopter des \textbf{hypothèses simplificatrices}. La \textit{formalisation mathématique du modèle permet d’expliciter ces hypothèses, de mettre en évidence leurs implications et, en même temps, de souligner les limites qu’elles entraînent}.

Nous retiendrons les hypothèses suivantes :

\begin{enumerate}
\item \textbf{La transmission de la maladie se fait par contact direct entre individus infectés et individus susceptibles.} Cette hypothèse convient à de \textit{nombreuses maladies virales} comme la grippe, la rougeole ou le SIDA, ainsi qu’à des \textit{maladies bactériennes comme la tuberculose}. En revanche, elle n’est \textbf{pas} adaptée aux maladies nécessitant un \textit{vecteur animal} (par exemple la rage ou la malaria).
\item \textbf{ Absence de temps de latence} : les individus deviennent \textit{immédiatement infectieux dès leur contamination}.
\item \textbf{Homogénéité des contacts} : on suppose que tous les \textit{individus susceptibles le sont de manière uniforme et que tous les infectés transmettent la maladie avec la même intensité}. Cette hypothèse correspond à un « mélange homogène », dont on déduit une \textit{loi d’action de masse} : le nombre de nouveaux cas est alors \textbf{proportionne}l à la fois à la \textit{taille du groupe des susceptibles (S)} et à celle du \textit{groupe des infectieux (I)}.
\item \textbf{Population totale constante} : la taille de la population reste fixée à une valeur N. Dans une version plus générale, on peut intégrer des \textit{naissances et des morts}, mais il est souvent supposé que ces deux \textit{phénomènes s’équilibrent}. Cette hypothèse est surtout pertinente lorsque l’on s’intéresse à l’évolution d’une \textit{épidémie sur une courte période de temps, comme c’est le cas pour la grippe ou certaines maladies infantiles}.
\end{enumerate}

\section{Action de masse}

On considère une ville de $100\,000$ habitants. On note l’ensemble de la population par $\mathcal{P}$, de sorte que

\[ Card(\mathcal{P}) = 10^5. \]

La population reste constante au cours du temps : il n’y a donc ni naissances, ni décès. Parmi cette population, certains individus peuvent être malades.

Supposons qu’un individu infecté, noté $i_0$, rencontre en moyenne $10$ personnes par jour.
Comme la population totale est de $10^5$ habitants, la probabilité que $i_0$ rencontre un individu donné est

\[ p = \frac{10}{10^5} = 10^{-4}. \]

Si l’on note $S$ le nombre d’individus susceptibles (c’est-à-dire non infectés mais pouvant contracter la maladie), alors l’individu infecté $i_0$ peut espérer entrer en contact, par jour, avec en moyenne

\[ pS \]

individus susceptibles.

Ainsi, $pS$ représente le nombre moyen quotidien de contacts entre un individu infectieux et des individus susceptibles.

On suppose maintenant que $q = 0.1$ désigne la probabilité qu’un individu susceptible contracte effectivement l’infection lors d’un contact avec un individu infectieux.
Le nombre moyen de nouveaux infectés produits par $i_0$ au cours d’une journée est alors

\[ pqS. \]

En particulier, avec $p = 10^{-4}$, on obtient

\[ pqS = 10^{-5} \cdot S. \]

En posant 

\[ \lambda = pq, \]

on obtient la loi d’infection dite « de masse » :

\[ \lambda S I \]

représente le nombre moyen journalier de nouveaux infectés dans la population, où $I$ désigne le nombre d’individus déjà infectés.

\documentclass[12pt,a4paper]{book}

% ===== ENCODAGE ET LANGUE =====
\usepackage[utf8]{inputenc}
\usepackage[T1]{fontenc}
\usepackage[french]{babel}
\usepackage{lmodern}

% ===== MISE EN PAGE =====
\usepackage[left=2.5cm, right=2.5cm, top=2.5cm, bottom=2.5cm]{geometry}
\usepackage{setspace}
\onehalfspacing

% ===== MATHÉMATIQUES =====
\usepackage{amsmath, amssymb, amsthm}
\usepackage{mathtools}
\usepackage{bm}

% ===== FIGURES, TABLEAUX, CODE =====
\usepackage{graphicx}
\usepackage{caption}
\usepackage{subcaption}
\usepackage{float}
\usepackage{booktabs}
\usepackage{array}
\usepackage{multirow}
\usepackage{xcolor}
\usepackage{listings}

\lstloadlanguages{Python, R, SQL}

% ===== BOÎTES MINIMALISTES =====
\usepackage{mdframed}

% ===== STRUCTURE ET NAVIGATION =====
\usepackage{titlesec}
\usepackage{titletoc}
\usepackage{fancyhdr}
\setlength{\headheight}{14.5pt}
\usepackage{hyperref}
\hypersetup{
    colorlinks=true,
    linkcolor=black,
    citecolor=blue,
    urlcolor=blue,
    bookmarks=true,
    bookmarksopen=true
}

% ===== LISTES PERSONNALISÉES =====
\usepackage{enumitem}
\setlist[itemize]{itemsep=0.3em, topsep=0.5em}
\setlist[enumerate]{itemsep=0.3em, topsep=0.5em}

% ===== EN-TÊTES ET PIEDS =====
\pagestyle{fancy}
\fancyhf{}
\fancyhead[L]{\leftmark}
\fancyhead[R]{\thepage}
\renewcommand{\headrulewidth}{0.4pt}

% ===== NUMÉROTATION DES ÉQUATIONS =====
\numberwithin{equation}{chapter}

% ===== DÉFINITION STYLES DE LISTINGS =====
% --- Packages nécessaires ---
\usepackage[most]{tcolorbox}
\usepackage{listings}
\usepackage{xcolor}

% --- Couleurs des headers ---
% Python
\definecolor{pyhead}{RGB}{30,60,160}        % bleu foncé
\definecolor{pyheadlight}{RGB}{180,200,255} % bleu clair output

% R
\definecolor{rhead}{RGB}{180,20,20}         % rouge foncé
\definecolor{rheadlight}{RGB}{255,200,200}  % rouge clair output

% SQL
\definecolor{sqlhead}{RGB}{40,140,230}       % bleu clair
\definecolor{sqlheadlight}{RGB}{200,230,255} % bleu très clair output

% --- Style listings : texte noir, syntaxe sobre ---
\lstset{
    basicstyle=\ttfamily\small\color{black},
    commentstyle=\color{gray}\itshape,
    stringstyle=\color{gray!70},
    keywordstyle=\bfseries\color{black},
    breaklines=true,
    numbers=left,
    numberstyle=\tiny\color{gray},
}

% ====== PYTHON ======
\newtcolorbox{pycodebox}{
    colback=white,
    colframe=pyhead,
    fonttitle=\bfseries\ttfamily\small,
    coltitle=white,
    title=Python,
    enhanced,
    sharp corners,
    top=2pt, bottom=2pt, left=2pt, right=2pt,
    attach boxed title to top left={
        xshift=2pt, yshift*=-\tcboxedtitleheight
    },
    boxed title style={
        colback=pyhead,
        colframe=pyhead,
        boxrule=0.8pt,
        sharp corners
    }
}

\newtcolorbox{pyoutputbox}{
    colback=white,
    colframe=pyheadlight,
    fonttitle=\bfseries\ttfamily\small,
    coltitle=black,
    title=Python~Output,
    enhanced,
    sharp corners,
    top=2pt, bottom=2pt, left=2pt, right=2pt,
    attach boxed title to top left={
        xshift=2pt, yshift*=-\tcboxedtitleheight
    },
    boxed title style={
        colback=pyheadlight,
        colframe=pyheadlight,
        boxrule=0.8pt,
        sharp corners
    }
}

% ====== R ======
\newtcolorbox{rcodebox}{
    colback=white,
    colframe=rhead,
    fonttitle=\bfseries\ttfamily\small,
    coltitle=white,
    title=R,
    enhanced,
    sharp corners,
    top=2pt, bottom=2pt, left=2pt, right=2pt,
    attach boxed title to top left={
        xshift=2pt, yshift*=-\tcboxedtitleheight
    },
    boxed title style={
        colback=rhead,
        colframe=rhead,
        boxrule=0.8pt,
        sharp corners
    }
}

\newtcolorbox{routputbox}{
    colback=white,
    colframe=rheadlight,
    fonttitle=\bfseries\ttfamily\small,
    coltitle=black,
    title=R~Output,
    enhanced,
    sharp corners,
    top=2pt, bottom=2pt, left=2pt, right=2pt,
    attach boxed title to top left={
        xshift=2pt, yshift*=-\tcboxedtitleheight
    },
    boxed title style={
        colback=rheadlight,
        colframe=rheadlight,
        boxrule=0.8pt,
        sharp corners
    }
}

% ====== SQL ======
\newtcolorbox{sqlcodebox}{
    colback=white,
    colframe=sqlhead,
    fonttitle=\bfseries\ttfamily\small,
    coltitle=white,
    title=SQL,
    enhanced,
    sharp corners,
    top=2pt, bottom=2pt, left=2pt, right=2pt,
    attach boxed title to top left={
        xshift=2pt, yshift*=-\tcboxedtitleheight
    },
    boxed title style={
        colback=sqlhead,
        colframe=sqlhead,
        boxrule=0.8pt,
        sharp corners
    }
}

\newtcolorbox{sqloutputbox}{
    colback=white,
    colframe=sqlheadlight,
    fonttitle=\bfseries\ttfamily\small,
    coltitle=black,
    title=SQL~Output,
    enhanced,
    sharp corners,
    top=2pt, bottom=2pt, left=2pt, right=2pt,
    attach boxed title to top left={
        xshift=2pt, yshift*=-\tcboxedtitleheight
    },
    boxed title style={
        colback=sqlheadlight,
        colframe=sqlheadlight,
        boxrule=0.8pt,
        sharp corners
    }
}

% ===== COMMANDES PERSONNALISÉES MATHS =====
\newcommand{\dd}{\mathrm{d}}
\newcommand{\R}{\mathbb{R}}
\newcommand{\N}{\mathbb{N}}
\newcommand{\Z}{\mathbb{Z}}
\newcommand{\Q}{\mathbb{Q}}
\newcommand{\C}{\mathbb{C}}

% ===== COULEURS DOUCES =====
\definecolor{redsoft}{RGB}{230, 100, 100}
\definecolor{redbg}{RGB}{255, 245, 245}
\definecolor{yellowsoft}{RGB}{220, 180, 80}
\definecolor{yellowbg}{RGB}{255, 252, 240}
\definecolor{graysoft}{RGB}{140, 140, 140}
\definecolor{graybg}{RGB}{248, 248, 248}

% ===== STYLES DES BOÎTES =====
\mdfdefinestyle{thmstyle}{
    linecolor=redsoft,
    linewidth=1.5pt,
    backgroundcolor=redbg,
    innerleftmargin=8pt,
    innerrightmargin=8pt,
    innertopmargin=6pt,
    innerbottommargin=6pt,
    skipabove=0.8em,
    skipbelow=0.8em
}

\mdfdefinestyle{exstyle}{
    linecolor=yellowsoft,
    linewidth=1.5pt,
    backgroundcolor=yellowbg,
    innerleftmargin=8pt,
    innerrightmargin=8pt,
    innertopmargin=6pt,
    innerbottommargin=6pt,
    skipabove=0.8em,
    skipbelow=0.8em
}

\mdfdefinestyle{remstyle}{
    linecolor=graysoft,
    linewidth=1.5pt,
    backgroundcolor=graybg,
    innerleftmargin=8pt,
    innerrightmargin=8pt,
    innertopmargin=6pt,
    innerbottommargin=6pt,
    skipabove=0.8em,
    skipbelow=0.8em
}

% ===== ENVIRONNEMENTS =====
\theoremstyle{definition}
\newmdtheoremenv[style=thmstyle]{definition}{Définition}[chapter]
\newmdtheoremenv[style=thmstyle]{notation}{Notation}[chapter]
\newmdtheoremenv[style=exstyle]{exemple}{Exemple}[chapter]
\newmdtheoremenv[style=remstyle]{remarque}{Remarque}[chapter]
\newmdtheoremenv[style=remstyle]{note}{Note}[chapter]

\theoremstyle{plain}
\newmdtheoremenv[style=thmstyle]{theoreme}{Théorème}[chapter]
\newmdtheoremenv[style=thmstyle]{proposition}{Proposition}[chapter]
\newmdtheoremenv[style=thmstyle]{lemme}{Lemme}[chapter]
\newmdtheoremenv[style=thmstyle]{corollaire}{Corollaire}[chapter]

% ===== DOCUMENT =====
\begin{document}

\frontmatter
\title{Estimation}
\author{Lorenzo Segoni}
\date{\today}
\maketitle

\tableofcontents

\mainmatter

% Exemple de chapitre
\chapter{Méthodes d'estimation}

\section{Modèles et estimations statistiques}

\subsection{Modèles statistiques}

\subsubsection{Première étape de la démarche statistique : la modélisation}

On observe une donnée $x$.

\begin{definition}[Modèle statistique]
On suppose que $x$ est la réalisation d’une variable aléatoire $X$ de loi $P_X$ (inconnue).  
On postule que $P_X$ appartient à une famille de lois de probabilités spécifiée :
\[
\{ P_\theta, \ \theta \in \Theta \}.
\]
\end{definition}

\begin{remarque}
Un modèle est toujours faux : la réalité est trop complexe pour être décrite parfaitement.  
L’enjeu est que l’approximation apportée par le modèle soit suffisamment fidèle pour répondre à la question posée.  
En pratique, ce sont souvent les spécialistes du domaine (médecine, biologie, économie, etc.) qui proposent ou discutent le modèle.
\end{remarque}

\subsubsection{Deuxième étape : l’inférence statistique}

À partir du modèle $\{P_\theta, \theta \in \Theta\}$ et d’une observation $X$, on cherche à obtenir des informations sur la loi réelle $P_X$.

\begin{definition}[Modèle statistique]
Un modèle statistique est un triplet :
\[
(E, \mathcal{E}, (P_\theta)_{\theta \in \Theta})
\]
où :
\begin{itemize}
  \item $(E, \mathcal{E})$ est un espace mesurable (dans ce cours, on retiendra surtout que $X \in E$) ;
  \item $(P_\theta)_{\theta \in \Theta}$ est une famille de lois de probabilité sur $E$ ;
  \item $\Theta$ est l’ensemble des paramètres.  
\end{itemize}

Si $\Theta \subset \mathbb{R}^d$ avec $d \geq 1$, le modèle est dit \textbf{paramétrique}.  
Sinon, il est dit \textbf{non paramétrique}.
\end{definition}

\begin{exemple}
\begin{enumerate}
  \item \textbf{Jeu de pile ou face répété $n$ fois :}
  \[
  E = \{0,1\}^n, \quad 
  \Theta = ]0,1[, \quad
  P_\theta = \big(\mathcal{B}(\theta)\big)^{\otimes n}.
  \]
  Une observation est $X = (X_1,\ldots,X_n) \in {0,1}^n$, avec $X_i$ des variables aléatoires indépendantes de loi $\mathcal{B}(\theta)$.

  \item \textbf{Modélisation de la taille d’individus par une loi normale.} On observe sur un échantillon de taille n:
  \[
  E = \mathbb{R}^n, \quad
  X = (X_1,\ldots,X_n), \quad
  X_i \text{ variables aléatoires indépendantes de loi } \mathcal{N}(m,\sigma^2),
  \]
  avec 
  \[
  \Theta = \{ (m,\sigma^2) : m \in \mathbb{R}, \ \sigma^2 >0 \} \subset \mathbb{R}^2.
  \]

  \item \textbf{Même population, mais modélisation par une loi à densité sur $[0.5,2.5]$:}
  \[
  \Theta = \{ f : [0.5,2.5] \to \mathbb{R}_+ \ \mid f \text{ est une densité} \}.
  \]
  Il s’agit alors d’un \textbf{modèle non paramétrique}.
\end{enumerate}
\end{exemple}

\paragraph{Valeur vraie du paramètre.}  
Le point clé de la modélisation est qu’il existe une valeur \emph{vraie} $\theta^* \in \Theta$ telle que :
\[
P_X = P_{\theta^*}.
\]
Mais $\theta^*$ est inconnue

Dans toute la suite, on notera simplement $\theta$ pour $\theta^*$.

On utilisera les notations :
\[
\mathbb{P}_\theta \quad \text{et} \quad \mathbb{E}_\theta
\]
pour bien marquer la dépendance en $\theta$. Et on a que X dépende de $\theta$

\begin{align*}
P(X \in A) &= P_X(A) = P_{\theta^*}(A), \\
P_\theta(X \in A) &= P_\theta(A).
\end{align*}

\begin{remarque}
On considérera souvent une observation $X$ comme un $n$-échantillon de variables aléatoires $iid$ de loi $Q_\theta$ :
\[
P_\theta = (Q_\theta)^{\otimes n}.
\]
On peut alors définir le modèle statistique comme la famille $\{Q_\theta, \theta \in \Theta\}$.
\end{remarque}


\begin{definition}[Identifiabilité]
Un modèle $\{P_\theta, \theta \in \Theta\}$ est dit \textbf{identifiable} si :
\[
P_{\theta_1} = P_{\theta_2} \ \Rightarrow \ \theta_1 = \theta_2.
\]
\end{definition}

\subsection{Estimateur statistique}

On suppose que le modèle $\{P_\theta, \theta \in \Theta\}$ est identifiable.  
On observe $X$ et on s’intéresse à une quantité d’intérêt $g(\theta)$, où :
\[
g : \Theta \to \mathbb{R}^p, \quad p \geq 1.
\]

\begin{definition}[Statistique et estimateur]
\begin{itemize}
  \item Une \textbf{statistique} $T(X)$ est une fonction mesurable de $X$ (éventuellement de paramètres connus), qui ne dépend pas de $\theta$.  
  \item Un \textbf{estimateur} de $g(\theta)$ est une statistique $\hat{g}=h(X)$ à valeurs dans $\mathbb{R}^p$ (avec $h$ mesurable).
\end{itemize}
\end{definition}

\begin{exemple}
Pour un $n$-échantillon $(X_1,\ldots,X_n)$, les objets suivants sont des statistiques :  
\[
\overline{X}_n, \quad X_1, \quad X_2^2 \cdot X_8, \quad 0.
\]
\end{exemple}

\begin{remarque}
Si l’on dispose d’une réalisation $x$ de $X$, alors $\hat{g}(x)$ est appelé un \textbf{estimé} de $g(\theta)$.  
Par abus, on note souvent $\hat{g}$ à la fois l’estimateur (variable aléatoire) et son estimé (valeur numérique).
\end{remarque}

\begin{notation}[Estimateur plug-in]
Si $\hat{\theta}$ est un estimateur de $\theta$, un estimateur naturel de $g(\theta)$ est :
\[
g(\hat{\theta}).
\]
On appelle cela un \textbf{estimateur plug-in}.  
Sa qualité dépend de la qualité de $\hat{\theta}$ et de la régularité de $g$.
\end{notation}


\begin{definition}[Risque quadratique]
Le risque quadratique de $\hat{g}$ est défini, pour tout $\theta \in \Theta$, par :
\[
R(\hat{g}, g(\theta)) = \mathbb{E}_\theta \big[ (\hat{g} - g(\theta))^2 \big] \in [0, + \infty].
\]
\end{definition}

\begin{remarque} 
\textbf{Lien avec inégalité de Markov: }
Par l’inégalité de Markov,
\[
\mathbb{P}_\theta \big( |\hat{g}-g(\theta)| > \varepsilon \big) \leq \frac{R(\hat{g},g(\theta))}{\varepsilon^2}.
\]
Ainsi, plus le risque quadratique est petit, plus l’estimateur est concentré autour de la valeur vraie.
\end{remarque}

\paragraph{Existe-t-il un estimateur optimal ?}  

On pourrait souhaiter qu’il existe un estimateur $\hat{g}$ meilleur que tous les autres pour tout $\theta$, c’est-à-dire :

\[ \exists ? \hat{g} \text{ tel que } \forall \theta \in \Theta, \forall \hat{g}' \text{ estimateur }, R(\hat{g}, g(\theta)) \leq R(\hat{g}', g(\theta)) \]

La réponse est en général \textbf{non}.

\begin{exemple}[Comparaison de deux estimateurs]
Considérons le modèle :
\[
\{ (\mathcal{N}(\theta,1))^{\otimes n}, \ \theta \in \mathbb{R} \}.
\]

Deux estimateurs de $\theta$ :
\[
\hat{\theta}_1 = \overline{X}_n, 
\qquad \hat{\theta}_2 = 0.
\]

On a :
\[
R(\hat{\theta}_1, \theta) = \frac{1}{n}, 
\qquad R(\hat{\theta}_2, \theta) = \theta^2.
\]

De plus :
\[
\mathbb{E}_\theta[\hat{\theta}_1] = \theta, \qquad
\mathrm{Var}_\theta(\hat{\theta}_1) = \frac{1}{n}.
\]

Conclusion :
\begin{itemize}
  \item $\hat{\theta}_1$ est sans biais avec un risque constant $1/n$.
  \item $\hat{\theta}_2$ est biaisé, mais pour $|\theta| < \frac{1}{\sqrt{n}}$, son risque est inférieur à celui de $\hat{\theta}_1$.
\end{itemize}

Ainsi, même un estimateur « stupide » comme $\hat{\theta}_2$ peut être meilleur que l’estimateur usuel $\overline{X}_n$ pour certaines valeurs de $\theta$.  
Bien sûr, plus $n$ est grand, plus l’intervalle de $\theta$ où $\hat{\theta}_2$ est meilleur devient petit.
\end{exemple}


\begin{proposition}[Décomposition biais-variance]
Pour tout $\theta \in \Theta$, on a :
\[
R(\hat{g}, g(\theta)) 
= \big( \mathbb{E}_\theta[\hat{g}] - g(\theta) \big)^2 
+ \mathbb{E}_\theta\!\left[ (\hat{g} - \mathbb{E}_\theta[\hat{g}])^2 \right].
\]
Autrement dit :
\[
R(\hat{g}, g(\theta)) = B(\hat{g}, g(\theta))^2 + \mathrm{Var}_\theta(\hat{g}),
\]
où $B(\hat{g}, g(\theta))$ est le biais de $\hat{g}$.
\end{proposition}


\begin{definition}[Biais]
Le biais d’un estimateur $\hat{g}$ est défini par la fonction:
\[
\theta \mapsto{} B(\hat{g}, g(\theta)) = \mathbb{E}_\theta[\hat{g}] - g(\theta).
\]

On dit que $\hat{g}$ est \textbf{sans biais} si, pour tout $\theta \in \Theta$ :
\[
\mathbb{E}_\theta[\hat{g}] = g(\theta).
\]
\end{definition}












%%%%%%%%%%%%%%%%%%%%%%%%%%%%%%%%%%%%%
--------------------------------------------------------------------------------------------------

\begin{proof}[Idée de la preuve]
On écrit :
\[
(\hat{g} - g(\theta))^2 = \big( \hat{g} - \mathbb{E}_\theta[\hat{g}] + \mathbb{E}_\theta[\hat{g}] - g(\theta) \big)^2,
\]
puis on développe et on prend l’espérance.
\end{proof}

\begin{proof}
On part de l'identité :
\[
\hat{g} - g(\theta)
= \big( \hat{g} - \mathbb{E}_\theta[\hat{g}] \big)
+ \big( \mathbb{E}_\theta[\hat{g}] - g(\theta) \big).
\]
En élevant au carré, on obtient :
\[
(\hat{g} - g(\theta))^2
= \big(\hat{g} - \mathbb{E}_\theta[\hat{g}]\big)^2
+ 2\big(\hat{g} - \mathbb{E}_\theta[\hat{g}]\big)\big(\mathbb{E}_\theta[\hat{g}] - g(\theta)\big)
+ \big(\mathbb{E}_\theta[\hat{g}] - g(\theta)\big)^2.
\]
En prenant l’espérance, le terme croisé disparaît car
\[
\mathbb{E}_\theta[\hat{g} - \mathbb{E}_\theta[\hat{g}]] = 0.
\]
Ainsi :
\[
\mathbb{E}_\theta\big[(\hat{g} - g(\theta))^2\big]
= \mathbb{E}_\theta\big[(\hat{g} - \mathbb{E}_\theta[\hat{g}])^2\big]
+ \big(\mathbb{E}_\theta[\hat{g}] - g(\theta)\big)^2.
\]
On reconnaît la variance et le carré du biais, ce qui donne :
\[
R(\hat{g}, g(\theta)) = \mathrm{Var}_\theta(\hat{g}) + B(\hat{g}, g(\theta))^2.
\]
\end{proof}

\begin{remarque}
Si $\hat{g}$ est sans biais, c’est-à-dire $\mathbb{E}_\theta[\hat{g}] = g(\theta)$,
alors $B(\hat{g}, g(\theta)) = 0$ et
\[
R(\hat{g}, g(\theta)) = \mathrm{Var}_\theta(\hat{g}).
\]
\end{remarque}

--------------------------------------------------------------------------------------------------
%%%%%%%%%%%%%%%%%%%%%%%%%%%%%%%%%%%%%












\paragraph{Interprétation :}
\begin{itemize}
  \item Le \textbf{biais} mesure l’erreur systématique faite par l’estimateur (erreur d’approximation).
  \item La \textbf{variance} mesure les fluctuations aléatoires de l’estimateur autour de sa moyenne (erreur d’estimation).
\end{itemize}

\subsubsection*{Remarques sur les estimateurs sans biais}

Si $\hat{g}$ est sans biais :
\[
R(\hat{g}, g(\theta)) = \mathrm{Var}_\theta(\hat{g}).
\]

On s’intéresse alors aux estimateurs sans biais ayant la variance minimale.  
On appelle un tel estimateur un \textbf{estimateur UVMB} (Uniformément de Variance Minimale parmi les estimateurs Sans Biais).  

Il existe des méthodes (exhaustivité, complétude, théorème de Lehmann-Scheffé) pour caractériser directement de tels estimateurs.  

\begin{remarque}
Un estimateur sans biais n’est pas toujours le meilleur choix :  
un estimateur \textbf{avec biais} peut avoir un risque quadratique plus petit.  
De plus, il n’existe pas toujours d’estimateur sans biais pour une quantité donnée.
\end{remarque}

\begin{exemple}[Impossibilité d’un estimateur sans biais]
Soit $X \sim \mathcal{E}(\theta)$ une loi exponentielle de paramètre $\theta > 0$.  
Supposons qu’il existe un estimateur sans biais $\hat{\theta} = h(X)$ de $\theta$.  
Alors, pour tout $\theta > 0$, il faudrait :
\[
\int_0^\infty h(x) \, \theta e^{-\theta x} \, dx = \theta.
\]

Or, cette condition n’admet pas de solution mesurable $h$, donc un tel estimateur n’existe pas.
\end{exemple}

\begin{remarque}
Même si $\hat{\theta}$ est un estimateur sans biais de $\theta$, en général $g(\hat{\theta})$ n’est pas un estimateur sans biais de $g(\theta)$.  
Ainsi, la propriété « sans biais » n’est pas stable par transformation.
\end{remarque}

\subsection{Propriétés asymptotiques}

On se place avec $n$ observations.
On pose $\hat{g}_n$ un estimateur de $g(\theta)$.


\begin{definition}[Consistance]
On dit que la suite $(\hat{g}_n)_{n \in \mathbb{N}}$ est \textbf{consistante} si pour tout $\theta \in \Theta$,
\[
\hat{g}_n \xrightarrow[\mathbb{P}_\theta]{n \to +\infty} g(\theta).
\]
Elle est dite \textbf{fortement consistante} si pour tout $\theta \in \Theta$,
\[
\hat{g}_n \xrightarrow[\text{p.s.}]{n \to +\infty} g(\theta).
\]
\end{definition}


\begin{remarque}
On dira seulement que $\hat{g}_n$ est consistant si $(\hat{g}_n)_{n \in \mathbb{N}}$ est consistante.

\medskip

\begin{itemize}
    \item La convergence doit avoir lieu pour tout $\theta \in \Theta$.
    \item La consistance est une propriété ``faible''. S'il existe un estimateur consistant, alors il en existe une infinité : si $\hat{g}_n$ est consistant et si $(a_n)_{n \in \mathbb{N}}$ est telle que $a_n \to 1$, alors $a_n \hat{g}_n$ est consistant.
    \item En revanche, les estimateurs non consistants sont à exclure.
\end{itemize}
\end{remarque}


\begin{definition}[Estimateur asymptotiquement sans biais]
On suppose que pour tout $\theta \in \Theta$ et pour tout $n \in \mathbb{N}$,
\[
\mathbb{E}_\theta\left[\|\hat{g}_n\|\right] < +\infty.
\]
On dit que $\hat{g}_n$ est \textbf{asymptotiquement sans biais} si pour tout $\theta \in \Theta$,
\[
\mathbb{E}_\theta[\hat{g}_n] \xrightarrow{n \to +\infty} g(\theta).
\]
\end{definition}


\begin{definition}[Normalité asymptotique]
On dit que $\hat{g}_n$ est \textbf{asymptotiquement normal} si pour tout $\theta \in \Theta$, il existe $\Sigma(\theta) > 0$ tel que
\[
\sqrt{n}\left(\hat{g}_n - g(\theta)\right) \xrightarrow[\text{loi}]{n \to +\infty} \mathcal{N}(0, \Sigma(\theta)).
\]
\end{definition}

\begin{remarque}
\begin{itemize}
    \item Si $\hat{g}_n$ est asymptotiquement normal, alors $\hat{g}_n$ est consistant.
    \item Plus généralement, on peut remplacer ``$\sqrt{n}$'' par $a_n$ où $a_n \xrightarrow{n \to +\infty} +\infty$. On dit alors que $\frac{1}{a_n}$ est la vitesse de convergence asymptotique.
\end{itemize}
\end{remarque}


\begin{proposition}
Soit $\hat{g}_n$ un estimateur asymptotiquement normal de $g(\theta)$ de variance asymptotique $\Sigma(\theta)$.
Soit $\hat{\Sigma}_n$ un estimateur de $\Sigma(\theta)$ consistant, c'est-à-dire pour tout $\theta \in \Theta$,
\[
\hat{\Sigma}_n \xrightarrow[\mathbb{P}_\theta]{n \to +\infty} \Sigma(\theta).
\]
Alors
\[
\frac{\sqrt{n}\left(\hat{g}_n - g(\theta)\right)}{\sqrt{\hat{\Sigma}_n}} \xrightarrow[\text{loi}]{n \to +\infty} \mathcal{N}(0, 1).
\]
\end{proposition}

\begin{proof}
Conséquence du \textbf{lemme de Slutsky}.

On écrit
\[
\frac{\sqrt{n}\left(\hat{g}_n - g(\theta)\right)}{\sqrt{\hat{\Sigma}_n}} = \frac{\sqrt{n}\left(\hat{g}_n - g(\theta)\right)}{\sqrt{\Sigma(\theta)}} \cdot \frac{\sqrt{\Sigma(\theta)}}{\sqrt{\hat{\Sigma}_n}},
\]
avec :
\begin{itemize}
    \item $\displaystyle\frac{\sqrt{n}\left(\hat{g}_n - g(\theta)\right)}{\sqrt{\Sigma(\theta)}} \xrightarrow[\text{loi}]{n \to +\infty} \mathcal{N}(0, 1)$ par hypothèse,
    \item $\displaystyle\frac{\sqrt{\Sigma(\theta)}}{\sqrt{\hat{\Sigma}_n}} \xrightarrow[\mathbb{P}_\theta]{n \to +\infty} 1$ par consistance de $\hat{\Sigma}_n$.
\end{itemize}
Le lemme de Slutsky permet de conclure.
\end{proof}


\section{Propriétés asymptotiques}

On se place avec $n$ observations.
On pose $\hat{g}_n$ un estimateur de $g(\theta)$.


\begin{definition}[Consistance]
On dit que la suite $(\hat{g}_n)_{n \in \mathbb{N}}$ est \textbf{consistante} si pour tout $\theta \in \Theta$,
\[
\hat{g}_n \xrightarrow[\mathbb{P}_\theta]{n \to +\infty} g(\theta).
\]
Elle est dite \textbf{fortement consistante} si pour tout $\theta \in \Theta$,
\[
\hat{g}_n \xrightarrow[\text{p.s.}]{n \to +\infty} g(\theta).
\]
\end{definition}

\begin{remarque}
On dira seulement que $\hat{g}_n$ est consistant si $(\hat{g}_n)_{n \in \mathbb{N}}$ est consistante.

\medskip

\begin{itemize}
    \item La convergence doit avoir lieu pour tout $\theta \in \Theta$.
    \item La consistance est une propriété ``faible''. S'il existe un estimateur consistant, alors il en existe une infinité : si $\hat{g}_n$ est consistant et si $(a_n)_{n \in \mathbb{N}}$ est telle que $a_n \to 1$, alors $a_n \hat{g}_n$ est consistant.
    \item En revanche, les estimateurs non consistants sont à exclure.
\end{itemize}
\end{remarque}


\begin{definition}[Estimateur asymptotiquement sans biais]
On suppose que pour tout $\theta \in \Theta$ et pour tout $n \in \mathbb{N}$,
\[
\mathbb{E}_\theta\left[\|\hat{g}_n\|\right] < +\infty.
\]
On dit que $\hat{g}_n$ est \textbf{asymptotiquement sans biais} si pour tout $\theta \in \Theta$,
\[
\mathbb{E}_\theta[\hat{g}_n] \xrightarrow{n \to +\infty} g(\theta).
\]
\end{definition}


\begin{definition}[Normalité asymptotique]
On dit que $\hat{g}_n$ est \textbf{asymptotiquement normal} si pour tout $\theta \in \Theta$, il existe $\Sigma(\theta) > 0$ tel que
\[
\sqrt{n}\left(\hat{g}_n - g(\theta)\right) \xrightarrow[\text{loi}]{n \to +\infty} \mathcal{N}(0, \Sigma(\theta)).
\]
\end{definition}

\begin{remarque}
\begin{itemize}
    \item Si $\hat{g}_n$ est asymptotiquement normal, alors $\hat{g}_n$ est consistant.
    \item Plus généralement, on peut remplacer ``$\sqrt{n}$'' par $a_n$ où $a_n \xrightarrow{n \to +\infty} +\infty$. On dit alors que $\frac{1}{a_n}$ est la vitesse de convergence asymptotique.
\end{itemize}
\end{remarque}


\begin{proposition}
Soit $\hat{g}_n$ un estimateur asymptotiquement normal de $g(\theta)$ de variance asymptotique $\Sigma(\theta)$.
Soit $\hat{\Sigma}_n$ un estimateur de $\Sigma(\theta)$ consistant, c'est-à-dire pour tout $\theta \in \Theta$,
\[
\hat{\Sigma}_n \xrightarrow[\mathbb{P}_\theta]{n \to +\infty} \Sigma(\theta).
\]
Alors
\[
\frac{\sqrt{n}\left(\hat{g}_n - g(\theta)\right)}{\sqrt{\hat{\Sigma}_n}} \xrightarrow[\text{loi}]{n \to +\infty} \mathcal{N}(0, 1).
\]
\end{proposition}

\begin{proof}
Conséquence du \textbf{lemme de Slutsky}.

On écrit
\[
\frac{\sqrt{n}\left(\hat{g}_n - g(\theta)\right)}{\sqrt{\hat{\Sigma}_n}} = \frac{\sqrt{n}\left(\hat{g}_n - g(\theta)\right)}{\sqrt{\Sigma(\theta)}} \cdot \frac{\sqrt{\Sigma(\theta)}}{\sqrt{\hat{\Sigma}_n}},
\]
avec :
\begin{itemize}
    \item $\displaystyle\frac{\sqrt{n}\left(\hat{g}_n - g(\theta)\right)}{\sqrt{\Sigma(\theta)}} \xrightarrow[\text{loi}]{n \to +\infty} \mathcal{N}(0, 1)$ par hypothèse,
    \item $\displaystyle\frac{\sqrt{\Sigma(\theta)}}{\sqrt{\hat{\Sigma}_n}} \xrightarrow[\mathbb{P}_\theta]{n \to +\infty} 1$ par consistance de $\hat{\Sigma}_n$.
\end{itemize}
Le lemme de Slutsky permet de conclure.
\end{proof}

\subsection{Estimation de $\theta$ à partir d'un estimateur de $g(\theta)$}

\textbf{Question :} $\hat{g}_n$ estimateur de $g(\theta)$. Comment estimer $\theta$ ? Et quelles propriétés pour l'estimateur ?

\begin{proposition}[Méthode delta inverse]
Soit $\Theta$ un intervalle ouvert de $\mathbb{R}$.
\begin{itemize}
    \item Soit $\hat{g}_n$ estimateur de $g(\theta)$ tel que pour tout $\theta \in \Theta$,
    \[
    \sqrt{n}\left(\hat{g}_n - g(\theta)\right) \xrightarrow[\text{loi}]{n \to +\infty} \mathcal{N}(0, \sigma^2).
    \]
    \item On suppose que $g : \Theta \to g(\Theta)$ est un $\mathcal{C}^1$-difféomorphisme (c'est-à-dire $g$ bijective, $\mathcal{C}^1$, et $g^{-1} \in \mathcal{C}^1$). On définit :
    \[
    \hat{\theta}_n = g^{-1}(\hat{g}_n).
    \]
\end{itemize}
Alors,
\begin{itemize}
    \item $\hat{\theta}_n$ est consistant pour l'estimation de $\theta$.
    \item Pour tout $\theta \in \Theta$,
    \[
    \sqrt{n}\left(\hat{\theta}_n - \theta\right) \xrightarrow[\text{loi}]{n \to +\infty} \frac{\sigma}{|g'(\theta)|} \mathcal{N}(0, 1).
    \]
\end{itemize}
\end{proposition}

\begin{proof}
\begin{itemize}
    \item $g$ continue et bijective $\Theta \to g(\Theta)$ donc $g(\Theta)$ est un intervalle ouvert.
    \[
    \mathbb{P}\left(\hat{g}_n \in g(\Theta)\right) \xrightarrow{n \to +\infty} 1.
    \]
    Donc $\hat{\theta}_n$ est bien défini (avec une probabilité qui tend vers 1).
    
    \item \textbf{Consistance de $\hat{\theta}_n$ :} $g^{-1}$ continue et $\hat{g}_n$ consistant donc par composition, $\hat{\theta}_n = g^{-1}(\hat{g}_n)$ est consistant.
    
    \item \textbf{$\hat{\theta}_n$ asymptotiquement normal :} méthode delta appliquée à $g^{-1}$.
    
    On a $g^{-1}(g(\theta)) = \theta$ et pour rappel,
    \[
    (g^{-1})'(y) = \frac{1}{g'(g^{-1}(y))}.
    \]
    Donc
    \[
    (g^{-1})'(g(\theta)) = \frac{1}{g'(g^{-1}(g(\theta)))} = \frac{1}{g'(\theta)}.
    \]
    Par la méthode delta,
    \[
    \sqrt{n}\left(\hat{\theta}_n - \theta\right) = \sqrt{n}\left(g^{-1}(\hat{g}_n) - g^{-1}(g(\theta))\right) \xrightarrow[\text{loi}]{n \to +\infty} (g^{-1})'(g(\theta)) \mathcal{N}(0, \sigma^2) = \frac{1}{g'(\theta)} \mathcal{N}(0, \sigma^2).
    \]
    Ce qui équivaut à
    \[
    \sqrt{n}\left(\hat{\theta}_n - \theta\right) \xrightarrow[\text{loi}]{n \to +\infty} \mathcal{N}\left(0, \frac{\sigma^2}{(g'(\theta))^2}\right) = \frac{\sigma}{|g'(\theta)|} \mathcal{N}(0, 1).
    \]
\end{itemize}
\end{proof}

\section{Méthode des moments}


\begin{definition}[Estimateur par la méthode des moments]

\begin{itemize}
    \item Soit un échantillon $X_1, \ldots, X_n$ de loi $\mathbb{P}_\theta$, $\theta \in \Theta$.
    \item Soit $\phi$ une fonction telle que $\mathbb{E}_\theta\left[\|\phi(X_1)\|\right] < +\infty$.
\end{itemize}
On souhaite estimer
\[
g(\theta) = \mathbb{E}_\theta[\phi(X_1)].
\]
On appelle \textbf{estimateur par la méthode des moments} :
\[
\hat{g}_n = \frac{1}{n} \sum_{i=1}^n \phi(X_i).
\]
\end{definition}



\begin{notation}
\begin{itemize}
    \item $\hat{g}_n$ est consistant (fortement) par la loi des grands nombres.
    \item $\hat{g}_n$ est sans biais (par linéarité de l'espérance).
    \item Plus généralement, si $\hat{g}_n$ est construit par la méthode des moments comme ci-dessus, et si $h$ est une fonction, on dira aussi que
    \[
    h(\hat{g}_n)
    \]
    est un estimateur par la \textbf{méthode des moments} pour l'estimation de $h(g(\theta))$.
\end{itemize}
\end{notation}


\begin{remarque}
À priori, on ne connaît pas les propriétés de cet estimateur (biais, variance, normalité asymptotique). Il faudra étudier ces propriétés au cas par cas, éventuellement en utilisant la méthode delta.
\end{remarque}

\begin{exemple}
\begin{enumerate}
    \item \textbf{Probabilité d'un événement}
    
    Soit $A$ un événement. On souhaite estimer
    \[
    g(\theta) = \mathbb{P}_\theta(X_1 \in A) = \mathbb{E}_\theta[\mathbb{1}_{X_1 \in A}],
    \]
    où
    \[
    \mathbb{1}_{X_1 \in A} = \begin{cases}
    1 & \text{si } X_1 \in A, \\
    0 & \text{sinon}.
    \end{cases}
    \]
    
    Par la méthode des moments :
    \[
    \hat{g}_{A,n} = \frac{1}{n} \sum_{i=1}^n \mathbb{1}_{X_i \in A}.
    \]
    
    Cet estimateur est la fréquence empirique de l'événement $A$.
    
    \item \textbf{Loi uniforme}
    
    Soit $X_i$ pour $i \in [ 1, n ]$ de loi $\mathcal{U}([\theta-1 , \theta+1])$ avec $\theta \in \mathbb{R}$.
    
    Alors
    \[
    \mathbb{E}_\theta[X_1] = \theta = g(\theta).
    \]
    
    Par la méthode des moments :
    \[
    \hat{g}_n = \overline{X}_n = \frac{1}{n} \sum_{i=1}^n X_i,
    \]
\end{enumerate}
\end{exemple}






\chapter{Méthode du maximum de vraisemblance}

On se place dans le cadre de l'estimation de $g(\theta) = \theta$.

\begin{definition}[Vraisemblance]
On appelle \textbf{vraisemblance} associée à l'observation $X$, la fonction :

\begin{align*}
L &: \Theta \to \mathbb{R}, \\
&\theta \mapsto L(\theta; X)
\end{align*}

définie par :
\begin{itemize}
    \item si $X$ est une variable aléatoire \textbf{discrète}, alors :
    \[
    L(\theta; X) = p_\theta(X)
    \quad \text{où} \quad
    p_\theta(x) = \mathbb{P}_\theta(X = x)
    \]
    
    \item si $X$ est une variable aléatoire \textbf{à densité}, alors :
    \[
    L(\theta; X) = f_\theta(X)
    \quad \text{où } f_\theta \text{ est la densité de } X.
    \]
\end{itemize}
\end{definition}

\begin{remarque}
Si $X = (X_1, \ldots, X_n)$ est un échantillon de variables aléatoires i.i.d., alors :
\begin{itemize}
    \item si les $X_i$ sont discrètes :
    \[
    L(\theta; X) = \prod_{i=1}^n p_\theta(X_i)
    \]
    \item si les $X_i$ admettent une densité :
    \[
    L(\theta; X) = \prod_{i=1}^n f_\theta(X_i)
    \]
\end{itemize}
\end{remarque}













\begin{exemple}
On considère une variable aléatoire $X \sim \mathcal{B}(15, p)$ avec $p$ inconnu.

On observe $x = 5$.

\medskip
La fonction de vraisemblance est :
\[
L(p; 5) = \mathbb{P}_p(X = 5)
= \binom{15}{5} p^5 (1-p)^{10}
\]

C’est la probabilité d’observer la valeur $5$ lorsque le paramètre est $p$.

\medskip
\begin{center}
\begin{tabular}{c|ccccc}
$p$ & 0,1 & 0,2 & 0,3 & 0,4 & 0,5 \\
\hline
$L(p;5)$ & 0,01 & 0,10 & 0,21 & 0,19 & 0,09
\end{tabular}
\end{center}

\begin{itemize}
    \item Quand $p = 0{,}1$, il y a environ 1 chance sur 100 d’observer $x = 5$.
    \item Quand $p = 0{,}3$, il y a environ 21 chances sur 100 d’observer $x = 5$.
\end{itemize}

Il est donc plus \og vraisemblable \fg{} que la vraie valeur de $p$ soit proche de $0{,}3$ plutôt que de $0{,}1$.

\medskip
En suivant ce raisonnement, la valeur de $p$ la plus vraisemblable est celle qui maximise la probabilité d’observer $x=5$, c’est-à-dire qui maximise la vraisemblance.

\medskip
On se place, pour simplifier, sur la fonction :
\[
\ell(p) = \ln L(p; 5)
\]

Alors :
\[
\frac{d\ell(p)}{dp}
= \frac{x}{p} - \frac{15 - x}{1 - p}
= \frac{x - 15p}{p(1 - p)}
\]

Cette dérivée s’annule en :
\[
p = \frac{x}{15}
\]

Ici, avec $x = 5$, on obtient :
\[
\hat{p} = \frac{1}{3}
\]

Ainsi, la valeur de $p$ la plus vraisemblable au vu de l’observation $x = 5$ est :
\[
\boxed{p = \frac{1}{3}}
\]
\end{exemple}

\begin{definition}[Estimateur du maximum de vraisemblance]
On appelle \textbf{estimateur du maximum de vraisemblance (EMV)} de $\theta$ toute quantité satisfaisant :
\[
\hat{\theta}^{MV} \in \underset{\theta \in \Theta}{\mathrm{argmax}} \; L(\theta; X)
\]

Souvent on considère la log-vraisemblance.

\[ \hat{\theta} \in \arg \max_{\theta \in \Theta} L(\theta,X) \]

où $l(\theta,X) = \log L(\theta,X)$
\end{definition}

\begin{exemple}
On a $P_\theta = \mathcal{N}(0,1)$

EMV : $\hat{\theta} = \bar{X}_n$

On dira que l'EMV de $\theta^2$ est $(\bar{X}_n)^2$
\end{exemple}

\begin{remarque}
L'inconvénient de cette méthode

\begin{itemize}
\item Il peut ne pas être unique
\item Ne pas exister
\item peut être difficile à expliciter
\end{itemize}
\end{remarque}


\end{document}

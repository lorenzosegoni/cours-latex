\chapter*{Exercice}

\exo

\begin{enonce}
Chercher tous les points d'équilibre du système d'équations différentielles:

\[
\begin{cases}
 \frac{dx}{dt}=(x-1)(y-1) \\
\frac{dy}{dt}=(x+1)(y+1)
\end{cases}
\]
\end{enonce}
\begin{correction}
On cherche les points d'équilibre 

\[
\begin{cases}
x' = (x-1)(y-1) \\
y' = (x+1)(y+1)
\end{cases}
\]

On a donc au point d'équilibre $P_{eq} = (x_{eq},y_{eq}) \in \mathbb{R}^2$ si et seulement si,

\[
\begin{cases}
(x_{eq}-1)(y_{eq}-1) =0 \\
(x_{eq}+1)(y_{eq}+1) =0
\end{cases}
\]

Donc on obtient que les points possibles sont $(1,-1)$ ou $(-1,1)$.

On vérifie

\[ (1,-1) : \begin{cases}
x' = (1-1)(-1-1) = 0 \\
y' = (1+1)(-1+1) = 0
\end{cases} \quad
(-1,1) : \begin{cases}
x' = (-1-1)(1-1) = 0 \\
y' = (-1+1)(1+1) = 0
\end{cases} 
\]

On a donc les points d'équilibres $P_{eq}^{(1)} = (1,-1)$ et $P_{eq}^{(2)} = (-1,1)$.
\end{correction}

\exo

\begin{enonce}
Soit $A = \begin{pmatrix} 2 & -3 & 0 \\ 0 & -6 & -2 \\ -6 & 0 & -3 \end{pmatrix}$. Montrer que la solution nulle de l'équation différentielle $\dot{x} = Ax$ est instable.
\end{enonce}
\begin{correction}
On a le
\end{correction}

\exo

\begin{enonce}
Déterminer pour chacun des systèmes diférentiels suivants, les points d'équilibre ainsi que leur comportement:

\ques

\[
\begin{cases}
x' = xy-y \\
y' = xy-x
\end{cases}
\]

\ques

\[
\begin{cases}
x' = x-xy \\
y' = y-x^2
\end{cases}
\]

\ques

\[
\begin{cases}
x' = 3x-2y \\
y' = 4x+y
\end{cases}
\]

Résoudre le système linéaire.
\end{enonce}

\begin{correction}
\ques On cherche les points d'équilibre 

\[
\begin{cases}
x' = xy-y \\
y' = xy-x
\end{cases}
\]

On a donc au point d'équilibre $P_{eq} = (x_{eq},y_{eq}) \in \mathbb{R}^2$ si et seulement si,

\[
\begin{cases}
x_{eq}y_{eq}-y_{eq} = 0 \\
x_{eq}y_{eq}-x_{eq} = 0
\end{cases}
\Rightarrow
\begin{cases}
y_{eq}(x_{eq}-1) = 0 \\
x_{eq}(y_{eq}-1) = 0
\end{cases}
\]

Donc on obtient que les points possibles sont $(1,0)$ ou $(0,1)$ ou $(0,0)$ ou $(1,1)$

On vérifie

\[ (1,0) : \begin{cases}
x' = 1 \cdot 0 - 0 = 0 \\
y' = 1 \cdot 0 - 1 = -1
\end{cases} \quad
(0,1) : \begin{cases}
x' = 0 \cdot 1 - 1 = -1 \\
y' = 0 \cdot 1 - 0 = 0
\end{cases} 
\]
\[
(0,0) : \begin{cases}
x' = 0 \cdot 0 - 0 = 0 \\
y' = 0 \cdot 0 - 0 = 0
\end{cases} \quad
(1,1) : \begin{cases}
x' = 1 \cdot 1 - 1 = 1 \\
y' = 1 \cdot 1 - 1 = 1
\end{cases} 
\]

On a donc les points d'équilibres $P_{eq}^{(1)} = (0,0)$ et $P_{eq}^{(2)} = (1,1)$.

------------------------------------------------------------------------------------------------------------------------

Soit $A(P(x,y))$ la matrice jacobienne du système linéaire au tour du point $P(x,y)$, alors:

\[
A(P(x,y))=
\begin{pmatrix}
\frac{\partial x'}{\partial x} & \frac{\partial x'}{\partial y} \\
\frac{\partial y'}{\partial x} & \frac{\partial y'}{\partial y}
\end{pmatrix}=
\begin{pmatrix}
y & x-1\\
y-1 & x
\end{pmatrix}
\]

\begin{enumerate}
\item On regarde $P_{eq}^{(1)} = (0,0)$.

\[ A_1 = A(P_{eq}^{(1)}) = \begin{pmatrix}
0 & -1 \\
-1 & 0
\end{pmatrix} \]

On a le système d'équilibre $X' = A_1 X$. Le point d'équilibre de ce système est $Q_{eq}^{(1)} = (0,0)$. 

On regarde le spectre de $A_1$:

\[ \chi_{\lambda}(A_1) = A_1 - \lambda I_2 = \lambda^2 - Tr(A_1) \lambda + det(A_1) = \lambda^2 - 0 \lambda -1 =0 \Rightarrow Sp(A_1) = \{ -1 , +1 \} \]

On constate que $\forall \lambda \in Sp(A_1) , \lambda = \{ -1 , +1 \}$, on a $Re(\lambda) \neq 0$ Donc $Q_{eq}^{(1)}$ est hyperbolique pour le $X' = A_1 X$. Par hyperbolité le point d'équilibre $P_{eq}^{(1)}$ est donc hyperbolique.

Pour $Q_{eq}^{(1)}$, on a $\lambda_1 = -1$ et $ \lambda _2 = 1 > 0$, par le théorème de la stabilité des système lineaire permet de qualifier que $Q_{eq}^{(1)}$ est instable et par hyperbolité, on conclut que $P_{eq}^{(1)}$ est instable.
\item On regarde $P_{eq}^{(2)} = (1,1)$.
\[ A_1 = A(P_{eq}^{(2)}) = \begin{pmatrix}
1 & 0 \\
0 & 1
\end{pmatrix} = I_2 \]

On a le système d'équilibre $X' = A_2 X$. Le point d'équilibre de ce système est $Q_{eq}^{(2)} = (0,0)$. 

On regarde le spectre de $A_2$:

\[ \chi_{\lambda}(A_2) = A_2 - \lambda I_2 =0 \Rightarrow Sp(A_2) = \{ +1 \} \quad \text{avec multiplicité $2$} \]

On constate que $\forall \lambda \in Sp(A_2), \lambda = \{ +1 \}$, on a $Re(\lambda) \neq 0$ Donc $Q_{eq}^{(2)}$ est hyperbolique pour le $X' = A_2 X$. Par hyperbolité le point d'équilibre $P_{eq}^{(2)}$ est donc hyperbolique.

Pour $Q_{eq}^{(2)}$, on a $\lambda_1 = 1 >0$ et $\lambda_2 = 1 >0$, par le théorème de la stabilité des système lineaire permet de qualifier que $Q_{eq}^{(2)}$ est instable et par hyperbolité, on conclut que $P_{eq}^{(2)}$ est instable.
\end{enumerate}

\ques On cherche les points d'équilibre 

\[
\begin{cases}
x' = x-xy \\
y' = y-x^2
\end{cases}
\]

On a donc au point d'équilibre $P_{eq} = (x_{eq},y_{eq}) \in \mathbb{R}^2$ si et seulement si,

\[
\begin{cases}
x_{eq}-x_{eq}y_{eq} = 0 \\
y_{eq}-x_{eq}^2 = 0
\end{cases}
\Rightarrow
\begin{cases}
x_{eq}(1- y_{eq}) = 0 \\
y_{eq} = x_{eq}^2
\end{cases}
\]

Donc on obtient que les points possibles sont $(0,0)$ ou $(1,1)$ ou $(-1,1)$ ou $(1,1)$

On vérifie

\[ (0,0) : \begin{cases}
x' = 0 \cdot 0 - 0 = 0 \\
y' = 1  - 0 = 0
\end{cases} \quad
(1,1) : \begin{cases}
x' = 1 \cdot 1 - 1 = 0 \\
y' = 1 - 1 = 0
\end{cases} 
\]
\[
(-1,1) : \begin{cases}
x' = -1 \cdot 1 +1 = 0 \\
y' = 1 - 1 = 0
\end{cases}
\]

On a donc les points d'équilibres $P_{eq}^{(1)} = (0,0)$ et $P_{eq}^{(2)} = (1,1)$ et $P_{eq}^{(3)} = (-1,1)$ .

------------------------------------------------------------------------------------------------------------------------

Soit $A(P(x,y))$ la matrice jacobienne du système linéaire au tour du point $P(x,y)$, alors:

\[
A(P(x,y))=
\begin{pmatrix}
\frac{\partial x'}{\partial x} & \frac{\partial x'}{\partial y} \\
\frac{\partial y'}{\partial x} & \frac{\partial y'}{\partial y}
\end{pmatrix}=
\begin{pmatrix}
1-y & -x\\
-2x & 1
\end{pmatrix}
\]

\begin{enumerate}
\item On regarde $P_{eq}^{(1)} = (0,0)$.

\[ A_1 = A(P_{eq}^{(1)}) = \begin{pmatrix}
1 & 0 \\
0 & 1
\end{pmatrix} = I_2 \]

On a le système d'équilibre $X' = A_1 X$. Le point d'équilibre de ce système est $Q_{eq}^{(1)} = (0,0)$. 

On regarde le spectre de $A_1$:

\[ \chi_{\lambda}(A_1) = A_1 - \lambda I_2 =0 \Rightarrow Sp(A_1) = \{ +1 \} \quad \text{avec multiplicité $2$} \]

On constate que $\forall \lambda \in Sp(A_1) , \lambda = \{ +1 \}$, on a $Re(\lambda) \neq 0$ Donc $Q_{eq}^{(1)}$ est hyperbolique pour le $X' = A_1 X$. Par hyperbolité le point d'équilibre $P_{eq}^{(1)}$ est donc hyperbolique.

Pour $Q_{eq}^{(1)}$, on a $\lambda_1 = 1 >1$ et $ \lambda _2 = 1 > 0$, par le théorème de la stabilité des système lineaire permet de qualifier que $Q_{eq}^{(1)}$ est instable et par hyperbolité, on conclut que $P_{eq}^{(1)}$ est instable.
\item On regarde $P_{eq}^{(2)} = (1,1)$.
\[ A_1 = A(P_{eq}^{(2)}) = \begin{pmatrix}
0 & -1 \\
-2 & 1
\end{pmatrix} = I_2 \]

On a le système d'équilibre $X' = A_2 X$. Le point d'équilibre de ce système est $Q_{eq}^{(2)} = (0,0)$. 

On regarde le spectre de $A_2$:

\[ \chi_{\lambda}(A_2) = A_2 - \lambda I_2 =\lambda^2 - Tr(A_2) \lambda + det(A_2)= \lambda^2 - \lambda -1 = 0 \Rightarrow Sp(A_2) = \{ -1 , 2  \}  \]

On constate que $\forall \lambda \in Sp(A_2) , \lambda=  \{ -1 , 2  \}$, on a $Re(\lambda) \neq 0$ Donc $Q_{eq}^{(2)}$ est hyperbolique pour le $X' = A_2 X$. Par hyperbolité le point d'équilibre $P_{eq}^{(2)}$ est donc hyperbolique.

Pour $Q_{eq}^{(2)}$, on a $\lambda_1 = -1 < 0$ et $\lambda_2 =  2 >0$, par le théorème de la stabilité des système lineaire permet de qualifier que $Q_{eq}^{(2)}$ est instable et par hyperbolité, on conclut que $P_{eq}^{(2)}$ est instable.




\item On regarde $P_{eq}^{(3)} = (-1,1)$.
\[ A_1 = A(P_{eq}^{(3)}) = \begin{pmatrix}
0 & 1 \\
2 & 1
\end{pmatrix} = I_2 \]

On a le système d'équilibre $X' = A_3 X$. Le point d'équilibre de ce système est $Q_{eq}^{(3)} = (0,0)$. 

On regarde le spectre de $A_3$:

\[ \chi_{\lambda}(A_3) = A_3 - \lambda I_2 =\lambda^2 - Tr(A_3) \lambda + det(A_3)= \lambda^2 - \lambda -2 = 0 \Rightarrow Sp(A_3) = \{ -1,2 \}  \]

On constate que $\forall \lambda \in Sp(A_3) =   \{ -1,2 \} $, on a $Re(\lambda) \neq 0$ Donc $Q_{eq}^{(3)}$ est hyperbolique pour le $X' = A_3 X$. Par hyperbolité le point d'équilibre $P_{eq}^{(3)}$ est donc hyperbolique.

Pour $Q_{eq}^{(3)}$, on a $Re(\lambda_1) = -1 $ et $Re(\lambda_2) =  2 >0$, par le théorème de la stabilité des système lineaire permet de qualifier que $Q_{eq}^{(3)}$ est instable et par hyperbolité, on conclut que $P_{eq}^{(3)}$ est instable.
\end{enumerate}

\ques On cherche les points d'équilibre 

\[
\begin{cases}
x' = 3x-2y \\
y' = 4x+y
\end{cases}
\]

On a donc au point d'équilibre $P_{eq} = (x_{eq},y_{eq}) \in \mathbb{R}^2$ si et seulement si,

\[
\begin{cases}
3x_{eq}-2y_{eq} = 0 \\
4 x_{eq}+y_{eq} = 0
\end{cases}
\Rightarrow
\begin{cases}
x_{eq}= \frac{2}{3} y_{eq} \\
4 \cdot \frac{2}{3} y_{eq} +y_{eq} = \frac{11}{3} y_{eq} = 0
\end{cases}
\]

Donc on obtient que les points possibles sont $(0,0)$ .

On vérifie

\[ (0,0) : \begin{cases}
x' = 3 \cdot 0 - 2 \cdot 0 = 0 \\
y' = 4 \cdot 0  - 1 \cdot 0 = 0
\end{cases} \]

On a donc les points d'équilibres $P_{eq} = (0,0)$ .

------------------------------------------------------------------------------------------------------------------------

Soit $A(P(x,y))$ la matrice jacobienne du système linéaire au tour du point $P(x,y)$, alors:

\[
A(P(x,y))=
\begin{pmatrix}
\frac{\partial x'}{\partial x} & \frac{\partial x'}{\partial y} \\
\frac{\partial y'}{\partial x} & \frac{\partial y'}{\partial y}
\end{pmatrix}=
\begin{pmatrix}
3 & -2 \\
4 & 1
\end{pmatrix} = A
\]

On regarde le spectre:

\[ \chi_{\lambda}(A) = A - \lambda I_2 = \lambda^2 - Tr(A) \lambda + det(A) =  \lambda^2 - 4 \lambda +11  = 0 \Rightarrow Sp(A) = \{ 2-i \sqrt{7} , 2+i\sqrt{7} \}  \]

Puisque $\forall \lambda \in Sp(A) , Re(\lambda) = 2 > 0 $ On a des points d'équilibre hyperbolique instable. 

------------------------------------------------------------------------------------------------------------------------





\bigskip
\bigskip
\bigskip
\bigskip
\bigskip
\bigskip
\bigskip
\bigskip
\bigskip
\bigskip
\bigskip
\bigskip
\bigskip
\bigskip
\bigskip
\bigskip
\bigskip
\bigskip
\bigskip
\bigskip
\bigskip
\bigskip
\bigskip
\bigskip
\bigskip
\bigskip
\bigskip
\bigskip
\bigskip
\bigskip
\bigskip
\bigskip
\bigskip
\bigskip
\bigskip
\bigskip
\bigskip
\bigskip
\bigskip
\bigskip
\bigskip
\bigskip
\bigskip
\bigskip


On peut écrire l'equation: 

\[
\begin{cases}
x' = 3x-2y \\
y' = 4x+y
\end{cases} \Rightarrow
X' = \begin{pmatrix}
3 & -2 \\ 
4 & 1
\end{pmatrix}
X = AX \quad \text{ avec } X = \begin{pmatrix} x \\ y \end{pmatrix}
\]

Comme $Sp(A)=\{ 2-i \sqrt{7} , 2+i\sqrt{7} \}$, on a que:


\[ A - Re(\lambda) I_2 = N \Rightarrow N = \begin{pmatrix} 1 & -2 \\ 4 & -1 \end{pmatrix} \]

De plus on remarque que:

\[ N^2 = = \begin{pmatrix} -7 & 0 \\ 0 & -7 \end{pmatrix} = - Im^2(\lambda) I_2 \]

Donc on a:

\begin{align*}
e^{At} &= e^{(2 I_2 + N) t} \\
&= e^{2t} e^{Nt} \\
&= e^{2t} \cdot \left( \cos(\sqrt{7} t) I_2 + \frac{\sin(\sqrt{7} t)}{\sqrt{7}} N \right) \\
&= e^{2t} \begin{pmatrix} \cos(\sqrt{7} t) + \frac{1}{\sqrt{7}} \sin(\sqrt{7} t ) & -\frac{2}{\sqrt{7}} \sin(\sqrt{7} t) \\ 
\frac{4}{\sqrt{7}} \sin(\sqrt{7} t) &  \cos(\sqrt{7} t) - \frac{1}{\sqrt{7}} \sin(\sqrt{7} t) \end{pmatrix}
\end{align*}

Puisque la solution générale est $X(t) = e^{At} X_0$:

\[ \begin{cases}
x(t) = e^{2t} \left( ( \cos(\sqrt{7} t) + \frac{1}{\sqrt{7}} \sin(\sqrt{7} t) ) x_0 - \frac{2}{\sqrt{7}} \sin(\sqrt{7} t) y_0  \right) \\
y(t) = e^{2t} \left( \frac{4}{\sqrt{7}} \sin(\sqrt{7} t) x_0 + ( \cos(\sqrt{7} t) - \frac{1}{\sqrt{7}} \sin(\sqrt{7} t) ) y_0 \right)
\end{cases} \]

\end{correction}













\exo

\begin{enonce}
Soit le système différentiel suivants:

\[
\begin{cases}
x'(t) = x(4-x-2y) \\
y'(t) = y(x-y-1) \\
x(0) \geq 0 \quad , \quad y(0)\geq 0
\end{cases}
\]

\ques Déterminer les ponts d'équilibre de ce système, puis la matrice jacobienne du système linéarité en chacun de ces points.

\ques Pour chacun de ces points d'équilibre, dire s'il est hyperbolique ou non.

\ques Pour chaque point d'équilibre hyperbolique, étudier sa stabilité.

\end{enonce}
\begin{correction}
\end{correction}

\exo

\begin{enonce}
Soient $D = \{ (u,v) \in \mathbb{R}^2 , \quad u \geq 0, v \geq 0 \} $, le premier quadrant de $\mathbb{R}^2$ et le système différentiel suivant :

\[ \begin{cases} 
x' = -axy-bx+c
y' = axy-dy
a >0 \quad b>0 \quad d>0 \\
c \geq 0 \\
( x(0) , y(0) ) \in D
\end{cases} \]

\ques Déterminer les points d'équilibre de ce système.
\ques Ces points d'équilibre existe-t-ils toujours dans $D$? Sinon donner une condition nécessaire d'existence dans $D$. Lorsque l'existence dans $D$ de l'un d'entre eux n'est pas toujours assurée, exprimer cette condition (d'existence dans $D$) à l'aide d'une relation faisant intervenir une quantité que l'on notera $\mathcal{R}_0$.
\ques Pour chaque point d'équilibre, quand il existe dans $D$, dire s'il est hyperbolique ou non.
\ques Pour chaque point d'équilibre hyperbolique de $D$ étudier sa stabilité.
\ques Ces points d'équilibre lorsqu'ils existent dans $D$ peuvent-ils être conjointement stables?
\ques Etudier le cas $c=0$.

\end{enonce}
\begin{correction}
\end{correction}

\exo

\begin{enonce}
Soit le systéme différentiel suivant:

\[ \begin{cases}
x'(t) = - y - x \sqrt{x^2 + y^2} \\
y'(t) = x - y \sqrt{x^2 + y^2} \\
x(0) \quad. y(0) \quad \text{ donnés}
 \end{cases} \]

\ques Montrer que ce systéme admet un unique point d'équilibre $P* = (x*,y*)$ de $\mathbb{R}^2$.
\ques Etudier la stabilité du point $0$, point d'équilibre du système linéarisé autour de $P*$. Peut-on conclure quant à la stabilité de $P*$?
\ques Soit $d(P(t),P*)$, la distance euclidienne entre le point $P(t) = (x(t),y(t))$ solution du système donné et $P*$. C'est une fonction numérique de la variable $t$ d'expression

\[ d(P(t),P*) = \sqrt{ ( x(t) - x*)^2 + ( y(t) - y*)^2 } \]

Etudier les variations de la fonction $d$ puis en déduire la stabilité du point $P$.
\end{enonce}
\begin{correction}
\end{correction}

\exo

\begin{enonce}
Soit $A = \begin{pmatrix} 0 & -3 \\ 2 & 0 \end{pmatrix}$
\ques Montrer que la solution nulle de l'équation différentielle $\dot{x} = Ax$ est stable mais pas asymptotiquement stable.
\ques Trouver l'espace vectoriel des solutions.
\end{enonce}
\begin{correction}
\ques On a le polynôme caractéristique:

\[ \chi_A(\lambda) = \lambda^2 + 6 \Rightleftarrow Sp(A) = \{ -i\sqrt{6} , i \sqrt{6} \} \]

Les valeurs propres sont purement imaginaires.

\commentaire{On peut remarque que $A^2=-6 I_2$}





La solution générale s’écrit alors :
\[
e^{At} = \cos(\omega t)\,I + \frac{\sin(\omega t)}{\omega}\,A.
\]

\smallskip
Soit $X(0)=X_0=(x_0,y_0)^T$. On a donc :
\[
X(t) = e^{At} X_0
= \Big( \cos(\omega t)\,I + \frac{\sin(\omega t)}{\omega}A \Big) X_0.
\]

En composantes :
\[
\begin{cases}
x(t) = \cos(\omega t)\,x_0 - \dfrac{3}{\omega}\sin(\omega t)\,y_0, \\[6pt]
y(t) = \dfrac{2}{\omega}\sin(\omega t)\,x_0 + \cos(\omega t)\,y_0,
\end{cases}
\qquad \text{avec } \omega = \sqrt{6}.
\]

\smallskip
\textbf{Stabilité.}  
On remarque que la quantité
\[
Q(X) = 2x^2 + 3y^2
\]
est conservée le long des trajectoires, car
\[
\frac{d}{dt} Q(X(t)) = X(t)^T (A^T M + M A) X(t) = 0,
\]
avec $M = \begin{pmatrix} 2 & 0 \\ 0 & 3 \end{pmatrix}$.

Ainsi, $Q(X(t)) = Q(X(0))$ pour tout $t$.  
Cela montre que si $X(0)$ est proche de $0$, alors $X(t)$ reste proche de $0$ pour tout $t$ :  
l’origine est donc \textbf{stable au sens de Lyapunov}.

\smallskip
\textbf{Non-asymptotique.}  
Cependant, si $X(0) \neq 0$, alors $Q(X(t)) = Q(X(0)) > 0$ pour tout $t$ :  
les trajectoires ne s’approchent jamais de l’origine.  
Les solutions sont périodiques, donc la solution nulle est \textbf{stable mais non asymptotiquement stable}.

\bigskip
\textbf{2) Espace vectoriel des solutions.}

L’espace des solutions est :
\[
\mathcal{S} = \{\, X : \mathbb{R} \to \mathbb{R}^2 \mid X(t) = e^{At}v, \; v \in \mathbb{R}^2 \,\}.
\]
C’est un espace vectoriel de dimension $2$.

En prenant $v = e_1 = (1,0)^T$ et $v = e_2 = (0,1)^T$, on obtient une base de $\mathcal{S}$ :
\[
\begin{aligned}
u_1(t) &= e^{At} e_1 =
\begin{pmatrix}
\cos(\omega t) \\[4pt]
\dfrac{2}{\omega}\sin(\omega t)
\end{pmatrix}, \\[10pt]
u_2(t) &= e^{At} e_2 =
\begin{pmatrix}
-\dfrac{3}{\omega}\sin(\omega t) \\[6pt]
\cos(\omega t)
\end{pmatrix}.
\end{aligned}
\]

Toute solution réelle s’écrit donc :
\[
X(t) = c_1\,u_1(t) + c_2\,u_2(t),
\qquad (c_1,c_2)\in\mathbb{R}^2.
\]

Les solutions sont périodiques de période :
\[
T = \frac{2\pi}{\omega} = \frac{2\pi}{\sqrt{6}}.
\]

\textbf{Conclusion :} l’origine est un \emph{centre} — stable mais non asymptotiquement stable —,  
et les solutions décrivent des ellipses centrées en l’origine.






\end{correction}

\exo

\begin{enonce}
Soient $f,g \in \mathcal{C}^1 (\mathbb{R}^2)$ et le système différentiel suivant:

\[
\begin{cases}
x'(t) = f(x,y) \\
y'(t) = g(x,y) \\
x(0) \quad , \quad y(0) \quad \text{donnés}
\end{cases}
\]

Montrer que si $\lim_{t\to \infty} (x(t),y(t)) = (x_\infty,y_\infty) \in \mathbb{R}^2$, alors $(x_\infty,y_\infty)$ est un point d'équilibre de ce système différentiel.

\end{enonce}
\begin{correction}
\end{correction}
\section{Bases de Données (BD) et SGBD}

\subsection{Introduction : La problématique des fichiers}

Avant l'invention des bases de données (années 1960), l'informatique reposait sur des systèmes de gestion de fichiers classiques. Cette méthode présentait des limites majeures qui ont conduit à la création des SGBD

\subsubsection{Limite des systèmes de gestion de fichiers}

Lorsqu'une application gère ses données via de simples fichiers, nous rencontrons trois problèmes fondamentaux :

\begin{itemize}
\item \textbf{La Redondance des données :} Les mêmes informations sont souvent répétées dans plusieurs fichiers pour différentes applications.

\textit{Conséquence :} Gaspillage d'espace et risque d'incohérence (si on modifie une info à un endroit mais pas à l'autre).
\item \textbf{La Dépendance Programmes / Données :} La structure des données est "codée en dur" dans le programme.

\textit{Conséquence :} Si l'on change l'organisation physique d'un fichier, il faut réécrire tous les programmes qui l'utilisent. C'est une gestion complexe.
\item \textbf{La Gestion des accès :} Il est difficile de permettre à plusieurs utilisateurs d'accéder et de modifier le même fichier en même temps sans créer de conflits.
\end{itemize}

\subsection{Définitions Fondamentales}

Il est crucial de ne pas confondre le contenu (la base) et le contenant/gestionnaire (le système).


\subsubsection{Base de Données (BD)}

\begin{definition}
Une Base de Données est une collection de données représentant des informations du monde réel.
\end{definition}

Pour être qualifiée de BD, cette collection doit respecter quatre critères :

\begin{itemize}
\item \textbf{Cohérence et Structure :} Les données suivent un schéma logique défini.
\item \textbf{Indépendance :} Les données existent indépendamment des applications qui les utilisent.
\item \textbf{Non-redondance :} On évite de stocker deux fois la même information (redondance minimale).
\item \textbf{Accessibilité :} Les données sont accessibles par plusieurs utilisateurs simultanément.
\end{itemize}

\subsubsection{Système de Gestion de Base de Données (SGBD)}

\begin{definition}
Le SGBD est le logiciel qui sert d'interface entre les utilisateurs (ou applications) et la base de données.
\end{definition}

Ses rôles principaux sont :

\begin{itemize}
\item La structuration des données.
\item Le stockage physique.
\item La mise à jour et la consultation.
\end{itemize}

\textbf{Exemples de domaines d'application :} Gestion d'entreprise, systèmes transactionnels (banques), e-commerce, bibliothèques numériques, etc.

\subsection{Objectifs d'une approche Base de Données}

L'utilisation d'un SGBD vise à résoudre les problèmes des systèmes de fichiers (vus en partie 1) en atteignant les objectifs suivants :

\begin{itemize}
\item \textbf{Indépendance Physique et Logique :} Le changement de la structure interne des données ou de leur stockage physique ne doit pas impacter les programmes. C'est l'objectif le plus important.
\item \textbf{Manipulation aisée :} Permettre à des non-informaticiens d'interroger et de mettre à jour les données facilement.
\item \textbf{Partage et Sécurité :} Plusieurs applications peuvent utiliser les mêmes données sans conflit.
\item \textbf{Performance :} Garantir une efficacité d'accès (temps de réponse rapide) même avec de gros volumes de données.
\end{itemize}

\subsection{Les Fonctions du SGBD}
Pour atteindre ces objectifs, le SGBD offre quatre fonctions techniques majeures. Il est important de bien distinguer les deux langages (LDD et LMD).

\subsubsection{Le Langage de Définition des Données (LDD)}
Il permet de décrire la \textbf{structure} de la base (le squelette).

\begin{itemize}
\item \textbf{Rôle :} Définir les objets (tables), leurs attributs (colonnes), les liens entre eux et les contraintes.
\item \textbf{Résultat :} On obtient le Schéma de la Base de Données.
\end{itemize}

\subsubsection{Le Langage de Manipulation des Données (LMD)}
Il permet de gérer le \textbf{contenu} de la base (les données elles-mêmes).

\begin{itemize}
\item \textbf{Rôle :} Créer, modifier, supprimer ou consulter des données.
\item \textbf{Outil :} C'est ici qu'intervient le langage SQL.
\end{itemize}

\subsubsection{Le Contrôle de l'intégrité}
Le SGBD s'assure que les données respectent les règles définies (par le schéma ou le programme). Il empêche l'insertion de données aberrantes.

\subsubsection{La Sécurité de fonctionnement}
Le SGBD gère les aspects critiques de l'exploitation :

\begin{itemize}
\item \textbf{Les Transactions et la journalisation :} Assurer que si une opération plante au milieu, on peut revenir en arrière (rollback) pour ne pas corrompre la base.
\item \textbf{Les Accès concurrents :} Gérer plusieurs utilisateurs en même temps.
\item \textbf{La Confidentialité :} Gérer les droits d'accès (qui a le droit de voir quoi).
\end{itemize}

%Image

\newpage

\section{Le Modèle Conceptuel de Données (E/A)}

Le modèle Entité / Association

\subsection{Le Principe du Modèle E/A}

Avant de créer des tables dans l'ordinateur, il faut dessiner le schéma sur papier.

\begin{itemize}
\item \textbf{Origine :} Proposé par Peter Chen en 1976.
\item \textbf{Objectif :} C'est une représentation graphique standardisée pour décrire les données d'un Système d'Information (SI).
\item \textbf{Utilité :} Il sert de pont. Une fois le modèle E/A terminé, il est très facile de le traduire en tables SQL.
\end{itemize}

\subsection{Les Composants Fondamentaux}

Pour dessiner ce modèle, nous avons besoin de trois briques de base : l'Entité, l'Attribut et l'Identifiant.

\subsubsection{L'Entité (L'objet)}

\begin{definition}
Une entité est un objet (concret ou abstrait) à propos duquel on souhaite gérer des informations.
\end{definition}

Il ne faut pas confondre le "moule" et l'objet créé :

\begin{enumerate}
\item \textbf{Type d'entité (Le moule) :} C'est la classe générale, le concept.

\textit{Exemple :} L'entité \texttt{Étudiant}, \texttt{Client}, \texttt{Département}.

\item \textbf{Occurrence d'entité (L'individu) :} C'est un élément précis, un individu spécifique qui appartient à ce type.

\textit{Exemple :} L'employé \texttt{Alex Térieur} ou \texttt{Paul Auchon}.
\end{enumerate}

\subsubsection{Les Attributs (Les détails)}

\begin{definition}
Ce sont les propriétés qui décrivent une entité (ou une association).
\end{definition}

Chaque attribut possède :

\begin{enumerate}
\item \textbf{Un Nom :} (ex: \texttt{Nom}, \texttt{Prix}, \texttt{Couleur}).
\item \textbf{Un Domaine :} L'ensemble des valeurs possibles (ex: Entier, Réel positif, Chaîne de caractères, liste de choix {Rouge, Vert, Bleu}).
\item \textbf{Une Occurrence :} La valeur précise pour un individu (ex: "Rouge" est une occurrence de l'attribut \texttt{Couleur}).
\end{enumerate}

\textbf{Représentation Graphique :} Dans le schéma, l'Entité est un Rectangle et les Attributs sont listés à l'intérieur (ou dans des bulles reliées au rectangle).

\medskip

\begin{tikzpicture}[node distance=3cm]

% Style pour les entités
\tikzstyle{entity} = [rectangle, draw, minimum width=2cm, minimum height=1cm, text centered]
\tikzstyle{attribute} = [ellipse, draw, minimum width=1.2cm, minimum height=0.6cm, text centered, font=\small]
\tikzstyle{relation} = [diamond, draw, minimum width=1.5cm, minimum height=1cm, text centered]

% Entité Employé au centre
\node[entity] (employe) {Employé};

% Attributs de Employé
\node[attribute, above of=employe] (nom) {nom};
\node[attribute, above left of=employe, xshift=-1.5cm] (prenom) {prenom};
\node[attribute, above right of=employe, xshift=1.5cm] (datedenaissance) {date\_\-naissance};
\node[attribute, left of=employe] (rue) {rue};
\node[attribute, right of=employe] (codepostal) {code\_postal};
\node[attribute, below left of=employe, xshift=-1.5cm] (ville) {ville};

% Connexions entités-attributs
\draw (employe) -- (nom);
\draw (employe) -- (prenom);
\draw (employe) -- (datedenaissance);
\draw (employe) -- (rue);
\draw (employe) -- (codepostal);
\draw (employe) -- (ville);

\end{tikzpicture}

Que on peut representer aussi:

\begin{table}[h]
\centering
\begin{tabular}{|c|}
\hline
\textbf{Entité : Employé}  \\
\hline
 nom  \\
 prénom   \\
date de naissance \\
rue \\
code postal  \\
ville  \\
\hline
\end{tabular}
\caption{ Tableau Exemple }
\end{table}

\subsubsection{L'Identifiant (La clé)}

C'est le concept le plus important pour retrouver une info précise.

\begin{definition}
L'identifiant est l'ensemble minimal d'attributs qui permet de distinguer de façon unique chaque occurrence.
\end{definition}

\textbf{Évolution de l'identifiant (Exemple du cours) :}

\begin{enumerate}
\item \textbf{Mauvaise pratique :} Utiliser \{\texttt{Nom}, \texttt{Prénom}, \texttt{Date de naissance} \}. C'est lourd et il y a toujours un risque d'homonyme parfait.

\begin{table}[h]
\centering
\begin{tabular}{|c|}
\hline
\textbf{Entité : Employé}  \\
\hline
\underline{nom}  \\
\underline{prénom}  \\
\underline{date de naissance} \\
rue \\
code postal  \\
ville  \\
\hline
\end{tabular}
\caption{ Tableau avec la mauvaise pratique }
\end{table}
\item \textbf{ Bonne pratique (Identifiant artificiel) :} On ajoute un attribut dédié, souvent souligné dans le schéma.

\textit{ Exemple : } On ajoute \texttt{Numero\_Employe}.

\smallskip

Dans le schéma graphique, l'identifiant est toujours souligné.

\begin{table}[h]
\centering
\begin{tabular}{|c|}
\hline
\textbf{Entité : Employé}  \\
\hline
\underline{Numero\_Employé} \\
nom  \\
prénom  \\
date de naissance \\
rue \\
code postal  \\
ville  \\
\hline
\end{tabular}
\caption{ Tableau avec la bonne pratique }
\end{table}
\end{enumerate}

\subsection{Les Associations (Les liens)}

Les données ne vivent pas seules, elles sont reliées entre elles.

\subsubsection{Définition}

\begin{definition}
Une Association est un lien sémantique entre plusieurs entités. Elle est souvent représentée par un verbe.
\end{definition}

\begin{itemize}
\item \textit{Exemple :}

\begin{tikzpicture}[node distance=3cm]

% Style pour les entités
\tikzstyle{entity} = [rectangle, draw, minimum width=2.5cm, minimum height=1cm, text centered, font=\bfseries]
\tikzstyle{relation} = [diamond, draw, minimum width=2cm, minimum height=1.2cm, text centered, font=\bfseries]

% Entité Client (gauche)
\node[entity] (client) {Client};

% Relation Commander (centre)
\node[relation, right of=client, xshift=3.5cm] (commander) {Commander};

% Entité Produit (droite)
\node[entity, right of=commander, xshift=3.5cm] (produit) {Produit};

% Liaisons entités-relations avec cardinalités et étiquettes
% Client vers Commander
\draw (client) -- (commander) 
    node[midway, above] {1} 
    node[midway, below, font=\small, color=blue] {a commandé};

% Commander vers Produit
\draw (commander) -- (produit) 
    node[midway, above] {N} 
    node[midway, below, font=\small, color=blue] {est commandé par};

\end{tikzpicture}

\item \textbf{Attributs d'association :} Parfois, une donnée n'appartient ni à l'un, ni à l'autre, mais au lien lui-même.

\textit{Exemple :} La \texttt{Quantité} (des produits commandé) et la \texttt{Date} (de la commande). Elles n'existent que parce qu'il y a une commande entre le client et le produit.

\begin{tikzpicture}[node distance=3cm]

% Style pour les entités
\tikzstyle{entity} = [rectangle, draw, minimum width=2.5cm, minimum height=1cm, text centered, font=\bfseries]
\tikzstyle{relation} = [diamond, draw, minimum width=2cm, minimum height=1.2cm, text centered, font=\bfseries]
\tikzstyle{attribute} = [ellipse, draw, minimum width=1.2cm, minimum height=0.6cm, text centered, font=\small]

% Entité Client (gauche)
\node[entity] (client) {Client};

% Relation Commander (centre)
\node[relation, right of=client, xshift=3.5cm] (commander) {Commander};

% Entité Produit (droite)
\node[entity, right of=commander, xshift=3.5cm] (produit) {Produit};

% Attributs de la relation Commander
\node[attribute, above of=commander] (date) {date};
\node[attribute, below of=commander] (qte) {qté};

% Liaisons entités-relations
\draw (client) -- (commander) 
    node[midway, above] {1} 
    node[midway, below, font=\small, color=blue] {a commandé};

\draw (commander) -- (produit) 
    node[midway, above] {N} 
    node[midway, below, font=\small, color=blue] {est commandé par};

% Liaisons relation-attributs
\draw (commander) -- (date);
\draw (commander) -- (qte);

\end{tikzpicture}
\end{itemize}

\subsubsection{Typologie des Associations}

\begin{itemize}
\item \textbf{Binaire :} Relie 2 entités (le plus courant).
\item \textbf{Ternaire :} Relie 3 entités.
\item \textbf{Réflexive :} Une entité est reliée à elle-même (ex: Un employé est marié à un autre employé).
\end{itemize}

\subsubsection{Les Cardinalités (La règle du jeu)}

\begin{definition}
Les cardinalités définissent les règles de quantité dans une association. Elles s'écrivent sous la forme (min, max) à côté de chaque entité.
\end{definition}

\begin{tikzpicture}[node distance=3cm]

% Style pour les entités
\tikzstyle{entity} = [rectangle, draw, minimum width=2.5cm, minimum height=1cm, text centered, font=\bfseries]
\tikzstyle{relation} = [diamond, draw, minimum width=2cm, minimum height=1.2cm, text centered, font=\bfseries]

% Entité 1 (gauche)
\node[entity] (entite1) {Identité1};

% Relation (centre)
\node[relation, right of=entite1, xshift=3.5cm] (assoc) {Assoc};

% Entité 2 (droite)
\node[entity, right of=assoc, xshift=3.5cm] (entite2) {Identité2};

% Liaisons entités-relations avec cardinalités (min,max)
% De Identité1 à Assoc
\draw (entite1) -- (assoc) 
    node[midway, above, font=\small] {1} 
    node[midway, below, font=\small] {(min1, max1)};

% De Assoc à Identité2
\draw (assoc) -- (entite2) 
    node[midway, above, font=\small] {2} 
    node[midway, below, font=\small] {(min2, max2)};

\end{tikzpicture}

\subsubsection{ Comprendre le (min, max)}

\begin{enumerate}
\item \textbf{Min (0 ou 1) :} Est-ce que l'entité est obligée de participer ?

\begin{itemize}
\item 0 = Non (Optionnel).
\item 1 = Oui (Obligatoire).
\end{itemize}
\item \textbf{Max (1 ou N) :} Combien de fois maximum peut-elle participer ?

\begin{itemize}
\item 1 = Une seule fois.
\item N (ou M) = Plusieurs fois (No limit).
\end{itemize}
\end{enumerate}

\subsubsection{ Les 3 grands types de relations }
On classe les associations selon leur cardinalité \textbf{maximale} (le chiffre de droite) des deux côtés :

\begin{enumerate}
\item \textbf{Relation 1-1 (One-to-One) :}

Une occurrence de E est liée à \textbf{une seule} occurrence de F, et inversement.
\item \textbf{Relation 1-N (One-to-Many) :}

D'un côté, c'est unique (1), de l'autre c'est multiple (N).

\medskip

\textit{Exemple :} Un \texttt{Auteur} écrit plusieurs \texttt{Livres}, mais un \texttt{Livre} est écrit par un seul \texttt{Auteur} (dans ce modèle simplifié).

\item \textbf{Relation N-M (Many-to-Many) :}

Plusieurs des deux côtés.

\medskip

\textit{Exemple :} Un \texttt{Client} commande plusieurs \texttt{Produits}, et un \texttt{Produit} peut être commandé par plusieurs \texttt{Clients}.
\end{enumerate}

\subsection{ Concepts Complémentaires }

\subsubsection{ Entités Faibles }

\begin{definition}
Une entité qui ne possède pas d'identifiant propre.
\end{definition}

\begin {itemize}
\item Elle ne peut exister que si elle est rattachée à une "Entité Forte".
\item Son identifiant est composé de celui de l'entité forte + un identifiant partiel.
\end{itemize}

La cardinalité vers l'entité forte est toujours \textbf{(1,1)} (dépendance totale).

\subsubsection{Contraintes d'Intégrité (CI)}
Ce sont des règles pour garantir que les données restent logiques.

\begin{itemize}
\item \textbf{CI Statiques :} Doivent être vraies tout le temps (ex: Le \texttt{Nom} est obligatoire, \texttt{Date naissance} < \texttt{Date mariage} , \texttt{Fax} attributs facultatifs, ... ).
\item \textbf{CI Dynamiques :} Règles logiques sur des valeurs (ex: le \texttt{salaire} ne peut qu'augmenter).
\end{itemize}

\subsection{ Formalisation et Documentation }

Pour qu’un projet de base de données soit valide, il ne suffit pas de faire un dessin (le schéma). Il faut respecter des \textbf{règles de complétude} : chaque objet dessiné doit être décrit textuellement de manière exhaustive.

L'ensemble de ces descriptions constitue \textbf{l'Univers du Discours} (le résumé synthétique de l'application).

\subsubsection{ La Description des Objets (Entités et Associations) }

Chaque élément graphique doit avoir sa fiche d'identité textuelle.


\begin{table}[h]
\centering
\begin{tabular}{|c|c|}
\hline
\textbf{Nom} & \texttt{Auteur} \\
\hline
\textbf{Définition (Contexte)} & Personne ayant écrit un livre référencé par l'éditeur \\
\hline
\textbf{Liste d'attributs} & \{nom, prénom, adresse\} \\
\hline
\textbf{Identifiant} & \{nom, prénom\} \\
\hline
\end{tabular}
\caption{ Description d'une Entité }
\end{table}


\subsubsection{ Description d'une Association}

Prenons l'association \texttt{Écriture} qui relie les auteurs aux livres.

\begin{table}[h]
\centering
\begin{tabular}{|c|c|}
\hline
\textbf{Nom} & \texttt{Ecriture} \\
\hline
\textbf{Définition} & L'écriture associe les livres à l'auteur qui les a écrits \\
\hline
\textbf{Entités} & Auteur, Livre \\
\hline
\textbf{Rôles \& Cardinalités} & Un Auteur écrit $1$ à $N$ Livres. Un Livre est écrit par $1$ à $1$ Auteur (1-1) \\
\hline
\textbf{Attributs propres} & $\emptyset$ (Aucun attribut sur l'association elle-même) \\
\hline
\end{tabular}
\caption{ Description d'une Association }
\end{table}

\subsubsection{ Le Dictionnaire des Données }

Le dictionnaire des données descend au niveau le plus fin : \textbf{l'attribut}. Il précise le format et les règles de chaque donnée.

\begin{table}[h]
\centering
\begin{tabular}{|c|c|}
\hline
\textbf{Nom} & \texttt{Ville auteur} \\
\hline
\textbf{Définition} & Nom de la ville où réside un auteur. \\
\hline
\textbf{Structure} & Atomique (Mono-valué) \\
\hline
\textbf{Rôles / cardinalité} & Chaîne de caractères alphabétiques. \\
\hline
\textbf{Obligatoire ?} & Non. On peut créer un auteur sans connaître sa ville. \\
\hline
\end{tabular}
\caption{ Description d'un Attribut }
\end{table}

\subsubsection{ Les Contraintes d'Intégrité (CI) }

Certaines règles logiques ne peuvent pas être dessinées sur le schéma. Il faut les écrire sous forme d'expressions logiques ou mathématiques.

\begin{table}[h]
\centering
\begin{tabular}{|c|c|}
\hline
\textbf{Nom de la contrainte} & Existence d'un mariage \\
\hline
\textbf{Éléments concernés} & Association : \underline{\texttt{Mariage}}, Attribut :`\texttt{Âge de l'entité} et \underline{\texttt{Personne}}. \\
\hline
\textbf{Expression logique (La règle)} & Une occurrence de l'association \texttt{Mariage} n'est valide que si :$Age(P_1) \geq 18$ et $Age(P_2) \geq 18$ \\
\hline
\hline
\end{tabular}
\caption{ Contrainte sur une Association }
\end{table}
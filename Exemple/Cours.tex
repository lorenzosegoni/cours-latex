\documentclass[12pt,a4paper]{book}

% ===== ENCODAGE ET LANGUE =====
\usepackage[utf8]{inputenc}
\usepackage[T1]{fontenc}
\usepackage[french]{babel}
\usepackage{lmodern}

% ===== MISE EN PAGE =====
\usepackage[left=2.5cm, right=2.5cm, top=2.5cm, bottom=2.5cm]{geometry}
\usepackage{setspace}
\onehalfspacing

% ===== MATHÉMATIQUES =====
\usepackage{amsmath, amssymb, amsthm}
\usepackage{mathtools}
\usepackage{bm}

% ===== FIGURES, TABLEAUX, CODE =====
\usepackage{graphicx}
\usepackage{caption}
\usepackage{subcaption}
\usepackage{float}
\usepackage{booktabs}
\usepackage{array}
\usepackage{multirow}
\usepackage{xcolor}
\usepackage{listings}

\lstloadlanguages{Python, R, SQL}

% ===== BOÎTES MINIMALISTES =====
\usepackage{mdframed}

% ===== STRUCTURE ET NAVIGATION =====
\usepackage{titlesec}
\usepackage{titletoc}
\usepackage{fancyhdr}
\setlength{\headheight}{14.5pt}
\usepackage{hyperref}
\hypersetup{
    colorlinks=true,
    linkcolor=black,
    citecolor=blue,
    urlcolor=blue,
    bookmarks=true,
    bookmarksopen=true
}

% ===== LISTES PERSONNALISÉES =====
\usepackage{enumitem}
\setlist[itemize]{itemsep=0.3em, topsep=0.5em}
\setlist[enumerate]{itemsep=0.3em, topsep=0.5em}

% ===== EN-TÊTES ET PIEDS =====
\pagestyle{fancy}
\fancyhf{}
\fancyhead[L]{\leftmark}
\fancyhead[R]{\thepage}
\renewcommand{\headrulewidth}{0.4pt}

% ===== NUMÉROTATION DES ÉQUATIONS =====
\numberwithin{equation}{chapter}

% ===== DÉFINITION STYLES DE LISTINGS =====
\definecolor{codebg}{RGB}{250, 250, 250}
\definecolor{outputbg}{RGB}{240, 240, 240}

\lstset{
    basicstyle=\ttfamily\small,
    breaklines=true,
    numbers=left,
    numberstyle=\tiny\color{gray},
    numbersep=8pt,
    backgroundcolor=\color{codebg}
}

\lstdefinestyle{output}{
    backgroundcolor=\color{outputbg},
    basicstyle=\ttfamily\small,
    frame=leftline,
    framerule=2pt,
    rulecolor=\color{gray!70}
}

% ===== COMMANDES PERSONNALISÉES MATHS =====
\newcommand{\dd}{\mathrm{d}}
\newcommand{\R}{\mathbb{R}}
\newcommand{\N}{\mathbb{N}}
\newcommand{\Z}{\mathbb{Z}}
\newcommand{\Q}{\mathbb{Q}}
\newcommand{\C}{\mathbb{C}}

% ===== COULEURS DOUCES =====
\definecolor{redsoft}{RGB}{230, 100, 100}
\definecolor{redbg}{RGB}{255, 245, 245}
\definecolor{yellowsoft}{RGB}{220, 180, 80}
\definecolor{yellowbg}{RGB}{255, 252, 240}
\definecolor{graysoft}{RGB}{140, 140, 140}
\definecolor{graybg}{RGB}{248, 248, 248}

% ===== STYLES DES BOÎTES =====
\mdfdefinestyle{thmstyle}{
    linecolor=redsoft,
    linewidth=1.5pt,
    backgroundcolor=redbg,
    innerleftmargin=8pt,
    innerrightmargin=8pt,
    innertopmargin=6pt,
    innerbottommargin=6pt,
    skipabove=0.8em,
    skipbelow=0.8em
}

\mdfdefinestyle{exstyle}{
    linecolor=yellowsoft,
    linewidth=1.5pt,
    backgroundcolor=yellowbg,
    innerleftmargin=8pt,
    innerrightmargin=8pt,
    innertopmargin=6pt,
    innerbottommargin=6pt,
    skipabove=0.8em,
    skipbelow=0.8em
}

\mdfdefinestyle{remstyle}{
    linecolor=graysoft,
    linewidth=1.5pt,
    backgroundcolor=graybg,
    innerleftmargin=8pt,
    innerrightmargin=8pt,
    innertopmargin=6pt,
    innerbottommargin=6pt,
    skipabove=0.8em,
    skipbelow=0.8em
}

% ===== ENVIRONNEMENTS =====
\theoremstyle{definition}
\newmdtheoremenv[style=thmstyle]{definition}{Définition}[chapter]
\newmdtheoremenv[style=thmstyle]{notation}{Notation}[chapter]
\newmdtheoremenv[style=exstyle]{exemple}{Exemple}[chapter]
\newmdtheoremenv[style=remstyle]{remarque}{Remarque}[chapter]
\newmdtheoremenv[style=remstyle]{note}{Note}[chapter]

\theoremstyle{plain}
\newmdtheoremenv[style=thmstyle]{theoreme}{Théorème}[chapter]
\newmdtheoremenv[style=thmstyle]{proposition}{Proposition}[chapter]
\newmdtheoremenv[style=thmstyle]{lemme}{Lemme}[chapter]
\newmdtheoremenv[style=thmstyle]{corollaire}{Corollaire}[chapter]

% ===== DOCUMENT =====
\begin{document}

\frontmatter
\title{Exemple Cours}
\author{Lorenzo Segoni}
\date{\today}
\maketitle

\tableofcontents

\mainmatter

% Exemple de chapitre
\chapter{Introduction}

Votre contenu ici.

\section{Première section}

Texte introductif de la section.

\begin{definition}
Ceci est une définition avec boîte minimaliste rouge.
\end{definition}

Un peu de texte après.

\begin{theoreme}
Ceci est un théorème avec la même mise en évidence.
\end{theoreme}

\begin{proposition}
Une proposition importante.
\end{proposition}

\begin{exemple}
Voici un exemple avec boîte jaune doux.
\end{exemple}

Du texte normal après.

\begin{remarque}
Une remarque que vous placez où vous voulez, avec boîte grise.
\end{remarque}

\section{Deuxième section}

\begin{lemme}
Lemme important.
\end{lemme}

\begin{corollaire}
Corollaire du lemme précédent.
\end{corollaire}

\begin{exemple}
Autre exemple pour illustrer.
\end{exemple}

% Exemple d'équation
\begin{equation}
e = mc^2
\end{equation}

% Exemple de tableau
\begin{table}[H]
\centering
\begin{tabular}{@{}lcc@{}}
\toprule
Colonne 1 & Colonne 2 & Colonne 3 \\
\midrule
Données 1 & 10 & 20 \\
Données 2 & 30 & 40 \\
\bottomrule
\end{tabular}
\caption{Exemple de tableau}
\end{table}

% Exemple de figure
\begin{figure}[H]
\centering
% \includegraphics[width=0.6\textwidth]{votre-image.pdf}
\caption{Exemple de figure}
\end{figure}

% PYTHON
\begin{lstlisting}[language=Python, caption=Exemple Python]
def calcul(x):
    return x**2 + 3*x + 1
\end{lstlisting}

% OUTPUT
\begin{lstlisting}[caption=Output]
>>> resultat
41
\end{lstlisting}

% R
\begin{lstlisting}[language=r, caption=Exemple R]
calcul <- function(x) {
  return(x^2 + 3*x + 1)
}
\end{lstlisting}

% OUTPUT
\begin{lstlisting}[style=output]
[1] 41
\end{lstlisting}

% SQL
\begin{lstlisting}[language=SQL, caption=Exemple SQL]
SELECT nom, age FROM utilisateurs WHERE age > 18;
\end{lstlisting}

% OUTPUT
\begin{lstlisting}[style=output]
nom        | age
-----------|----
Alice      | 28
Bob        | 25
\end{lstlisting}

\backmatter

% Décommenter si vous activez natbib
% \bibliography{votre-biblio}

\end{document}

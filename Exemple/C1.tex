\chapter{Introduction}

Votre contenu ici.

\section{Première section}

Texte introductif de la section.

\begin{definition}
Ceci est une définition avec boîte minimaliste rouge.
\end{definition}

Un peu de texte après.

\begin{theoreme}
Ceci est un théorème avec la même mise en évidence.
\end{theoreme}

\begin{proposition}
Une proposition importante.
\end{proposition}

\begin{exemple}
Voici un exemple avec boîte jaune doux.
\end{exemple}

Du texte normal après.

\begin{remarque}
Une remarque que vous placez où vous voulez, avec boîte grise.
\end{remarque}

\section{Deuxième section}

\begin{lemme}
Lemme important.
\end{lemme}

\begin{corollaire}
Corollaire du lemme précédent.
\end{corollaire}

\begin{exemple}
Autre exemple pour illustrer.
\end{exemple}

% Exemple d'équation
\begin{equation}
e = mc^2
\end{equation}

% Exemple de tableau
\begin{table}[H]
\centering
\begin{tabular}{@{}lcc@{}}
\toprule
Colonne 1 & Colonne 2 & Colonne 3 \\
\midrule
Données 1 & 10 & 20 \\
Données 2 & 30 & 40 \\
\bottomrule
\end{tabular}
\caption{Exemple de tableau}
\end{table}

% Exemple de figure
\begin{figure}[H]
\centering
\includegraphics[width=0.6\textwidth]{votre-image.jpg}
\caption{Exemple de figure}
\end{figure}

% Python - entrée
\begin{pycodebox}
\begin{lstlisting}[language=Python]
def calcul(x):
    return x**2 + 3*x + 1
\end{lstlisting}
\end{pycodebox}

% Python - output (fond plus clair)
\begin{pyoutputbox}
\begin{lstlisting}
>>> calcul(4)
41
\end{lstlisting}
\end{pyoutputbox}

% R - entrée
\begin{rcodebox}
\begin{lstlisting}[language=R]
calcul <- function(x){
  return(x^2 + 3*x + 1)
}
\end{lstlisting}
\end{rcodebox}


% R - output
\begin{routputbox}
\begin{lstlisting}
[1] 41
\end{lstlisting}
\end{routputbox}


% SQL - entrée
\begin{sqlcodebox}
\begin{lstlisting}[language=SQL]
SELECT nom, age
FROM utilisateurs
WHERE age > 18;
\end{lstlisting}
\end{sqlcodebox}

% SQL - output
\begin{sqloutputbox}
\begin{lstlisting}
nom   | age
------+-----
Alice | 28
Bob   | 25
\end{lstlisting}
\end{sqloutputbox}

\exo
\begin{enonce}
Montrer que $f(x)=x^2$ est croissante sur $[0,+\infty[$.
\end{enonce}

\begin{correction}
$f'(x)=2x \ge 0$ sur $[0,+\infty[$, donc $f$ est croissante.
\end{correction}


\backmatter

% Décommenter si vous activez natbib
% \bibliography{votre-biblio}
